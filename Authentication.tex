% This is the LaTeX file for the chapter:
%   Authentication
% for the manual:
%   Highly Secure Systems
% Written by Andrew Ensor
% Last updated 5 June 2006.

\chapter{Authentication}

%%%%%%%%%%%%%%%%%%%%%%%%%%%%%%%%%%%%%%%%%%%%%%%%%%%%%%%%%%%%%%%%%%%%%%%%%%%%%%%%%%%%%%%%%
%%%%% Message Authentication                                                        %%%%%
%%%%%%%%%%%%%%%%%%%%%%%%%%%%%%%%%%%%%%%%%%%%%%%%%%%%%%%%%%%%%%%%%%%%%%%%%%%%%%%%%%%%%%%%%

\section{Message Authentication}\label{Se:MessageAuthentication}

\reading{pp318-344}

Symmetric and public-key encryption both provide confidentiality of messages,
ensuring that only the intended recipient can decrypt the ciphertext and
obtain the original plaintext. However, unless the recipient knows the format of the
plaintext to expect, such as English text, a recognizable image,
or the response of a nonce value,
the recipient can not be sure of the actual source of a particular message,
or whether it may have been tampered with en route.
The difficulty stems from the inability of a recipient to determine what is valid plaintext
from a block that was encrypted with an incorrect key.
Some file formats start with a recognizable header, but this might not
inhibit an adversary replacing later blocks of ciphertext with their own blocks
in an attempt to disrupt the system (in fact it might be better to put recognizable
information at the end of the file rather than at the start if CBC mode encryption were used).

The verification of the integrity of a message is known as
\defn{message authentication}, ensuring that the message did originate
from the alleged source and has not been altered in any way.
This is achieved by adding extra information to the message calculated from its contents.
One simple approach is to append an error-detecting code known as a
\defn{frame check sequence} (FCS) or \defn{checksum} to the message before the
message is encrypted, either with a secret key for a symmetric cipher or with the private key
of a public-key cipher of the sender. When the recipient decrypts the message the FCS can be
recalculated and compared with the value sent with the message.
If the two FCS values agree then the recipient presumes the message was not
altered between when the two values were calculated.
So long as it is certain that the key used for encryption was kept confidential
it can be sure that the message was encrypted by the alleged source.

More generally, a \defn{(cryptographic) hash code} or \defn{message digest} is a fixed-length
value computed for a message which acts as a fingerprint for it.
Ideally, a function $h$ used to calculate hash codes should be able to be applied
to messages of arbitrary lengths and be relatively quick to compute.
For security, a hash function is designed to be \defn{one-way}, meaning
that given a hash code $y$ it is computationally infeasible to create a message
$M$ which has the hash code $y$ ($h(M)=y$).
Hash functions do not need to be one-to-one functions, so it is possible that there are
two messages $M$ and $M^\prime$ which have the same hash code ($h(M^\prime)=h(M)$),
in which case a \defn{collision} is said to occur.
They should however have \defn{weak collision resistance}, meaning that
given a message $M$ it is computationally infeasible to create another message
$M^\prime$ which has the same hash code ($h(M^\prime)=h(M)$).

Once a suitable hash function has been devised, it can be used for the
authentication of a message.
If the message itself should remain confidential then the hash code
can be appended to the plaintext before it all gets encrypted as a single message.
Then when the intended recipient decrypts the message it can check whether the appended
hash code matches that of the decrypted message, and so determine whether the message
was authentic.

Often however, only authentication of a message is required, as the message
itself might need to be readable by all processes in the system (such as a
broadcast) or else confidentiality might
already be provided by the communication protocol.
In this case just the hash code itself needs to be encrypted with the private key
of the sender, and the encrypted hash code appended to the plaintext message.
As only the holder of the key can produce the valid encrypted hash code,
any recipient can decrypt the hash code to determine whether the
message $M$ was actually created by the holder of the private key.
It is important to note that if an adversary could devise a message $M^\prime$
with exactly the same hash code as the original message $M$,
then $M$ could be substituted by $M^\prime$ and appended
with the original encrypted hash code without the encryption ever needing to be broken,
resulting in a forged message.
Thus it is essential for authenticity that the hash function have weak collision resistance.

One simple hash code is the \defn{longitudinal redundancy check}, which calculates
an $n$-bit hash code by breaking a message $M$ into blocks $B_1, B_2, \dots, B_m$
each of length $n$, padding with zeros if required and taking a bitwise
XOR of each block:
\begin{displaymath}
  h(M) = B_1 \oplus B_2 \oplus \cdots \oplus B_m.
\end{displaymath}
Often there are regularities in the blocks, such as for ASCII strings where the
most-significant bit of each byte is typically $0$.
To avoid such regularities reducing the possible hash values the hash code might be
rotated one bit before each XOR is calculated.
However, the longitudinal redundancy check does not have weak collision resistance,
given a message $M$ it is not difficult to modify it slightly into a message $M^\prime$
(perhaps changing the amount for an on-line payment or a name) while still
maintaining the same hash code.
Even if the hash code were encrypted with the plaintext message using
a secure symmetric block cipher with CBC mode this particular hash function might not be able
to detect a reordering of the ciphertext blocks by an adversary.

Most contemporary hash functions make use of multiple rounds in which the
message is used block by block $B_1, B_2, \dots, B_m$ to calculate its hash code.
\begin{figure*}[htb]\begin{center}
\begin{pspicture}(0,0)(11.5,2)
  \rput(2,1.75){\rnode{B1}{$B_1$}}
  \rput(1,0.35){\rnode{IV}{\scriptsize{\begin{tabular}{c}initialization\\vector\end{tabular}}}}
  \psline[fillstyle=solid,fillcolor=yellow](2.5,0)(2.5,1.5)(3.25,1.15)(3.25,0.35)(2.5,0)
  \pnode(2.5,1.15){C11}\pnode(2.5,0.35){C12}\pnode(3.25,0.75){C13}

  \rput(4,1.75){\rnode{B2}{$B_2$}}
  \psline[fillstyle=solid,fillcolor=yellow](4.5,0)(4.5,1.5)(5.25,1.15)(5.25,0.35)(4.5,0)
  \pnode(4.5,1.15){C21}\pnode(4.5,0.35){C22}\pnode(5.25,0.75){C23}

  \rput(6,1.75){\rnode{B3}{$B_3$}}
  \psline[fillstyle=solid,fillcolor=yellow](6.5,0)(6.5,1.5)(7.25,1.15)(7.25,0.35)(6.5,0)
  \pnode(6.5,1.15){C31}\pnode(6.5,0.35){C32}\pnode(7.25,0.75){C33}

  \rput(8.5,0.35){\rnode{dots}{\dots}}
  \rput(7.5,1.75){\rnode{dots2}{\dots}}

  \rput(9,1.75){\rnode{Bm}{$B_m$}}
  \psline[fillstyle=solid,fillcolor=yellow](9.5,0)(9.5,1.5)(10.25,1.15)(10.25,0.35)(9.5,0)
  \pnode(9.5,1.15){Cm1}\pnode(9.5,0.35){Cm2}\pnode(10.25,0.75){Cm3}

  \rput(11.25,0.75){\rnode{out}{$h(M)$}}

  \psset{arrows=->}
  \ncangle[angleA=270,angleB=180]{B1}{C11}\ncline{IV}{C12}
  \ncangle[angleA=270,angleB=180]{B2}{C21}\ncangles[angleA=0,angleB=180]{C13}{C22}
  \ncangle[angleA=270,angleB=180]{B3}{C31}\ncangles[angleA=0,angleB=180]{C23}{C32}
  \ncangles[angleA=0,angleB=180]{C33}{dots}\ncline{dots}{Cm2}
  \ncangle[angleA=270,angleB=180]{Bm}{Cm1}\ncline{Cm3}{out}
\end{pspicture}
\end{center}\end{figure*}
In each round the result from the previous round is used along
with the next block of the message as input to a \defn{compression function}
which is designed to produce as few collisions as possible, with a fixed
initialization vector used for the first round.
So long as the compression function is collision resistent it can be shown that
the resulting hash function will also be collision resistent.
Thus attention can be focused on designing a good compression function for which it
would be very difficult for an adversary to create a block that collides with another.

However, weak collision resistance might not be sufficient to ensure the
authenticity of a message if an adversary is permitted to submit its
messages to a process responsible for calculating and encrypting hash codes before they
are sent.
In such a situation the hash function must have \defn{strong collision resistance},
meaning that it is computationally infeasible to create two messages $M$ and $M^\prime$
that have the same hash code ($h(M^\prime)=h(M)$).
Without strong collision resistance, an adversary could submit an innocuous
message $M$ to have its hash code calculated and encrypted by a process which is
trusted in the system. Then the adversary could swap the message $M$ with
a non-approved message $M^\prime$ and append the correct encrypted hash code.
When a recipient decrypted the hash code appended to $M^\prime$ it would presume
that the trusted process did approve it, since that process did appear to have
encrypted a valid hash code for it.

\newcommand{\wordvariants}[1]%
  {\hspace{-2mm}\begin{scriptsize}\begin{tabular}{c}#1\end{tabular}\end{scriptsize}\hspace{-2mm}}

This form of attack is known as a \defn{birthday attack}, due to a result
from probability which states that the probability that in a room of $n$ people
there are at least two that have the same birthday is
$1-\frac{365!}{(365-n)!365^n}$.
Surprisingly, with only $n=23$ people the probability is over $0.5$.
The attack proceeds as follows. First two messages are prepared by the adversary,
one innocuous message $M$ such as:
\begin{quote}\begin{code}
Hi Jack, the paper Highly Secure Systems is great
\end{code}\end{quote}
and another subversive message $M^\prime$:
\begin{quote}\begin{code}
Please transfer all the funds from the account of Jack
\end{code}\end{quote}
Then many slight variants of each is produced, such as the following $2^5=32$ variants of $M$:
\begin{quote}\begin{code}
\wordvariants{Hi\\Greetings} Jack, \wordvariants{the\\my} \wordvariants{paper\\course}
Highly Secure \wordvariants{Systems\\Software} is \wordvariants{great\\fun}
\end{code}\end{quote}
and the following $2^5=32$ variants of $M^\prime$:
\begin{quote}\begin{code}
\wordvariants{Please\\Immediately} \wordvariants{transfer\\remove} all
\wordvariants{the\\available} \wordvariants{funds\\money} from
\wordvariants{the\\\emph{}} account of Jack
\end{code}\end{quote}
These variants could be automated by the adversary using a thesaurus, or more easily
by alternating sequences of space-backspace-space with space-space-backspace.
For each variant of $M$ and of $M^\prime$ its hash code is calculated, and more variants
produced until a collision is found between a variant of $M$ and a variant of $M^\prime$.

If variants $M, M_1, M_2, \dots, M_{k-1}$ are produced with distinct hash codes,
and variants\\ $M^\prime, M_1^\prime, M_2^\prime, \dots, M_{k-1}^\prime$ are produced
with distinct hash codes,
then the probability that any selected $M_i$ has the same hash code as any
particular $M_j^\prime$ is $1/N$ (presuming all hash codes are equally likely),
where $N$ is the number of possible values of the hash code.
Thus the probability that $M_i$ does not have the same hash code as any of the
$k$ variants of $M^\prime$ is $\left(1-\frac{1}{N}\right)^k$.
If $N$ is large this probability is very close to $1$, making a brute-force attack infeasible.
However, there are also $k$ variants of $M$, so the probability that none of its variants
have the same hash code as any of the variants of $M^\prime$ is
$\left(\left(1-\frac{1}{N}\right)^k\right)^k=\left(1-\frac{1}{N}\right)^{k^2}$.
This probability drops below $0.5$ when:
\begin{displaymath}
  k > \sqrt{\frac{\ln 0.5}{\ln(1-1/N)}}\approx0.83\sqrt{N}.
\end{displaymath}
If the hash function produces hash codes with $m$ bits, so $N=2^m$, then
producing $k=2^{m/2}$ variants of $M$ and $M^\prime$ will probably be sufficient
for a successful birthday attack. This makes any hash function which produces
hash codes with less than about $128$ bits vulnerable.

A \defn{message authentication code} (MAC) is a type of hash code, but where the
hash function also takes a secret key as input as well as a message of arbitrary length.
Hence only the holders of the secret key are able to recalculate a MAC
and verify the authenticity of a message.
Essentially, a MAC acts like hash code encrypted using a secret key rather than
a private key. Unlike a symmetric cipher, a MAC hash function need not be one-to-one
which can make it harder for a cryptanalysis attack on the key used by a MAC. The inclusion
of the secret key in the hash function also increases the difficulty of attacks.
So long as only the receiver and the sender of a message share the secret key used
in calculating the MAC, the receiver can be assured the message was not altered and
did originate from the sender
(however, it can not prove to others that the sender and not itself was the creator of the MAC).

The \defn{Data Authentication Code} (DAC) was a popular MAC based on the DES cipher.
Essentially, it processes a message in $64$-bit blocks by encrypting each block
using CBC mode with a zero initialization vector, using padding with zeros if
required in the final block.
Either the entire final encrypted block is used as the DAC, or else just some of
the left-most bits of that block.
Despite the popularity of DAC, due to security weaknesses in DES the DAC is
being replaced by newer algorithms that are more resistant to attack.


\begin{exercise}[Implementation of DAC]
Prepare an implementation of the Data Authentication Code using a DES cipher.
\end{exercise}


%%%%%%%%%%%%%%%%%%%%%%%%%%%%%%%%%%%%%%%%%%%%%%%%%%%%%%%%%%%%%%%%%%%%%%%%%%%%%%%%%%%%%%%%%
%%%%% Hash and MAC Algorithms                                                       %%%%%
%%%%%%%%%%%%%%%%%%%%%%%%%%%%%%%%%%%%%%%%%%%%%%%%%%%%%%%%%%%%%%%%%%%%%%%%%%%%%%%%%%%%%%%%%

\section{Hash and MAC Algorithms}

\reading{pp352-374}

The \defn{Message Digest 5} (MD5) algorithm has been a popular cryptographic hash function
that produces a $128$-bit hash code for a message.
It processes a variable length message by breaking it into $512$-bit blocks and using the
hash code produced from one round as the initialization vector for the next round,
where the first round uses the fixed IV
\begin{code}67452301EFCDAB8998BADCFE10325476\end{code}.
The message is padded so that its length is divisible by $512$ by first appending a
single $1$ bit, followed by as many $0$ bits as required to leave exactly $64$ bits
available in the final block which are used to store the number of bits in the message
(hence MD5 is only designed for a message up to $2^{64}$ bits in length).
\begin{codebox}
\Procname{$\proc{MD5-Padding}(M)$}
\li \Comment pad the message $M$ including appending the $64$-bit integer length of $M$
\li $l \gets \left(\id{length}(M)+65\right)\mbox{ mod }512$
   \`\Comment number of occupied bits in last block
\li append bit $1$ to $M$
\li \For $i\gets l+1$ \To $512$ \Do
\li    append bit $0$
    \End
\li append $\id{length}(M)$ as a $64$-bit integer value
\end{codebox}
Each $512$-bit block is split into $16$ integer $32$-bit values which are used to modify
four $32$-bit buffer values $a$, $b$, $c$, $d$ created from the initialization vector.
These buffer values pass through $64$ iterations of a loop resulting in a hash value for
the block (used as the initialization vector for the next round) as can be seen by the
$\proc{MD5-Round}$ algorithm.
\begin{figure*}
\begin{codebox}
\Procname{$\proc{MD5-Round}(B, \id{IV})$}
\li \Comment perform message digest MD5 on 512-bit block $B$ using init vector $\id{IV}$
\li \Comment create 2D array $R$ of left rotate amounts for each iteration of loop
\li $R \gets \{\{7, 12, 17, 22\}, \{5, 9, 14, 20\}, \{4, 11, 16, 23\}, \{6, 10, 15, 21\}\}$
\li \Comment create integer array $T$ of constants using $\sin$ function in radians
\li \For $i \gets 0$ \To $63$ \Do
\li    $T[i] \gets \left\lfloor \id{abs}(\id{sin}(i+1))\cdot2^{32}\right\rfloor$
    \End
\li \Comment split block $B$ into $16$ words $W[0], \dots, W[15]$ each $32$-bit integer
\li \For $k \gets 0$ \To $15$ \Do
\li    $W[k] \gets B[32k\, \dots\, 32k+31]$
    \End
\li \Comment create $32$-bit int buffer values $a$, $b$, $c$, $d$ from initialization vector
\li $a \gets \id{IV}[0\dots31]$
\li $b \gets \id{IV}[32\dots63]$
\li $c \gets \id{IV}[64\dots95]$
\li $d \gets \id{IV}[96\dots127]$
\li \For $i \gets 0$ \To $63$ \Do
\li    \If $0\leq i\leq 15$ \Then
\li       $j \gets (b\wedge c)\vee(\overline{b}\wedge d)$
\li       $k \gets i$
\li    \ElseIf $16\leq i\leq 31$ \Then
\li       $j \gets (d\wedge b)\vee(\overline{d}\wedge c)$
\li       $k \gets (5i+1)\mbox{ mod }16$
\li    \ElseIf $32\leq i\leq 47$ \Then
\li       $j \gets b \oplus c \oplus d$
\li       $k \gets (3i+5)\mbox{ mod }16$
\li    \Else \Comment $48\leq i\leq 63$
\li       $j \gets c \oplus(b\vee\overline{d})$
\li       $k \gets 7i\mbox{ mod }16$
       \End
\li    $r \gets R[i/16][i\mbox{ mod }4]$ \`\Comment number of bits by which to rotate left
\li    $\id{temp} \gets d$
\li    $d \gets c$
\li    $c \gets b$
\li    $b \gets \left((\id{leftRotate}(a+j+T[i]+W[k], r)+b\right)\mbox{ mod }2^{32}$
\li    $a \gets \id{temp}$
    \End
\li \Comment calculate the message digest from $a$, $b$, $c$, $d$
\li $a \gets a + \id{IV}[0\dots31]$
\li $b \gets b + \id{IV}[32\dots63]$
\li $c \gets c + \id{IV}[64\dots95]$
\li $d \gets d + \id{IV}[96\dots127]$
\li \Return $128$-bit value $abcd$ as the message digest
\end{codebox}
\end{figure*}

Since 1996 various security vulnerabilities have been found in the MD5 algorithm,
including an attack that can find a collision in less than a minute.
Hence, MD5 is very vulnerable to a birthday attack, and no longer considered sufficiently
secure to provide authentication.
For example, the $128$-byte message:

\begin{scriptsize}\begin{code}\begin{quote}
\begin{tabular}{cccccccccccccccc}
  D1 & 31 & DD & 02 & C5 & E6 & EE & C4 & 69 & 3D & 9A & 06 & 98 & AF & F9 & 5C \\
  2F & CA & B5 & \underline{87} & 12 & 46 & 7E & AB & 40 & 04 & 58 & 3E & B8 & FB & 7F & 89 \\
  55 & AD & 34 & 06 & 09 & F4 & B3 & 02 & 83 & E4 & 88 & 83 & 25 & \underline{71} & 41 & 5A \\
  08 & 51 & 25 & E8 & F7 & CD & C9 & 9F & D9 & 1D & BD & \underline{F2} & 80 & 37 & 3C & 5B \\
  D8 & 82 & 3E & 31 & 56 & 34 & 8F & 5B & AE & 6D & AC & D4 & 36 & C9 & 19 & C6 \\
  DD & 53 & E2 & \underline{B4} & 87 & DA & 03 & FD & 02 & 39 & 63 & 06 & D2 & 48 & CD & A0 \\
  E9 & 9F & 33 & 42 & 0F & 57 & 7E & E8 & CE & 54 & B6 & 70 & 80 & \underline{A8} & 0D & 1E \\
  C6 & 98 & 21 & BC & B6 & A8 & 83 & 93 & 96 & F9 & 65 & \underline{2B} & 6F & F7 & 2A & 70
\end{tabular}
\end{quote}\end{code}\end{scriptsize}

\noindent
has the MD5 hash code \begin{code}79054025255FB1A26E4BC422AEF54EB4\end{code}.
But changing just the most-significant bit of only six bytes in this message results in
a different message with the identical hash code.
Despite its deficiencies, MD5 is still used to perform (non-secure) error checking on messages and
as a component in more secure algorithms.

The \defn{Secure Hash Algorithm} (SHA-1) has been a popular
successor to MD5. It was developed by the US National Security
Agency based on a predecessor (MD4) of the MD5 algorithm, but
produces a (more secure) $160$-bit hash code using five $32$-bit
buffer values $a$, $b$, $c$, $d$, $e$ and $80$ iterations of a
loop. It is employed in many popular security protocols such as
SSL/TLS and used for the secure e-mail protocols
Secure/Multipurpose Internet Mail Extensions (S/MIME) and Pretty
Good Privacy (PGP). The \begin{code}MessageDigest\end{code} class
from the \begin{code}java.security\end{code} package can be used
in Java to calculate hash codes for hash functions such as
\begin{code}MD5\end{code} and \begin{code}SHA-1\end{code}:
\begin{quote}\begin{code}\begin{verbatim}
MessageDigest digest = MessageDigest.getInstance("SHA-1");
byte[] message = ...;
digest.update(message); // can update several times
byte[] hashCode = digest.digest(); // digest gets reset
\end{verbatim}\end{code}\end{quote}
In C\# the class \begin{code}SHA1Managed\end{code} from the
\begin{code}System.Security.Cryptography\end{code} namespace is instead used:
\begin{quote}\begin{code}\begin{verbatim}
SHA1 digest = new SHA1Managed();
byte[] message = ...;
byte[] hashCode = digest.ComputeHash(message);
\end{verbatim}\end{code}\end{quote}

Although still widely used, the security of SHA-1 was considered suspect even before 2005
when an attack was found that produced collisions in SHA-1.
The attack took advantage of the structure and some security weaknesses in SHA-1,
requiring only $2^{69}$ operations to find a collision,
$2048$ times fewer than expected for a birthday attack.
Hence, both MD5 and SHA-1 are being phased out in favour of more secure hash functions.

In response to the potential vulnerabilities of SHA-1 it has been modified to give
three new algorithms, SHA-256, SHA-384, SHA-512,
which give hash codes of length $256$, $384$, $512$ bits respectively.
SHA-512 processes a message using blocks of $1024$ bits at a time,
and first pads the message to $1024$-bit blocks by appending
a $1$ bit followed by $0$ bits and then a $128$-bit integer giving the
length of the message in bits.
It uses eight $64$-bit buffer values $a$, $b$, $c$, $d$, $e$, $f$, $g$, $h$
and uses $80$ iterations of a loop.
In each iteration $64$ bits are extracted from the $1024$-bit block and passed
as an \defn{input sequence} into an operation along with the $512$ bits of
buffer values. After $80$ iterations each of the eight updated buffer values
is added to its original value (modulo $2^{64}$) to produce a $512$-bit hash code.
\begin{figure*}[htb]\begin{center}
\begin{pspicture}(-3,0)(8.5,7.5)
  \psline(0,0)(0,7.5)(5.8,5.75)(5.8,1.75)(0,0)
  \rput(3.5,0){SHA-512 compression function}

  \rput(-1.2,7.5){\rnode{B}{Block $B_i$}}

  \rput(1,5){\rnode{IS0}{\rule[-1.5mm]{0mm}{5mm}$I\!S_0$}} % note: rule is strut
  \psline(0.8,4.75)(0.6,5.25)(1.4,5.25)(1.2,4.75)(0.8,4.75)
  \rput(2,5){\rnode{IS1}{\rule[-1.5mm]{0mm}{5mm}$I\!S_1$}} % note: rule is strut
  \psline(1.8,4.75)(1.6,5.25)(2.4,5.25)(2.2,4.75)(1.8,4.75)
  \rput(2.75,5){\rnode{ISdots}{\scriptsize$\dots$}}
  \rput(3.5,5){\rnode{IS79}{\rule[-1.5mm]{0mm}{5mm}$I\!S_{\!7\!9}$}} % note: rule is strut
  \psline(3.3,4.75)(3.1,5.25)(3.9,5.25)(3.7,4.75)(3.3,4.75)

  \rput(1,3){\rnode{R0}{\psframebox{\rule[-8mm]{0mm}{16mm}Op}}}
  \rput(2,3){\rnode{R1}{\psframebox{\rule[-8mm]{0mm}{16mm}Op}}}
  \rput(2.75,3){\rnode{Rdots}{\scriptsize$\dots$}}
  \rput(3.5,3){\rnode{R79}{\psframebox{\rule[-8mm]{0mm}{16mm}Op}}}

  \rput(-1.8,1.75){\rnode{In}{\shortstack{Round\\Input}}}

  \pnode(1,5.75){B0}
  \pnode(2,5.75){B1}
  \pnode(3.5,5.75){B79}

  \rput(5,3.75){\rnode{Add}{\psframebox{\shortstack{8\\word\\pair\\adds}}}}
  \rput(7.5,3.75){\rnode{Out}{\shortstack{Round\\Output}}}

  \ncangle[angleA=270, angleB=180]{B}{B79}\naput[npos=0.8]{\hspace{-1mm}\scriptsize{1024 bit}}
  \ncline[arrows=->]{B0}{IS0}\ncline[arrows=->]{IS0}{R0}\nbput{\scriptsize{64 bit}\hspace{-1mm}}
  \ncline[arrows=->]{B1}{IS1}\ncline[arrows=->]{IS1}{R1}
  \ncline[arrows=->]{B79}{IS79}\ncline[arrows=->]{IS79}{R79}

  \ncangle[arrows=->,angleA=0,angleB=180]{In}{R0}\naput[npos=0.4]{\scriptsize{512 bit}}
  \ncline[arrows=->]{R0}{R1}
  \ncline[arrows=->]{R1}{Rdots}
  \ncline[arrows=->]{Rdots}{R79}
  \ncangle[arrows=->,angleA=0,angleB=180]{R79}{Add}
  \ncangle[arrows=->,angleA=0,angleB=270]{In}{Add}
  \ncline[arrows=->]{Add}{Out}\naput[npos=0.6]{\scriptsize{512 bit}}
\end{pspicture}
\end{center}\end{figure*}

The \defn{Whirlpool} algorithm uses a different approach to generating a hash function.
It is based on the AES block cipher, but has been adapted to provide performance
comparable to SHA-512, and processes a variable length message by breaking it into
$512$-bit blocks to produce a $512$-bit hash code ($128$ bits was considered
too susceptible to a birthday attack to be used by a hash function).
It too starts by padding the message with a $1$ bit followed by some $0$ bits
and a $256$-bit integer length of the message, but uses 10 iterations of an S-box substitution,
a rotation of the columns, a mix of rows, and an XOR with a round key which is obtained
by performing the same iterations with a fixed initialization vector.
Although it is now an ISO standard, and along with SHA-256, SHA-384, SHA-512 is one of
the only hash functions endorsed by the New European Schemes for Signatures, Integrity, and
Encryption (NESSIE), the default \begin{code}SunJCE\end{code} security provider does not (yet) provide
support for Whirlpool. However, support is available through other Java providers
such as the Bouncy Castle provider using the algorithm name \begin{code}WhirlpoolDigest\end{code},
and in the Lightweight API with the following code fragment:
\begin{quote}\begin{code}\begin{verbatim}
Digest digest = new WhirlpoolDigest();
byte[] message = ...;
digest.update(message, 0, message.length);
byte[] hashCode = new byte[digest.getDigestSize()];
digest.doFinal(hashCode, 0);
\end{verbatim}\end{code}\end{quote}

Usually passwords are not stored as plaintext in a system (even on
a process that might be considered relatively secure), but rather
only ever stored in some encrypted form. This way, if a password
file is obtained by an adversary only the encrypted keys are
obtained. When a password needs to be checked by the system for
validity, its plaintext form is encrypted (and so can be safely
communicated) and compared with the stored encrypted key. Since
the original password never need be obtained from the encrypted
key, hash functions are usually chosen for the encryption as they
are designed to be one-way (whereas ciphers are designed to be
reversible). Although MD5 does not have strong collision
resistance, it still provides some security against a
\defn{preimage attack}, where a cryptanalyst attempts to find a
message that has a specific hash code, and so is still widely used
for encrypting passwords.

\defn{Password based encryption} (PBE) derives an secret key from a password.
In order to make attempts to deduce the password more time
consuming for an attacker, most PBE implementations mix a (fixed)
random number called a \defn{salt} with the key. Even if an
attacker did manage to find the secret key corresponding to a
password, the salt could be changed, thus totally changing the key
(and if different salts were used around a system, each would have
to be attacked separately). The following Java code fragment uses
MD5 and DES to generate a PBE encrypted key from a password
(specified as a \begin{code}char[]\end{code} array since
\begin{code}String\end{code} objects are immutable), a salt, and
an \begin{code}int\end{code} that specifies the number of
iterations to use in the PBE calculation:
\begin{quote}\begin{code}\begin{verbatim}
char[] password = ...; // never stored as a String
byte[] salt = ...; // fixed random salt bytes
PBEKeySpec pbeSpec = new PBEKeySpec(password, salt, 20);
SecretKeyFactory keyFactory
   = SecretKeyFactory.getInstance("PBEWithMD5AndDES");
SecretKey key = keyFactory.generateSecret(pbeSpec);
\end{verbatim}\end{code}\end{quote}
The resulting secret key could then be used as an encrypted
version of the password, or instead as the secret key for DES
encryption:
\begin{quote}\begin{code}\begin{verbatim}
Cipher cipher = Cipher.getInstance("PBEWithMD5AndDES");
\end{verbatim}\end{code}\end{quote}
PBE works by first appending the salt to the password and then
applying the hash function repeatedly to the result the specified
number of iterations. In the case of an MD5 hash function, a 128
bit hash code is produced, which for a DES cipher the first 64
bits get used as the secret key and the remaining 64 bits as the
initialization vector for CBC mode.

The Data Authentication Code described in Section \ref{Se:MessageAuthentication}
uses the DES algorithm with CBC mode to produce a MAC for a message.
However, cryptographic hash functions such as MD5, SHA-1 and Whirlpool execute faster than
a cipher such as DES, so it is generally preferable to create a MAC from a hash function
than from a symmetric cipher.
The \defn{Key-Hash Message Authentication Code} (HMAC) algorithm provides a way to
generate a MAC by using a cryptographic hash function $h$ combined with a secret key $K$.
Any hash function which operates block by block on a message can be used, and HMAC
is designed so that alternative hash functions can be easily plugged in,
enabling hash functions to be replaced when more secure ones are required.
HMAC produces a MAC with the same length as the hash function that it uses,
so \begin{code}HmacMD5\end{code} produces
a $128$-bit MAC and \begin{code}HmacSHA512\end{code} produces a $512$-bit MAC.
It starts by checking whether the length of the key is exactly one block.
If the key is less than one block it is padded with $0$ bits on the left,
whereas if it is more than one block $h(K)$ is calculated.
This results in a key $K^+$ with length one block.
Next, $K^+$ is passed through a bitwise XOR with the block
\id{ipad}=\begin{code}3636\dots36\end{code} (in hexadecimal).
Then the message $M$ is appended to this block
and the hash code calculated, giving:
\begin{displaymath}
  h\left(K^+\oplus\id{ipad} \,||\, M\right).
\end{displaymath}
For efficiency with multiple messages, the hash code $h\left(K^+\oplus\id{ipad}\right)$
can be precalculated and used as the initialization vector for the calculation of each $h(M)$.
Separately, $K^+$ is passed through an bitwise XOR with the block
\id{opad}=\begin{code}5C5C\dots5C\end{code}.
The result has $h\left(K^+\oplus\id{ipad} \,||\, M\right)$ appended
to it and the hash code calculated again to give the MAC:
\begin{displaymath}
  h\left(K^+\oplus\id{opad}||h\left(K^+\oplus\id{ipad} \,||\, M\right)\right).
\end{displaymath}
Again, for efficiency when dealing with multiple messages,
$h\left(K^+\oplus\id{opad}\right)$ can be precalculated and used as the initialization
vector for the calculation of each $h\left(h\left(K^+\oplus\id{ipad} \,||\, M\right)\right)$.
The JCA supports MAC algorithms such as HMAC with the \begin{code}Mac\end{code} class:
\begin{quote}\begin{code}\begin{verbatim}
KeyGenerator kg = KeyGenerator.getInstance("HmacSHA512");
SecretKey key = kg.generateKey();
Mac mac = Mac.getInstance("HmacSHA512");
mac.init(key);
byte[] message = ...;
byte[] hashCode = mac.doFinal(message);
\end{verbatim}\end{code}\end{quote}


\begin{exercise}[SMS Message Authentication]
  Enhance Exercise \ref{Ex:AESEncryptionMobile} so a Whirlpool hash code is appended
  to the SMS message which the receiver can use to authenticate the message.
\end{exercise}


%%%%%%%%%%%%%%%%%%%%%%%%%%%%%%%%%%%%%%%%%%%%%%%%%%%%%%%%%%%%%%%%%%%%%%%%%%%%%%%%%%%%%%%%%
%%%%% Digital Signatures                                                            %%%%%
%%%%%%%%%%%%%%%%%%%%%%%%%%%%%%%%%%%%%%%%%%%%%%%%%%%%%%%%%%%%%%%%%%%%%%%%%%%%%%%%%%%%%%%%%

\section{Digital Signatures}\label{Se:DigitalSignatures}

\reading{pp378-382,390-393,478-479}

A \defn{digital signature} is an encrypted sequence of bytes appended to a message
that can be used to verify the sender of the message, when the signature was created,
and authenticate the contents of the message.
The advent of public-key encryption made digital signatures feasible, enabling
an identifier of the sender, a timestamp, and a hash code of the message to be together
encrypted using the private key of the sender.
Any process with access to the corresponding public key could then decrypt the digital
signature, recalculate the hash code for the message and verify the authenticity
of the message (both the authenticity of the sender and the integrity of the message).
If confidentiality of the message were important, the message together with its
digital signature could be encrypted before it were sent.

Digital signatures also provide \defn{non-repudiation}, unless another message
with the same hash code has been created by the receiver or an adversary,
or else the sender has released its private key,
it cannot deny having placed its signature on the message.
Note that instead using symmetric encryption of a message appended with a hash code does
allow the receiver to authenticate the message, but does not itself
provide non-repudiation as the sender could claim that the receiver (who also
holds the secret key) might have produced the message itself, or that it is a
replay attack by an adversary.

To avoid disputes between the sender and receiver over the validity of the timestamp
provided in a digital signature any time-critical messages (such as on-line instructions
for buying or selling stocks) might be sent via a mutually-trusted \defn{arbitrator},
which itself checks the signature of the sender and puts its own timestamp on the
message before encrypting it itself for the receiver.

Any standard public-key encryption algorithm such as RSA or ECC combined with
a cryptographic hash function such as SHA-1 can be used to create a digital signature,
so long as recipients are aware of which encryption algorithm and hash function
have been chosen and the correct public key is available to the recipients.
Instead, the \defn{Digital Signature Algorithm} (DSA) is a standardized algorithm
designed specifically for signing and verifying digital signatures which makes
use of a hash function such as SHA-1.
First, global public key components $p$, $q$, $g$ are created and made available
in the system (which can be shared by all DSA users in the system) using the following steps:
\begin{enumerate}
  \item a $160$-bit prime number $q$ is selected (called the \defn{sub-prime}),
  \item an $L$-bit prime number of the form $p=mq+1$ for some integer multiple $m$ is found,
  where $L$ is either $1024$ (default for SHA-1) or else $3072$ (for SHA-256),
  $7680$ (for SHA-384), or $15360$ (for SHA-512),
  \item a value $n<p-1$ for which $n^m\mbox{ mod }p > 1$ is selected,
  and the \defn{base} $g$ is taken to be $g=n^m\mbox{ mod }p$.
\end{enumerate}
Then the signer creates a private key $x$ by selecting a secure random number $x<q$,
and calculates a public key $y=g^x\mbox{ mod }p$.
In order to sign a message $M$ using a cryptographic hash function $h$ the signer
uses the following steps:
\begin{enumerate}
  \item a nonce value $k<q$ is selected for the message and its multiplicative inverse
  $k^{-1}$ in $\mathbb{Z}_q$ is calculated using the Euclidean algorithm,
  \item the signature is the pair $(r,s)$ where $r=\left(g^k\mbox{ mod }p\right)\textrm{ mod }q$\\ and $s=\left(k^{-1}\left(h(M)+xr\right)\right)\mbox{ mod }q$.
\end{enumerate}
The recipient of a message $M$ can verify a signature $(r,s)$ for a received message $M$ by
obtaining the global public elements $p$, $q$, $g$, the public key $y$, and then use
the following steps:
\begin{enumerate}
  \item if $r\geq q$ or $s\geq q$ the signature is rejected,
  \item the multiplicative inverse $s^{-1}$ of $s$ in $\mathbb{Z}_q$ is calculated,
  \item the values $u_1=h(M)s^{-1}\mbox{ mod }q$ and $u_2=rs^{-1}\mbox{ mod }q$ are both calculated,
  \item the value $v=\left(g^{u_1}y^{u_2}\mbox{ mod }p\right)\mbox{ mod }q$ is calculated,
  \item the signature is accepted as valid if and only if $v=r$.
\end{enumerate}
The fact that correct DSA signatures do get accepted as valid can be verified once Fermat's Theorem
is applied to show that $g^q\mbox{ mod }p=1$. The security of DSA is based on
the difficulty of calculating the discrete logarithm $x$ given the values $g$, $p$,
and $y=g^x\mbox{ mod }p$.

An advantage to using DSA instead of a standard public-key encryption algorithm such as RSA
is that the only computationally intensive calculations for signing a message are the calculations
of $r$ and $k^{-1}$. Both these calculations can be performed beforehand for several
nonce values $k$ so that messages can be quickly signed.

In Java the keys required for DSA can be generated by first obtaining a
\begin{code}KeyPairGenerator\end{code}:
\begin{quote}\begin{code}\begin{verbatim}
KeyPairGenerator kpg = KeyPairGenerator.getInstance("DSA");
\end{verbatim}\end{code}\end{quote}
and initializing it to either generate new key components:
\begin{quote}\begin{code}\begin{verbatim}
kpg.initialize(1024);
\end{verbatim}\end{code}\end{quote}
or else use existing values of $p$, $q$, $g$:
\begin{quote}\begin{code}\begin{verbatim}
DSAParameterSpec dsaSpec = new DSAParameterSpec(p, q, g);
kpg.initialize(dsaSpec);
\end{verbatim}\end{code}\end{quote}
Then a key pair is generated:
\begin{quote}\begin{code}\begin{verbatim}
KeyPair keyPair = kpg.generateKeyPair();
PrivateKey privateKey = keyPair.getPrivate();
PublicKey publicKey = keyPair.getPublic();
\end{verbatim}\end{code}\end{quote}
From the two keys the values of $p$, $q$, $g$, $x$, $y$ can be
obtained by typecasting them to \begin{code}DSAPrivateKey\end{code}
and \begin{code}DSAPublicKey\end{code} respectively.
Typically the receiver is already provided with the public key $y$
and so a \begin{code}KeyFactory\end{code} would be used instead of a
\begin{code}KeyPairGenerator\end{code} to create the \begin{code}PublicKey\end{code}:
\begin{quote}\begin{code}\begin{verbatim}
DSAPublicKeySpec dsaPublicSpec=new DSAPublicKeySpec(y,p,q,g);
KeyFactory kf = KeyFactory.getInstance("DSA");
PublicKey publicKey = kf.generatePublic(dsaPublicSpec);
\end{verbatim}\end{code}\end{quote}

To sign a message a \begin{code}Signature\end{code} object is created
specifying the hash function (such as SHA-1) to use and is
initialized with the private key:
\begin{quote}\begin{code}\begin{verbatim}
Signature signer = Signature.getInstance("SHA1withDSA");
signer.initSign(privateKey);
byte[] message = ...;
signer.update(message);
byte[] signature = signer.sign();
\end{verbatim}\end{code}\end{quote}
To verify a signature the \begin{code}Signature\end{code} object is
instead initialized with the public key:
\begin{quote}\begin{code}\begin{verbatim}
Signature signer = Signature.getInstance("SHA1withDSA");
signer.initVerify(publicKey);
byte[] message = ...;
byte[] signature = ...;
signer.update(message);
boolean verifies = signer.verify(signature);
\end{verbatim}\end{code}\end{quote}

The \defn{Elliptic Curve Digital Signature Algorithm} (ECDSA) is a variant of DSA
which utilizes elliptic curves instead of using exponentiation, enabling smaller
keys to be used for the production of signatures.
First the chosen field (prime $p$), elliptic curve ($a$ and $b$ values),
point $G$ with order $n$ are made available in the system.
The signer creates a private key $x$ by selecting a random number $x<n$,
calculates a public key $Q=xG$, and then uses the following steps:
\begin{enumerate}
  \item a nonce value $k<n$ is selected (relatively prime to $n$)
  for which the point $kG=(x,y)$ has $x$-coordinate with $x\mbox{ mod }n\neq0$ and
  for which $k^{-1}\left(h(M)+x^2\right)\mbox{ mod }n\neq0$,
  \item the signature is $(r,s)$ where $r=x\mbox{ mod }n$ and
  $s=k^{-1}\left(h(M)+xr\right)\mbox{ mod }n$.
\end{enumerate}
The signature can then be validated by calculating
$u_1=h(M)s^{-1}\mbox{ mod }n$ and $u_2=r s^{-1}\mbox{ mod }n$,
and checking whether the $x$-coordinate of $u_1G+u_2Q$ is $r$.
ECDSA is supported by the Bouncy Castle Lightweight API.

The class \begin{code}DSAFileSigner\end{code} demonstrates how DSA can be used to add
a signature to a text file, and later verify that the signature is still valid.
If between the signing and the verification the file is modified in some way
the signature will no longer be correct (unless a collision is found).
In this example the signature has been stored using a Base 64 encoding,
which stores the signature as printable text characters rather than a
\begin{code}byte[]\end{code} array.
For simplicity it has used the non-standard \begin{code}sun.misc\end{code} classes
\begin{code}BASE64Encoder\end{code} and \begin{code}BASE64Decoder\end{code}
to perform the encoding and decoding between a \begin{code}byte[]\end{code}
array and a \begin{code}String\end{code} of printable characters.
These classes might not be supported in some Java virtual machines
and so should not be used in a production environment.
\begin{figure*}\begin{program}\begin{verbatim}
/**
   A class that demonstrates how DSA can be used to add a digital
   signature to a text file, using line "SIGNATURE:" followed by base
   64 encoding of signature, and verify if signature is still valid.
   @author Andrew Ensor
*/
...
public class DSAFileSigner
{
   public static String SIG_HEADER = "SIGNATURE:";
   private PrivateKey privateKey;
   private PublicKey publicKey;

   public DSAFileSigner()
   {  try
      {  // generate the public and private keys for DSA
         KeyPairGenerator kpg = KeyPairGenerator.getInstance("DSA");
         kpg.initialize(1024);
         KeyPair keyPair = kpg.generateKeyPair();
         privateKey = keyPair.getPrivate();
         publicKey = keyPair.getPublic();
      }
      catch (NoSuchAlgorithmException e)
      {  System.err.println("Encryption algorithm not available: "+e);
      }
   }

   // returns a Base64 encoded DSA signature for chars obtained from
   // reader and then closes the reader stream
   public String sign(Reader reader) throws IOException
   {  byte[] signature = null;
      BufferedReader br = new BufferedReader(reader);
      try
      {  Signature signer = Signature.getInstance("SHA1withDSA");
         signer.initSign(privateKey);
         String line = br.readLine();
         while (line != null)
         {  signer.update(line.getBytes());
            line = br.readLine();
         }
         signature = signer.sign();
      }
      ...
      finally
      {  try
         {  br.close();
         }
         catch (IOException e)
         {} // ignore
      }
      return (new BASE64Encoder()).encode(signature);
   }
\end{verbatim}\hfill \emph{cont-}\end{program}\end{figure*}%
\begin{figure*}\begin{program}\emph{-cont}\begin{verbatim}
   // obtains the information from reader until the signature header
   // is found, and then tries to verify the signature
   public boolean verify(Reader reader) throws IOException
   {
      BufferedReader br = new BufferedReader(reader);
      try
      {  Signature signer = Signature.getInstance("SHA1withDSA");
         signer.initVerify(publicKey);
         String line = br.readLine();
         while (line!=null && !line.equals(SIG_HEADER))
         {  signer.update(line.getBytes());
            line = br.readLine();
         }
         if (line == null)
            System.out.println("No signature found");
         else
         {  // obtain the signature from the remaining information
            StringBuffer sigString = new StringBuffer();
            line = br.readLine();
            while (line != null)
            {  sigString.append(line);
               line = br.readLine();
            }
            byte[] signature = (new BASE64Decoder()).decodeBuffer
               (sigString.toString());
            return signer.verify(signature);
         }
      }
      catch (NoSuchAlgorithmException e)
      {  System.err.println("Encryption algorithm not available: "+e);
      }
      catch (InvalidKeyException e)
      {  System.err.println("Invalid key: "+e);
      }
      catch (SignatureException e)
      {  System.err.println("Signature algorithm exception: "+e);
      }
      finally
      {  try
         {  br.close();
         }
         catch (IOException e)
         {} // ignore
      }
      return false;
   }
   ...
}
\end{verbatim}\end{program}\end{figure*}

\noindent
\begin{minipage}{95mm}
A \defn{Base 64} encoding is a binary to text encoding scheme where
an arbitrary number of bytes is converted into a sequence of printable
characters, using the 64 printable characters
\begin{code}A,B,\dots,Z,a,b,\dots,z,0,1,\dots,9,+,/\end{code}
(formats such as MIME and PGP use these characters but other encoding schemes might use
different characters instead of \begin{code}+\end{code} and \begin{code}/\end{code}).
Given a \begin{code}byte[]\end{code} array to be encoded (such as a digital signature
or other \begin{code}byte\end{code} data), a $24$-bit buffer is used to process the bits
in the array.
The bytes are processed in blocks of three bytes at a time and converted into four printable
characters by splitting the buffer into four $6$-bit parts and using a table of the
corresponding character encodings for each $6$-bit value.
If the length of the array is not a multiple of three then either one or two
\begin{code}0\end{code} bytes is appended to make a full block and the encoded
output ends with that number of \begin{code}=\end{code} characters
to indicate the number of padding bytes.\nudgedown{1}
\end{minipage}
\hfill
\begin{tabular}{|c|c|} \hline
  \multicolumn{2}{|l|}{\textit{Base 64 Encoding}} \\ \hline\hline
  \textit{6-bit Value} & \textit{Character} \\ \hline
  \begin{code}000000\end{code} & \begin{code}A\end{code} \\
  \begin{code}000001\end{code} & \begin{code}B\end{code} \\
  \vdots & \vdots \\
  \begin{code}011001\end{code} & \begin{code}Z\end{code} \\
  \begin{code}011010\end{code} & \begin{code}a\end{code} \\
  \begin{code}011011\end{code} & \begin{code}b\end{code} \\
  \vdots & \vdots \\
  \begin{code}110011\end{code} & \begin{code}z\end{code} \\
  \begin{code}110100\end{code} & \begin{code}0\end{code} \\
  \begin{code}110101\end{code} & \begin{code}1\end{code} \\
  \vdots & \vdots \\
  \begin{code}111101\end{code} & \begin{code}9\end{code} \\
  \begin{code}111110\end{code} & \begin{code}+\end{code} \\
  \begin{code}111111\end{code} & \begin{code}/\end{code} \\ \hline
\end{tabular}

For example, the bytes \begin{code}73,65,63,75,72,69,74,79\end{code}
(the hexadecimal representation of the string \begin{code}security\end{code})
would be encoded as follows:
\begin{center}
\newcommand{\sz}{\makebox[-2.6mm]{\scriptsize0}}
\newcommand{\so}{\makebox[-2.6mm]{\begin{scriptsize}1\end{scriptsize}}}
\noindent
\begin{tabular}{cccccccc|cccccccc|cccccccc||cccccccc|cccccccc|cccccccc||cccccccc|cccccccc|cccccccc}
  \multicolumn{8}{c|}{\begin{code}73\end{code}} &
  \multicolumn{8}{c|}{\begin{code}65\end{code}} &
  \multicolumn{8}{c||}{\begin{code}63\end{code}} &
  \multicolumn{8}{c|}{\begin{code}75\end{code}} &
  \multicolumn{8}{c|}{\begin{code}72\end{code}} &
  \multicolumn{8}{c||}{\begin{code}69\end{code}} &
  \multicolumn{8}{c|}{\begin{code}74\end{code}} &
  \multicolumn{8}{c|}{\begin{code}79\end{code}} &
  \multicolumn{8}{c}{\begin{code}00\end{code}} \\
  \sz & \so & \so & \so & \sz & \sz & \so & \so &
  \sz & \so & \so & \sz & \sz & \so & \sz & \so &
  \sz & \so & \so & \sz & \sz & \sz & \so & \so &
  \sz & \so & \so & \so & \sz & \so & \sz & \so &
  \sz & \so & \so & \so & \sz & \sz & \so & \sz &
  \sz & \so & \so & \sz & \so & \sz & \sz & \so &
  \sz & \so & \so & \so & \sz & \so & \sz & \sz &
  \sz & \so & \so & \so & \so & \sz & \sz & \so &
  \sz & \sz & \sz & \sz & \sz & \sz & \sz & \sz \\
  \multicolumn{6}{c}{\begin{code}c\end{code}} &
  \multicolumn{6}{c}{\begin{code}2\end{code}} &
  \multicolumn{6}{c}{\begin{code}V\end{code}} &
  \multicolumn{6}{c||}{\begin{code}j\end{code}} &
  \multicolumn{6}{c}{\begin{code}d\end{code}} &
  \multicolumn{6}{c}{\begin{code}X\end{code}} &
  \multicolumn{6}{c}{\begin{code}J\end{code}} &
  \multicolumn{6}{c||}{\begin{code}p\end{code}} &
  \multicolumn{6}{c}{\begin{code}d\end{code}} &
  \multicolumn{6}{c}{\begin{code}H\end{code}} &
  \multicolumn{6}{c}{\begin{code}k\end{code}} &
  \multicolumn{6}{c}{\begin{code}=\end{code}}
\end{tabular}\end{center}

\noindent
resulting in the printable string \begin{code}c2VjdXJpdHk=\end{code}.
Formats such as MIME also insert a newline (a combination of carriage return and newline characters)
every 76 encoded characters so that the resulting strings are not too long for
e-mail systems.


\begin{exercise}[Base 64 Encoding]
Prepare a program that performs Base 64 encoding and decoding.
\end{exercise}


%%%%%%%%%%%%%%%%%%%%%%%%%%%%%%%%%%%%%%%%%%%%%%%%%%%%%%%%%%%%%%%%%%%%%%%%%%%%%%%%%%%%%%%%%
%%%%% Certificates and Key Storage                                                   %%%%%
%%%%%%%%%%%%%%%%%%%%%%%%%%%%%%%%%%%%%%%%%%%%%%%%%%%%%%%%%%%%%%%%%%%%%%%%%%%%%%%%%%%%%%%%%

\section{Certificates and Key Storage}\label{Se:CertificatesKeyStorage}

\reading{pp419-430}

A \defn{(public-key) certificate} is a document that contains an identifier for a process
and its public key, together with a digital signature of the information
prepared by a signer that is trusted to ensure the authenticity of
the identifier and the public key.
The signer is often a \defn{certificate authority} (CA), a well-known
service that is entrusted to securely check information before signing and issuing
a certificate. The CA prepares a digital signature using a public key that has been securely
distributed around the system as a \defn{self-signed} or \defn{root} certificate,
a certificate containing the public key and signed using the corresponding private key.
Certificates make public-key encryption practical in large distributed systems,
where it is usually infeasible to securely distribute secret keys.
A process can pass its public key to another process simply by sending a
certificate that has been signed by a CA trusted by the other process.
When a certificate is received the digital signature can be verified using the
public key of the CA to check the authenticity of the identifier and public key.

The most commonly used standard format for public-key certificates is \defn{X.509}
\begin{figure*}[htb]\begin{center}
\begin{tabular}{|l|} \hline
  Version: \emph{1 (default), 2, or 3} \\
  Serial Number: \emph{integer ID unique amongst certificates issued by that CA} \\
  Signature Algorithm: \emph{identifier for algorithm used to sign certificate} \\
  Issuer: \emph{X.500 name of certificate issuer that created and signed certificate} \\ \hline
  Validity \\
  \qquad Not Before: \emph{first date from which certificate is valid} \\
  \qquad Not After: \emph{last date on which certificate is valid} \\ \hline
  Subject: \emph{name of subject for whom the certificate is issued} \\ \hline
  Subject Public Key Info \\
  \qquad Public Key Algorithm: \hspace{-1mm}\emph{identifier for algorithm used to create public key}\\
  \qquad Key: \emph{public key for the subject} \\ \hline
  Issuer Unique Identifier: \emph{optional unique name for certificate issuer} \\
  Subject Unique Identifier: \emph{optional unique name for certificate subject} \\ \hline
  \emph{Extensions to X.509 certificate} \\ \hline
  Signature Algorithm: \emph{identifier for algorithm used to sign certificate} \\
  Signature: \emph{digital signature of other fields of certificate} \\ \hline
\end{tabular}\end{center}
\end{figure*}
which holds details about the issuer of the certificate, the period of validity of
the certificate, details about the subject for whom the certificate is issued
including its public key, followed by a digital signature of the information
signed using the issuer's private key.
X.509 is particularly convenient for algorithms such as ECC and DSA whose
keys have multiple components as all the components get included in a single
certificate.

The \begin{code}java.security.cert\end{code} package contains the class
\begin{code}X509Certificate\end{code} that represents an X.509 certificate.
The \begin{code}getEncoded\end{code} method of \begin{code}PublicKey\end{code}
can be used to obtain a single \begin{code}byte[]\end{code} array for an X.509 public key
specification holding all the components of the public key (but the specification is not
itself an X.509 certificate as it only holds the public key information and is not signed):
\begin{quote}\begin{code}\begin{verbatim}
byte[] publicKeyBytes = publicKey.getEncoded();//X.509 format
String publicKeyAlg = publicKey.getAlgorithm();
\end{verbatim}\end{code}\end{quote}
This \begin{code}byte[]\end{code} array can later be used to reconstruct the public key
if the algorithm that was used to create the key is known:
\begin{quote}\begin{code}\begin{verbatim}
X509EncodedKeySpec publicKeySpec
   = new X509EncodedKeySpec(publicKeyBytes);
KeyFactory keyFactory = KeyFactory.getInstance(publicKeyAlg);
PublicKey publicKey=keyFactory.generatePublic(publicKeySpec);
\end{verbatim}\end{code}\end{quote}
If instead an X.509 certificate has been received via an \begin{code}InputStream\end{code}
then the \begin{code}X509Certificate\end{code} can be reconstructed
using a \begin{code}CertificateFactory\end{code}, in which
case the validity of the certificate can be verified if the public key of the signer is known:
\begin{quote}\begin{code}\begin{verbatim}
InputStream is = ...; // byte stream of X.509 certificate
CertificateFactory certFactory
   = CertificateFactory.getInstance("X.509");
X509Certificate certificate
   = (X509Certificate)certFactory.generateCertificate(is);
PublicKey signerPublicKey=...;//securely obtained from signer
PublicKey certPublicKey = certificate.getPublicKey();
boolean notExpired = certificate.checkValidity();
boolean validSignature=certificate.verify(signerPublicKey);
\end{verbatim}\end{code}\end{quote}

It is usually not practical to insist that all certificates in a large system
be signed and issued by a single CA. Hence it may occur that a process needs to verify
a certificate signed by a CA, but the process only has the public key for another
signer. In this case a \defn{certificate chain} can be used, which is a sequence
of certificates where each certificate in the chain can be verified using the public key
that is the subject in the following certificate,
where the last certificate in the chain is already trusted.
For instance, if a process $A$ has a trusted certificate for process $B$
then it can use the public key of $B$ in that certificate to verify any certificate
that has been signed by $B$ (using the corresponding private key of $B$).
If process $B$ signed a certificate for process $C$ then when $A$ received this certificate
it could verify the signature and thus obtain the public key of $C$ from the certificate.
Thus $A$ could then verify any certificates it received that were signed by $C$.
In this way when a certificate is received by a process it can check whether there
is a certificate chain that leads to a certificate that it trusts.

As the validity of each certificate in a certificate chain is essential
a CA typically has a \defn{certificate revocation list} (CRL),
which it maintains with an up-to-date list of certificates that it no longer considers
valid (such as if the private key were found to have been compromised,
or the certificate details were found to have been falsified).
To verify a certificate it is therefore important to check that it is not
in the CRL of the signer.
The procedures for creating, managing, storing, distributing, verifying,
and revoking certificates
using standardized certificate formats, certificate authorities, and
certificate revocation lists are collectively known as
\defn{public-key infrastructure} (PKI).

A \defn{keystore} is a password-protected file or database that stores keys
and trusted certificates.
A Java keystore can be conveniently managed using the command line \begin{code}keytool\end{code}
utility (found in the \begin{code}bin\end{code} folder of the Java SDK).
Every entry in a keystore is assigned a unique string \defn{alias}
to identify it, which is specified with the \begin{code}-alias\end{code} option.
A new keystore is automatically created if one with the specified name does not
already exist when a new key pair is generated or a certificate is imported.
The password for a keystore can be changed using the \begin{code}-storepasswd\end{code}
option
(note in practice \begin{code}keytool\end{code} should be left to
prompt for the passwords rather than passwords provided directly via the
option \begin{code}-storepass\end{code} in a command-line shell,
as the operating system might cache previous commands so leave the password
exposed):
\begin{quote}\begin{code}
keytool -storepasswd -keystore \emph{mykeystore.jks}
\end{code}\end{quote}
The entries in a keystore can be listed using the \begin{code}-list\end{code} option:
\begin{quote}\begin{code}
keytool -list -keystore \emph{mykeystore.jks}
\end{code}\end{quote}
which displays the number of entries currently in the key store and an MD5 signature
for each.
Also including the \begin{code}-v\end{code} option displays the entries in
a human-readable format, whereas including the \begin{code}-rfc\end{code} option
displays the entries using Base 64 encoding.

A new public/private key pair is created using the
option \begin{code}-genkeypair\end{code}, where the key pair
generation algorithm to use is specified with the \begin{code}-keyalg\end{code} option.
Besides generating a public/private key pair, the private key is stored securely using
a password and the public key is placed in a self-signed certificate,
signed using the private key.
For example, a command line statement such as:
\begin{quote}\begin{code}
keytool -genkeypair -alias \emph{mykeyalias} -keyalg RSA \\
\trind  -keystore \emph{mykeystore.jks}
\end{code}\end{quote}
generates a new key pair for RSA encryption together with a self-signed certificate,
where both the private key and the self-signed certificate are assigned the specified alias
of the entry.
If the certificate is to be deployed to a server such as \begin{code}http://localhost\end{code}
then the first and last name of the subject must exactly match \begin{code}localhost\end{code}
(or the option \begin{code}-dname "cn=localhost"\end{code} be included),
otherwise the certificate would be rejected during verification.
Any key pair or signature algorithm that is supported by a registered
Java cryptographic security provider can be used.

A certificate can be exported from a keystore to a binary certificate file
using the \begin{code}-export\end{code} option, specifying the alias of the
entry and the name of the new certificate file:
\begin{quote}\begin{code}
keytool -export -alias \emph{mykeyalias} -file \emph{mycertfile.cer} \\
\trind  -keystore \emph{mykeystore.jks}
\end{code}\end{quote}
The contents of the certificate file can be displayed using the
\begin{code}-printcert\end{code} option:
\begin{quote}\begin{code}
keytool -printcert -file \emph{mycertfile.cer}
\end{code}\end{quote}
Certificate files exported from another keystore that are trusted can also be imported
into the keystore with the \begin{code}-import\end{code} option:
\begin{quote}\begin{code}
keytool -import -alias \emph{mykeyalias} -file \emph{mycertfile.cer} \\
\trind  -keystore \emph{mykeystore.jks}
\end{code}\end{quote}
The \begin{code}keytool\end{code} utility attempts to build a certificate chain for
the new certificate from a self-signed certificate and already trusted certificates
in the keystore.
If it fails to establish a certificate chain, then the certificate information
is displayed and the user prompted to verify it
(by independently checking the signature with the alleged issuer of the certificate).
Some self-signed certificates for several popular CA are available in the
\begin{code}cacerts\end{code} keystore included with the Java Runtime
(found in the \begin{code}jre\bsl lib\bsl security\end{code} folder
and with the initial keystore password \begin{code}"changeit"\end{code}).

A standardized \defn{certificate authentication request} (CAR) for requesting
a certificate to be signed and issued by a certificate authority can be prepared
using the \begin{code}-certReq\end{code} option:
\begin{quote}\begin{code}
keytool -certReq -alias \emph{mykeyalias} -file \emph{mycertreq.csr} \\
\trind  -keystore \emph{mykeystore.jks}
\end{code}\end{quote}
When the reply has been received from the CA the self-signed certificate
in the keystore can be replaced by the certificate chain signed by the CA.

The \begin{code}KeyStore\end{code} class can be used by a program to access
a keystore independently of how it stores entries.
Either the default type of keystore can be used or else a particular type can be specified
(such as \begin{code}"JKS"\end{code} for a Java Key Standard keystore as produced by
\begin{code}keytool\end{code}, or \begin{code}"PKCS12"\end{code} for a keystore
in the PKCS\#12 format).
A default type of keystore can be loaded with the following statements:
\begin{quote}\begin{code}\begin{verbatim}
String keyStoreName = ...;
char[] storepw = ...; // don't store password in a String
FileInputStream fis = new FileInputStream(keyStoreName);
KeyStore keystore
   = KeyStore.getInstance(KeyStore.getDefaultType());
keystore.load(fis, storepw);
fis.close();
\end{verbatim}\end{code}\end{quote}
(passing a \begin{code}null\end{code} \begin{code}InputStream\end{code}
creates an empty keystore).
Once a keystore has been loaded an \begin{code}Enumeration<String>\end{code} of its aliases
can be obtained using the \begin{code}aliases\end{code} method.
Existing keys and certificates can be read using the alias for each entry:
\begin{quote}\begin{code}\begin{verbatim}
String keyAlias = ...;
char[] keypw = ...; // don't store password in a String
Key key = keystore.getKey(keyAlias, keypw);
String certAlias = ...;
Certificate certificate = keystore.getCertificate(certAlias);
\end{verbatim}\end{code}\end{quote}
New entries can also be added to the keystore or existing entries modified:
\begin{quote}\begin{code}\begin{verbatim}
Key key = ...;
Certificate[] chain = ...; // certificate chain if PrivateKey
keystore.setKeyEntry(keyAlias, key, keypw, chain);
Certificate certificate = ...;
keystore.setCertificateEntry(certificate, certAlias);
\end{verbatim}\end{code}\end{quote}
which is convenient for providing secure persistent storage of keys.
If changes have been made they can be saved to the keystore:
\begin{quote}\begin{code}\begin{verbatim}
FileOutputStream fos = new FileOutputStream(keyStoreName);
keystore.store(fos, storepw);
fos.close();
\end{verbatim}\end{code}\end{quote}

The \defn{Java Certification Path API} consists of classes and interfaces
in the package \begin{code}java.security.cert\end{code} for verifying certificate chains.
The \begin{code}CertPath\end{code} class represents a certificate chain, ordered
starting with the target certificate and ending with a certificate signed by
a trusted CA, so that the signer of one certificate has its public key as the subject
in the next certificate in the chain.
For example, the following code fragment obtains a certificate chain from a keystore
and uses it to create a \begin{code}CertPath\end{code} object:
\begin{quote}\begin{code}\begin{verbatim}
String certAlias = ...;
Certificate[] chain=keystore.getCertificateChain(certAlias);
CertificateFactory cf
   = CertificateFactory.getInstance("X.509");
CertPath path = cf.generateCertPath(Arrays.asList(chain));
\end{verbatim}\end{code}\end{quote}
Once a \begin{code}CertPath\end{code} has been created it can be
validated by the PKI X.509 certification path validation algorithm, which throws
a \begin{code}CertPathValidatorException\end{code} if the validation fails:
\begin{quote}\begin{code}\begin{verbatim}
PKIXParameters params = new PKIXParameters(keystore);
params.setRevocationEnabled(false); // no CRL
CertPathValidator validator = CertPathValidator.getInstance
   (CertPathValidator.getDefaultType());
PKIXCertPathValidatorResult vr=(PKIXCertPathValidatorResult)
   validator.validate(path, params); // exception if fails
X509Certificate trustedCert
   = vr.getTrustAnchor().getTrustedCert();
\end{verbatim}\end{code}\end{quote}
The Certification Path API also provides support for certificate
revocation lists, including the
\defn{Online Certificate Status Protocol} (OCSP) for determining the
current status of a certificate.


\begin{exercise}[Distributing Certificates]
Modify \begin{code}DSAFileSigner\end{code} from Section \ref{Se:DigitalSignatures}
so that it uses a DSA key pair that is generated by \begin{code}keytool\end{code}
and stored in a keystore.
Then export the DSA public key certificate to another keystore which is
used by another program to verify the signature of the file.
\end{exercise}
