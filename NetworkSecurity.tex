% This is the LaTeX file for the chapter:
%   Network Security
% for the manual:
%   Highly Secure Systems
% Written by Andrew Ensor
% Last updated December 12, 2018.

\chapter{Network Security}

%%%%%%%%%%%%%%%%%%%%%%%%%%%%%%%%%%%%%%%%%%%%%%%%%%%%%%%%%%%%%%%%%%%%%%%%%%%%%%%%%%%%%%%%%
%%%%% Authentication Services                                                       %%%%%
%%%%%%%%%%%%%%%%%%%%%%%%%%%%%%%%%%%%%%%%%%%%%%%%%%%%%%%%%%%%%%%%%%%%%%%%%%%%%%%%%%%%%%%%%

\section{Authentication Services}\label{Se:AuthenticationServices}

\reading{pp401-419}

User authentication is vital to the security of a distributed system,
access to some servers in the system might need to be controlled,
operations such as printing might need to be billed to the correct user,
and execution of some types of operations might need to be restricted to authorized users.

One of the earliest and most-widely used authentication services is called
\defn{Kerberos}. Kerberos allows users in an insecure network to
authenticate their identity using a standardized protocol and a trusted
key distribution centre, taking responsibility for the authentication of
users and controlling their access to services within its \defn{realm}
(part of the system for which it has authentication responsibility).
A Kerberos key distribution centre is logically divided into two parts,
although in a small system both might reside on the same server machine:
\begin{description}
  \item[authentication server] which has a secret (master) key for secure communication
  with each user in its realm and which is responsible for checking user
  credentials to provide authentication, typically listening on port 88,
  \item[ticket granting server] which the user consults once its identity
  has been authenticated to obtain authorization to use a service
  provided by a server in the realm, typically listening on port 749.
\end{description}
The protocol used by Kerberos follows an enhanced version of the steps
from Section \ref{Se:SecretKeyDistribution}.
The user obtains a \defn{ticket granting ticket} (TGT) from the authentication
server once its identity has been authenticated, which it can present
to the ticket granting server to receive a \defn{service granting ticket} (SGT)
for a particular service in the realm, authorizing the user to access that
service during a specified time period.

When a user logs on to the system via a client machine the client obtains
the secret master key from the user for decrypting responses from the
authentication server
(such as by using password based encryption with a user-supplied password).
The client sends a plaintext message to the authentication server
requesting a ticket granting ticket to be produced for the user which vouches
for the user's identity, including:
\begin{itemize}
  \item options for requesting the TGT to have particular properties,
  such as being renewable or be postdated,
  \item an identifier for the user,
  \item the name of the realm,
  \item an identifier for the ticket granting server that will be used,
  \item a requested validity time period for the TGT,
  \item a nonce value.
\end{itemize}
The authentication server checks whether the user is known to it and
if so responds to the client with a TGT suitably encrypted for the ticket granting server,
and with a session key for the client communication with the ticket generating
server during the lifetime of the TGT, including:
\begin{itemize}
  \item the name of the realm,
  \item an identifier for the user,
  \item the TGT which holds the session key, the name of the realm,
  an identifier for the user, the network address of the client,
  and validity period of the TGT, all encrypted by the authentication server
  for the ticket granting server using a secret key,
  \item the session key, the validity period of the TGT, the same nonce value,
  the name of the realm, and an identifier for the ticket granting server,
  all encrypted by the authentication server for the user using the master key.
\end{itemize}

\noindent
\begin{center}\begin{pspicture}(0,0)(12,3.8)
  \rput(6,3.8){\emph{Start of user session}}
  \psline[fillcolor=yellow,fillstyle=solid](3,1.75)(3,3.25)(9,3.25)(9,1.75)(3,1.75)
  \rput(8.45,3.45){\scriptsize Kerberos}
  \rput(2,0.75){\ovalnode{C}{\makebox[15mm]%
    {\scriptsize\begin{tabular}{c}Client machine\\with user\end{tabular}}}}
  \rput(4.5,2.5){\ovalnode[fillcolor=white,fillstyle=solid]{A}{\makebox[15mm]%
    {\scriptsize\begin{tabular}{c}Authentication\\Server\end{tabular}}}}
  \rput(7.5,2.5){\ovalnode[fillcolor=white,fillstyle=solid]{T}{\makebox[15mm]%
    {\scriptsize\begin{tabular}{c}Ticket Granting\\Server\end{tabular}}}}
  \rput(10,0.75){\ovalnode{S}{\makebox[15mm]%
    {\scriptsize\begin{tabular}{c}Server with\\available service\end{tabular}}}}
  \nccurve[arrows=->,linecolor=red,linewidth=1mm,angleA=75,angleB=195]{C}{A}
  \nccurve[arrows=->,linecolor=red,linewidth=1mm,angleA=255,angleB=15]{A}{C}
  \rput(1.5,1.75){\tiny\begin{tabular}{c}authentication\\request\end{tabular}}
  \rput(4.45,1.05){\tiny\begin{tabular}{c}TGT reply\\and session key\end{tabular}}
\end{pspicture}\end{center}

The ticket granting ticket can then be used by the client as proof of the user's
authenticity during the remainder of the session or until the TGT expires.
This avoids the need for the user to repeatedly enter its log on password during
a single session, providing a feature known as \defn{single sign-on}.
Whenever the user tries to use a new service during that session the client
presents the TGT to the ticket granting server, requesting a SGT to authorize
its access to that service, including with the request:
\begin{itemize}
  \item options for requesting the SGT to have particular properties,
  \item an identifier for the service,
  \item requested validity time period for the SGT,
  \item a nonce value,
  \item the TGT from the authentication server,
  \item an \defn{authenticator} for the request, consisting of an identifier
  for the user, the name of the realm, and a timestamp, all encrypted by the
  client using the session key.
\end{itemize}
The ticket granting server decrypts the TGT to check its validity,
determines the session key, and so can decrypt the authenticator to verify the authenticity
of the sender. If the ticket granting server determines that the user should
be authorized to use the service, then it responds to the client, including:
\begin{itemize}
  \item the name of the realm,
  \item an identifier for the user,
  \item the SGT which holds a new session key for the client communication with the
  service, the name of the realm, an identifier for the user, the network
  address of the client, and validity period of the SGT, all encrypted
  by the ticket granting server for the service using a secret key,
  \item the session key for the client communication with the service,
  the validity period of the SGT, the same nonce value, the name of the realm,
  and an identifier for the service, all encrypted by the ticket granting server
  using its session key with the client.
\end{itemize}

\noindent
\begin{center}\begin{pspicture}(0,0)(12,3.8)
  \rput(6,3.8){\emph{First usage of a particular service during session}}
  \psline[fillcolor=yellow,fillstyle=solid](3,1.75)(3,3.25)(9,3.25)(9,1.75)(3,1.75)
  \rput(8.45,3.45){\scriptsize Kerberos}
  \rput(2,0.75){\ovalnode{C}{\makebox[15mm]%
    {\scriptsize\begin{tabular}{c}Client machine\\with user\end{tabular}}}}
  \rput(4.5,2.5){\ovalnode[fillcolor=white,fillstyle=solid]{A}{\makebox[15mm]%
    {\scriptsize\begin{tabular}{c}Authentication\\Server\end{tabular}}}}
  \rput(7.5,2.5){\ovalnode[fillcolor=white,fillstyle=solid]{T}{\makebox[15mm]%
    {\scriptsize\begin{tabular}{c}Ticket Granting\\Server\end{tabular}}}}
  \rput(10,0.75){\ovalnode{S}{\makebox[15mm]%
    {\scriptsize\begin{tabular}{c}Server with\\available service\end{tabular}}}}
  \nccurve[arrows=->,linecolor=red,linewidth=1mm,angleA=30,angleB=210]{C}{T}
  \nccurve[arrows=->,linecolor=red,linewidth=1mm,angleA=255,angleB=0]{T}{C}
  \rput(2.5,1.5){\tiny\begin{tabular}{c}authorization\\request\end{tabular}}
  \rput(6.75,0.6){\tiny\begin{tabular}{c}SGT reply and\\service session key\end{tabular}}
\end{pspicture}\end{center}

Once the client has received a service granting ticket for a particular service
it can present it to the service with each request along with an authenticator
for the request (which can optionally include an alternative session key chosen by the
client, and a starting sequence number to detect replay attacks during the exchange).
When the SGT expires, the client can request another SGT from the
ticket granting server so long as it still has a valid TGT.
This process is repeated for each new service the user requests.

The \defn{Java Authentication and Authorization Service} (JAAS) is an API
included with Java Standard Edition to support the authentication of users and their
authorization to execute security-sensitive operations.
It is a flexible framework found in the package \begin{code}javax.security.auth\end{code}
and subpackages of it, allowing a client to authenticate a user independently of
the type of JAAS-compatible authentication service actually used in the system
(such as Kerberos, a smart card system, or a biometric authentication system).
A user or process that requires authentication is represented by the
\begin{code}Subject\end{code} class, and each identity that a
\begin{code}Subject\end{code} has,
such as a user name within a realm or a unique identity number,
is represented by the \begin{code}Principal\end{code} interface
(which is implemented by classes such as \begin{code}KerberosPrincipal\end{code}).
The class \begin{code}SimplePrincipal\end{code} demonstrates a simple
\begin{code}Principal\end{code} that represents a user name given by a string.
\begin{figure*}\begin{program}\begin{verbatim}
/**
   A class that represents a simple JAAS Principal which provides a
   String name as an identity for a Subject
   @see SimpleLoginModule.java
*/
package simplejaasmodule;

import java.security.Principal;

public class SimplePrincipal implements Principal
{
   private String name;

   public SimplePrincipal(String name)
   {  if (name == null)
         throw new IllegalArgumentException("No name for subject");
      this.name = name;
   }

   public boolean equals(Object another)
   {  if (another==null || !(another instanceof SimplePrincipal))
         return false;
      else
      {  SimplePrincipal principal = (SimplePrincipal)another;
         return name.equals(principal.getName());
      }
   }

   public String getName()
   {  return name;
   }

   public int hashCode()
   {  return name.hashCode();
   }

   public String toString()
   {  return "SimplePrincipal with name " + name;
   }
}
\end{verbatim}\end{program}\end{figure*}

A JAAS-compatible authentication service provides an implementation of the
\begin{code}LoginModule\end{code} interface, authenticating in a way
suitable for that type of service.
Authentication with a \begin{code}LoginModule\end{code} is performed
in two steps. Firstly, the \begin{code}login\end{code} method is used to
attempt authentication of a \begin{code}Subject\end{code},
using a provided \begin{code}CallbackHandler\end{code} to communicate
with the user (such as to prompt for a user name and password and obtain
the response).
As the authentication process could require authentication by several
successive \begin{code}LoginModule\end{code}, any of which might fail to authenticate
the user,
a \begin{code}LoginModule\end{code} has an \begin{code}abort\end{code} method
to abort the authentication (which gets called if authentication fails in another
\begin{code}LoginModule\end{code}), and a \begin{code}commit\end{code} method
to add a suitable \begin{code}Principal\end{code} to the \begin{code}Subject\end{code}.
As an example, the class \begin{code}SimpleLoginModule\end{code}
performs authentication of a user via a \begin{code}NameCallback\end{code}
and a \begin{code}PasswordCallback\end{code}.
\begin{figure*}\begin{program}\begin{verbatim}
/**
   A class that represents a simple JAAS LoginModule which performs
   the authentication of a Subject via a specified CallbackHandler
   and assigns a SimplePrincipal to the Subject if the authentication
   is committed
   @author Andrew Ensor
*/
package simplejaasmodule;

import java.io.IOException;
import java.util.Arrays;
import java.util.Map;
import javax.security.auth.Subject;
import javax.security.auth.callback.Callback;
import javax.security.auth.callback.CallbackHandler;
import javax.security.auth.callback.NameCallback;
import javax.security.auth.callback.PasswordCallback;
import javax.security.auth.callback.UnsupportedCallbackException;
import javax.security.auth.login.LoginException;
import javax.security.auth.spi.LoginModule;

public class SimpleLoginModule implements LoginModule
{
   // subject being authenticated
   private Subject subject;
   // handles communication with user in application-specific way
   private CallbackHandler callbackHandler;
   //credential sharing across multiple LoginModule for single sign on
   private Map<String,?> sharedState;
   // configuration options for this LoginModule (allows debug=true)
   private Map<String,?> options;
   private boolean debugEnabled;
   private SimplePrincipal principal;
   private boolean committed = false;
   ...

   public void initialize(Subject subject,
      CallbackHandler callbackHandler, Map<String,?> sharedState,
      Map<String,?> options)
   {  this.subject = subject;
      this.callbackHandler = callbackHandler;
      this.sharedState = sharedState;
      this.options = options;
      if (options.get("debug").toString().equalsIgnoreCase("true"))
         debugEnabled = true;
   }
\end{verbatim}\hfill \emph{cont-}\end{program}\end{figure*}%
\begin{figure*}\begin{program}\emph{-cont}\begin{verbatim}
   public boolean login() throws LoginException
   {  principal = null;
      committed = false;
      if (callbackHandler == null)
         throw new LoginException("No callback handler provided");
      // prepare callbacks to pass to callbackHandler
      NameCallback nameCallback = new NameCallback("user name:");
      PasswordCallback passwordCallback
         = new PasswordCallback("password:", false);
      Callback[] callbacks = {nameCallback, passwordCallback};
      try
      {  callbackHandler.handle(callbacks);
      }
      catch (IOException e)
      {  throw new LoginException(e.toString());
      }
      catch (UnsupportedCallbackException e)
      {  throw new LoginException(e.toString());
      }
      // extract the credentials obtained by callbackHandler
      String userName = nameCallback.getName();
      char[] tempPassword = passwordCallback.getPassword();
      if (tempPassword == null)
         tempPassword = new char[0];
      // copy the password to another array before it gets cleared
      char[] password = new char[tempPassword.length];
      System.arraycopy(tempPassword, 0, password, 0,
         tempPassword.length);
      passwordCallback.clearPassword();
      // output debugging information
      if (debugEnabled)
      {  System.out.print("User entered:" + userName +
            " with password:");
         for (int i=0; i<password.length; i++)
            System.out.print(password[i]);
         System.out.println();
      }
      if (validate(userName, password))
      {  principal = new SimplePrincipal(userName);
         return true;
      }
      else
      {  // don't wait for garbage collection to clear the password
         Arrays.fill(password, ' ');
         return false;
      }
   }
\end{verbatim}\hfill \emph{cont-}\end{program}\end{figure*}%
\begin{figure*}\begin{program}\emph{-cont}\begin{verbatim}
   // helper method that tries to validate the user name and password
   // note in practice this would be checked in a secure database
   private boolean validate(String name, char[] password)
   {  String validName = "Jack";
      char[] validPassword = {'c','h','a','n','g','e','i','t'};
      boolean validated = name.equals(validName) &&
         password.length==validPassword.length;
      if (validated)
      {  for (int i=0; i<password.length; i++)
            if (password[i] != validPassword[i])
               validated = false;
      }
      // output debugging information
      if (debugEnabled)
         System.out.println("Authenticated user:" + validated);
      return validated;
   }

   public boolean abort() throws LoginException
   {  if (principal == null)
         // own login failed so abort login is not performed
         return false;
      if (committed)
         // all LoginModule login succeeded but one's commit failed
         logout();
      else  // another LoginModule login failed
         principal = null;
      return true; // abort was successful
   }

   public boolean commit() throws LoginException
   {  if (principal == null)
         // own login failed so can not commit login
         return false;
      if (!subject.getPrincipals().contains(principal))
      {  subject.getPrincipals().add(principal);
         if (debugEnabled)
            System.out.println("SimplePrincipal added to subject");
      }
      principal = null;
      committed = true;
      return true;
   }

   public boolean logout() throws LoginException
   {  subject.getPrincipals().remove(principal);
      principal = null;
      committed = false;
      return true;
   }
}
\end{verbatim}\end{program}\end{figure*}
It should be remembered however that a \begin{code}LoginModule\end{code}
would usually be implemented by a JAAS provider (such as a Kerberos provider)
rather than by an application developer.
As there might be several alternative authentication services available in a system,
the \begin{code}LoginModule\end{code} can all be specified in a configuration file,
from which a client application can choose which configuration to use
for authentication.

\begin{program}\begin{verbatim}
/**
   JAASExample.config login configuration file with some
   configurations for a hypothetical distributed system
*/

SimpleLogin
{  simplejaasmodule.SimpleLoginModule required debug=true;
};

KerberosLogin
{  com.sun.security.auth.module.Krb5LoginModule required;
};

BiometricLogin
{  com.csp.jaasmodule.BioLoginModule sufficient;
   com.csp.jaasmodule.JavaCardLoginModule required matchOnCard="true";
};
\end{verbatim}\end{program}

\noindent
The name of this configuration file can be specified either when the
client application is executed:
\begin{quote}\begin{code}
java -Djava.security.auth.login.config==JAASExample.config\dots
\end{code}\end{quote}
or else in the Java security properties file \begin{code}java.security\end{code}
with the following lines:
\begin{quote}\begin{code}
login.configuration.provider=com.csp.JAASprovider \\
login.config.url.1=\emph{path}/JAASExample.config
\end{code}\end{quote}

A client application can use JAAS to authenticate a user regardless of which
authentication service is deployed in the system. First, a
\begin{code}CallbackHandler\end{code} is prepared to handle the
actual communication with the user in a way suitable for the application
(such as via \begin{code}System.in\end{code} and \begin{code}System.out\end{code}
or instead via a GUI). Then the client creates a \begin{code}LoginContext\end{code}
which is used to log in and log out,
specifying the name (given in the configuration file) of the login
configuration to use, and a \begin{code}CallbackHandler\end{code} to use during
the authentication:
\begin{quote}\begin{code}\begin{verbatim}
CallbackHandler callbackHandler = ...;
LoginContext lc
   = new LoginContext("SimpleLogin", callbackHandler);
lc.login();
Subject subject = lc.getSubject(); // authenticated subject
...
lc.logout();
\end{verbatim}\end{code}\end{quote}
The \begin{code}LoginContext\end{code} takes care of locating the appropriate
\begin{code}LoginModule\end{code} specified in the configuration file and
performs the necessary authentication(s).

Once a \begin{code}Subject\end{code} has been obtained it can optionally be used to
call a \begin{code}PrivilegedAction\end{code}, which has security-sensitive
operations (such as file write access) in a \begin{code}run\end{code} method.
In a security conscious system applications should not be allowed
to run unrestricted (the default for a local Java application), but instead
have a \defn{fine-grained security} as specified by a \defn{security policy}
(typically determined by a system administrator).
As described further in Section \ref{Se:SecuringCode}
a security policy file consists of a series of \begin{code}grant\end{code}
statements, each with optional signer field, a JAR file or URL codebase,
and a principal, that together specify the code and/or user to which the permissions apply:
\begin{quote}\begin{code}
grant signedBy \emph{alias} codeBase \emph{uri} Principal \emph{PrinClass} \emph{name} \\
\{\dbind permission \emph{SecurityPermissionClass} \emph{Action}; \\
\trind \vdots \\
\trind permission \emph{SecurityPermissionClass} \emph{Action}; \\
\};
\end{code}\end{quote}
When the static \begin{code}Subject\end{code} method
\begin{code}doAsPrivileged\end{code} is called, the \begin{code}run\end{code}
method of the \begin{code}Privileged\\Action\end{code} is called using
whatever security policies are in place for the specified subject.

As an example, the class \begin{code}JAASExampleClient\end{code}
demonstrates a simple client application that uses the
class \begin{code}JAASExampleCallbackHandler\end{code} to handle
communication with the user.
The client application uses the \begin{code}"SimpleLogin"\end{code}
configuration to attempt authentication of a user
(if a Kerberos authentication server were available then this
configuration could be swapped to \begin{code}"KerberosLogin"\end{code}).
It then tries to use that user's security permission as specified in
the security policy file \begin{code}JAASExample.policy\end{code}
to perform a security-sensitive operation, attempting to write a file
to the local drive.
Although \begin{code}JAASExampleCallbackHandler\end{code}
uses a \begin{code}Scanner\end{code} to handle the input of the
user name, note that it is more much careful when obtaining the user's
password, avoiding the creation of any string that might linger
on the client machine, and erasing all details of the password beside
that held in a single \begin{code}char[]\end{code} array.
\begin{figure*}\begin{program}\begin{verbatim}
/**
   A class that demonstrates how JAAS can be used to authenticate a
   user and authorize whether the user can execute a PrivilegedAction
   @author Andrew Ensor
*/
import java.io.FileWriter;
import java.io.IOException;
import java.security.PrivilegedAction;
import javax.security.auth.Subject;
import javax.security.auth.callback.CallbackHandler;
import javax.security.auth.login.LoginContext;
import javax.security.auth.login.LoginException;

public class JAASExampleClient implements PrivilegedAction<Boolean>
{
   // PrivilegedAction run method that contains security-sensitive
   // operation and has any specified generic return type
   public Boolean run()
   {  try
      {  FileWriter fw = new FileWriter("localfile.txt");
         fw.write("JAAS says hello on local drive");
         fw.close();
         return true; // worked
      }
      catch (IOException e)
      {  System.err.println("IO Exception while writing file: " + e);
         return false; // failed
      }
   }

   public static void main(String[] args)
   {  PrivilegedAction<Boolean> privilegedAction
         = new JAASExampleClient();
      CallbackHandler callbackHandler=new JAASExampleCallbackHandler();
      try
      {  System.out.println("Creating login context");
         LoginContext lc
            = new LoginContext("SimpleLogin", callbackHandler);
         System.out.println("Logging in");
         lc.login();
         Subject subject = lc.getSubject(); // authenticated subject
         System.out.println("Executing security-sensitive operation");
         // execute the privileged action as specified subject with
         // no protection domains taken from current thread's context
         boolean success = (Boolean)Subject.doAsPrivileged(subject,
            privilegedAction, null);
         System.out.println("Operation " +(success?"was":"not")
            + " successful");
         System.out.println("Logging out");
         lc.logout();
      }
\end{verbatim}\hfill \emph{cont-}\end{program}\end{figure*}%
\begin{figure*}\begin{program}\emph{-cont}\begin{verbatim}
      catch (LoginException e)
      {  System.err.println("Login Exception: " + e);
      }
      catch (SecurityException e)
      {  System.err.println("Security Exception: " + e);
      }
   }
}
\end{verbatim}\end{program}\end{figure*}
\begin{figure*}\begin{program}\begin{verbatim}
/**
   A class that represents a simple JAAS CallbackHandler that just
   uses System.in and System.out to communicate with the user
   Note a more elaborate CallbackHandler might eg use a GUI
   @see JAASExampleClient.java
*/
import java.io.InputStream;
import java.io.IOException;
import java.io.PushbackInputStream;
import java.util.Arrays;
import java.util.Scanner; // Java 1.5 equivalent of cs1.Keyboard
import javax.security.auth.callback.Callback;
import javax.security.auth.callback.CallbackHandler;
import javax.security.auth.callback.NameCallback;
import javax.security.auth.callback.PasswordCallback;
import javax.security.auth.callback.TextOutputCallback;
import javax.security.auth.callback.UnsupportedCallbackException;

public class JAASExampleCallbackHandler implements CallbackHandler
{
   private Scanner keyboardInput;

   public JAASExampleCallbackHandler()
   {  keyboardInput = new Scanner(System.in);
   }

   public void handle(Callback[] callbacks)
      throws IOException, UnsupportedCallbackException
   {  if (callbacks == null)
         return; // no callbacks to handle
      for (int i=0; i<callbacks.length; i++)
      {  if (callbacks[i] instanceof TextOutputCallback)
         {  TextOutputCallback callback
               = (TextOutputCallback)callbacks[i];
\end{verbatim}\hfill \emph{cont-}\end{program}\end{figure*}%
\begin{figure*}\begin{program}\emph{-cont}\begin{verbatim}
            switch (callback.getMessageType())
            {  case TextOutputCallback.INFORMATION:
                  System.out.println(callback.getMessage());
                  break;
               case TextOutputCallback.ERROR:
                  System.out.println("ERROR:"+callback.getMessage());
                  break;
               case TextOutputCallback.WARNING:
                  System.out.println("WARNING:"+callback.getMessage());
                  break;
               default:
                  throw new IOException("Unsupported message type: "
                     + callback.getMessageType());
            }
         }
         else if (callbacks[i] instanceof NameCallback)
         {  NameCallback callback = (NameCallback)callbacks[i];
            // prompt user for name and get name using Scanner
            System.out.print(callback.getPrompt());
            callback.setName(keyboardInput.nextLine());
         }
         else if (callbacks[i] instanceof PasswordCallback)
         {  PasswordCallback callback=(PasswordCallback)callbacks[i];
            // prompt user for password and get using util method
            System.out.print(callback.getPrompt());
            callback.setPassword(readPassword(System.in));
         }
         else
            throw new UnsupportedCallbackException(callbacks[i],
               "Unsupported callback");
      }
   }

   // utility method that obtains a password from an InputStream
   // without storing the password in any String
   private char[] readPassword(InputStream is) throws IOException
   {  final int INITIAL_CAPACITY = 128;
      char[] buffer = new char[INITIAL_CAPACITY];
      int numChars = 0; // number of characters currently in buffer
      boolean done = false;
      while (!done)
      {  int ch = is.read();
         switch (ch)
         {  case -1:
            case '\n':
               done = true;
               break;
\end{verbatim}\hfill \emph{cont-}\end{program}\end{figure*}%
\begin{figure*}\begin{program}\emph{-cont}\begin{verbatim}
            case '\r':
               int nextChar = is.read();
               if ((nextChar != '\n') && (nextChar != -1))
               {  if (!(is instanceof PushbackInputStream))
                     is = new PushbackInputStream(is);
                  ((PushbackInputStream)is).unread(nextChar);
               }
               else
                  done = true;
               break;
            default:
               if (numChars >= buffer.length)
               {  // buffer full so allocate a new larger buffer
                  char[] newBuffer = new char[buffer.length*2];
                  System.arraycopy(buffer, 0, newBuffer, 0, numChars);
                  Arrays.fill(buffer, ' '); // clear old buffer
                  buffer = newBuffer;
               }
               buffer[numChars++] = (char)ch;
               break;
         }
      }
      if (numChars == 0)
         return null;
      // copy used portion of buffer to new buffer that gets returned
      char[] returnBuffer = new char[numChars];
      System.arraycopy(buffer, 0, returnBuffer, 0, numChars);
      Arrays.fill(buffer, ' '); // clear buffer
      return returnBuffer;
   }
}
\end{verbatim}\end{program}\end{figure*}


\begin{exercise}[Fine-Grained Security Permissions]
  Test the class \begin{code}JAAS\-ExampleClient\end{code} with and
  without a restricted security policy specified at runtime,
  and arrange for several users to be granted different security permissions.
\end{exercise}


%%%%%%%%%%%%%%%%%%%%%%%%%%%%%%%%%%%%%%%%%%%%%%%%%%%%%%%%%%%%%%%%%%%%%%%%%%%%%%%%%%%%%%%%%
%%%%% Electronic Mail Security                                                      %%%%%
%%%%%%%%%%%%%%%%%%%%%%%%%%%%%%%%%%%%%%%%%%%%%%%%%%%%%%%%%%%%%%%%%%%%%%%%%%%%%%%%%%%%%%%%%

\section{Electronic Mail Security}

\reading{pp438-474}

When a user sends an email from a email client program
the message is first communicated via the \defn{Simple Mail Transfer Protocol} (SMTP)
to a mail server on the network
(such as \begin{code}mail.aut.ac.nz\end{code} on the AUT network
or \begin{code}smtp1.voda\-fone.net.nz\end{code} on the Vodafone network),
which runs an SMTP service on port 25 and has responsibility for processing all
incoming and outgoing email.
The client starts the communication by identifying itself (\begin{code}HELO\end{code})
and telling the server it has an email (\begin{code}MAIL FROM\end{code}) to deliver to
one or more recipients (\begin{code}RCPT TO\end{code}), followed by the contents
of the email (\begin{code}DATA\end{code}).
The contents of an email that adheres to the original \defn{RFC 822} standard
consists of a header (usually including fields such
as \begin{code}Date\end{code}, \begin{code}From\end{code}, \begin{code}To\end{code},
and \begin{code}Subject\end{code}), followed by a blank line and the text body
of the message using 7-bit ASCII character encoding.
It then indicates to the server that it has completed
(a \begin{code}.\end{code} followed by \begin{code}QUIT\end{code}).
\begin{quote}\begin{code}
HELO \emph{clientURL} \\
MAIL FROM: \emph{sender@aut.ac.nz} \\
RCPT TO: \emph{recipient@aut.ac.nz} \\
DATA \\
Date: \emph{1/1/2007 12:30pm} \\
From: \emph{sender@whitehouse.com} \\
To: \emph{recipient@aut.ac.nz} \\
Subject: \emph{subject} \\
\\
\emph{Text Message} \\
. \\
QUIT \\
\end{code}\end{quote}
The mail server responds to each directive with a numerical code and a message that
indicates the success or failure of the requested operation.
When a mail server receives an email message it checks whether it has an inbox for
the message recipient, and if not it relays (forwards) the email to another mail server
closer to the recipient's address.

When an email message arrives at the destination mail server it is stored in a file
until the email client program of the recipient accesses it.
The \defn{Post Office Protocol} (POP) can be used to retrieve the mail for
the recipient and remove it from the server.
Instead, the \defn{Internet Message Access Protocol} (IMAP) is a more advanced protocol
that can be used.
IMAP allows email messages to be stored on the mail server in folders rather than
requiring them to be downloaded to a client. Thus IMAP makes the mailbox folders
available across the network to the user but places a larger storage burden on the mail server.

\noindent
\begin{center}\begin{pspicture}(0,0.2)(12,3.1)
  \rput(1,0.75){\ovalnode{S}{\makebox[10mm]{\scriptsize Sender}}}
  \rput(3,2.5){\ovalnode[fillcolor=white,fillstyle=solid]{MS}{\makebox[12mm]%
    {\scriptsize\begin{tabular}{c}Sender's\\Mail Server\end{tabular}}}}
  \rput(6,2.5){\rnode{I}{\scriptsize\begin{tabular}{c}Intermediate\\Mail Servers\end{tabular}}}
  \rput(9,2.5){\ovalnode[fillcolor=white,fillstyle=solid]{MR}{\makebox[12mm]%
    {\scriptsize\begin{tabular}{c}Recipient's\\Mail Server\end{tabular}}}}
  \rput(11,0.75){\ovalnode{R}{\makebox[10mm]{\scriptsize Recipient}}}
  \ncline[arrows=->,linecolor=red,linewidth=1mm]{S}{MS}\naput{\scriptsize SMTP}
  \ncline[arrows=->,linecolor=red,linewidth=1mm]{MS}{I}\naput{\scriptsize SMTP}
  \ncline[arrows=->,linecolor=red,linewidth=1mm]{I}{MR}\naput{\scriptsize SMTP}
  \ncline[arrows=->,linecolor=red,linewidth=1mm]{MR}{R}\naput{\scriptsize POP or IMAP}
\end{pspicture}\end{center}

Although email is one of the most heavily utilized network-based applications
it is usually completely insecure, providing no confidentiality nor authentication of
the user or message contents.
By default email messages are passed unencrypted between servers, any of which
could modify the message, replay an old message, or create a new message
with a fabricated sender identity.
Furthermore, each message includes the client URL and often information about
the client email program, which raises privacy issues.

There are two standard alternative approaches for providing confidentiality and/or
authentication of email, PGP and S/MIME.
\defn{Pretty Good Privacy} (PGP) relies on a \defn{web of trust},
rather than a centralized CA, to decide the validity of a received public key,
allowing the user to assign a level of trust to each public key it receives from other users.
Since its introduction in 1991 PGP has been very popular amongst Internet users
for providing secure personal email.
It offers five services which get applied to a message in the following order:
\begin{description}
  \item[authentication] with a hash code created using SHA-1 which is then
  encrypted using either DSA or RSA with the sender's private key,
  \item[compression] using ZIP compression to reduce the size of the transmitted message
  and eliminate repeated patterns in the plaintext,
  \item[confidentiality] using either triple DES or CAST-128 (a 64-bit block cipher
  that uses a 128-bit key), with a (single-use) secret session key that has been generated
  specifically for encrypting that message,
  \item[transfer encoding] using Base 64 encoding so that the compressed and encrypted
  message can be transmitted as text content via SMTP,
  \item[segmentation] of long messages into several smaller segments which are
  mailed separately and reassembled at the receiving end.
\end{description}
The content of a PGP message consists of three parts, a session key component, a signature
component, and the message itself.
Both the signature component and the message component are first compressed and then
encrypted using the secret session key, and then the entire message content
(including the session key component) gets encoded using Base 64 encoding.
In case the sender or recipient might have several public/private key pairs in use
the message includes a key identifer for the public key of each to help identify
which key has been used for encryption (simply providing the least significant
64 bits of the public key).\nudgedown{1}
\begin{center}
\begin{tabular}{|l|l|} \hline
  Session   & Least significant 64 bits of recipient's public key \\
  key       & Session key encrypted using recipient's public key \\ \hline
            & Timestamp \\
  Signature & Least significant 64 bits of sender's public key \\
            & First 16 bits of hash code \\
            & Hash code encrypted using sender's private key \\ \hline
            & Filename \\
  Message   & Timestamp \\
            & Message Data \\ \hline
\end{tabular}\nudgedown{1}
\end{center}
The \defn{Secure/Multipurpose Internet Mail Extension} (S/MIME) provides digital signatures
and encryption for the standard MIME mail format, which itself extends the message format
of RFC 822 to allow non-ASCII character sets, non-text attachments, and multi-part message bodies.
The following MIME types can be specified in the \begin{code}Content-type\end{code} header
field of a MIME message:
\begin{description}
  \item[application] such as \begin{code}application/java-archive\end{code} for
  JAR attachments and \begin{code}application/\\msword\end{code} for MS Word documents,
  \item[audio] such as \begin{code}audio/x-midi\end{code} and
  \begin{code}audio/x-wav\end{code} for audio data,
  \item[image] such as \begin{code}image/jpeg\end{code} and
  \begin{code}image/png\end{code} for image data,
  \item[message] such as \begin{code}message/partial\end{code} for fragmentation
  of large messages and \begin{code}message/external-\\body\end{code} for
  references to external content,
  \item[multi-part] such as \begin{code}multipart/mixed\end{code} for several
  independent message parts and \begin{code}multipart/\\alternative\end{code} for
  several alternative forms of the same information, where the parts are
  separated by specified boundary strings,
  \item[text] such as \begin{code}text/plain\end{code} for plain unformatted
  ASCII text and \begin{code}text/html\end{code} for HTML content,
  \item[video] such as \begin{code}video/mpeg\end{code} and
  \begin{code}video/quicktime\end{code} for video data.
\end{description}
MIME also allows the transfer encoding scheme, such as \begin{code}7bit\end{code}
(for no encoding) or \begin{code}base64\end{code},
to be specified in the \begin{code}Content-Transfer-Encoding\end{code} header field.
The S/MIME content type \begin{code}application/pkcs7-mime\end{code} is used to
specify a digital signature and/or encryption of a MIME message or part of it.
Signatures are created using SHA-1 (or the less secure MD5) to generate a hash code that
gets encrypted using the sender's private key and then Base 64 encoded.
As in PGP, encryption is performed using triple DES (or possibly AES) with a single-use secret
session key that has been generated specifically for encrypting that message,
and the resulting ciphertext is referred to as \defn{enveloped data}.

The \defn{JavaMail} API is included as part of Java Enterprise Edition, but is
also available as an optional API for Java Standard Edition.
Although TCP can be used directly to communicate with a mail server for sending
email via SMTP, and retrieving email via either POP or IMAP,
JavaMail can be used to facilitate the process, hiding the details of each protocol.
The JavaMail API requires the \defn{JavaBeans Activation Framework} (JAF) API,
which supports handling arbitrary blocks of data and determining their correct MIME types.
Surprisingly, version 1.5 of JavaMail does not yet support PGP nor S/MIME
(it is awaiting finalization of the S/MIME IETF specification), so encryption
and authentication must either be added to messages or else a third-party extension
to JavaMail used.
The \begin{code}Session\end{code} class in the \begin{code}javax.mail\end{code} package
defines a session with a mail server. Its static \begin{code}getDefaultInstance\end{code} method
is passed the mail server URL in a \begin{code}Properties\end{code} object and
an optional \begin{code}Authenticator\end{code} which prompts for the user name and
password as required (and which can be designed to have single sign-on capability),
returning a default (shared) mail session:
\begin{quote}\begin{code}\begin{verbatim}
Properties properties = System.getProperties();
properties.put("mail.smtp.host", HOST); // or pop3 or imap
Authenticator authenticator = ...; // or null if not required
Session session = Session.getDefaultInstance(properties,
   authenticator);
\end{verbatim}\end{code}\end{quote}
To send a message a \begin{code}MimeMessage\end{code} can be created
from the session and the \begin{code}Transport\end{code} method
\begin{code}send\end{code} used to send the message to the mail server via SMTP:
\begin{quote}\begin{code}\begin{verbatim}
MimeMessage message = new MimeMessage(session);
message.setFrom(...);
message.addRecipient(Message.RecipientType.TO, ...);
... // set content of message
Transport.send(message);
\end{verbatim}\end{code}\end{quote}
To retrieve mail from a mail server a \begin{code}Store\end{code} is used via either POP or IMAP:
\begin{quote}\begin{code}\begin{verbatim}
Store store = session.getStore("pop3");
store.connect();
\end{verbatim}\end{code}\end{quote}
which provides access to the mailbox folders on the mail server.
The example classes \begin{code}EMailSender\end{code} and \begin{code}EMailReceiver\end{code}
demonstrate how JavaMail can be used to send a multipart message and retrieve the
messages in the \begin{code}INBOX\end{code} folder.

\begin{figure*}\begin{program}\begin{verbatim}
/**
   A class that demonstrates the optional JavaMail API for sending
   email messages with multiple MIME parts in the body of the message
   Note this class needs the optional JavaMail to compile, place the
   JAR file mail.jar in classpath
   @author Andrew Ensor
*/
import java.util.Properties;
import java.util.Scanner; // Java 1.5 equivalent of cs1.Keyboard
import javax.activation.DataHandler;
import javax.activation.DataSource;
import javax.activation.FileDataSource;
import javax.mail.BodyPart;
import javax.mail.Message;
import javax.mail.MessagingException;
import javax.mail.Multipart;
import javax.mail.Session;
import javax.mail.Transport;
import javax.mail.internet.AddressException;
import javax.mail.internet.InternetAddress;
import javax.mail.internet.MimeBodyPart;
import javax.mail.internet.MimeMultipart;
import javax.mail.internet.MimeMessage;

public class EMailSender
{
   private static final String HOST = "smtp.aut.ac.nz";

   public static void main(String[] args)
   {  Properties properties = System.getProperties();
      properties.put("mail.smtp.host", HOST);
      // obtain the sender and recipient email addresses
      Scanner keyboardInput = new Scanner(System.in);
      System.out.print("Please enter sender address:");
      String from = keyboardInput.nextLine();
      System.out.print("Please enter recipient address:");
      String to = keyboardInput.nextLine();
      System.out.print("Please enter file to attach:");
      String filename = keyboardInput.nextLine();
      // obtain shared default mail session
      Session session = Session.getDefaultInstance(properties, null);
\end{verbatim}\hfill \emph{cont-}\end{program}\end{figure*}%
\begin{figure*}\begin{program}\emph{-cont}\begin{verbatim}
      // create a message
      MimeMessage message = new MimeMessage(session);
      try
      {  message.setFrom(new InternetAddress(from));
         message.addRecipient(Message.RecipientType.TO,
            new InternetAddress(to));
      }
      catch (AddressException e)
      {  System.err.println
            ("Address exception while preparing addresses: " + e);
      }
      catch (MessagingException e)
      {  System.err.println
            ("Messaging exception while preparing addresses: " + e);
      }
      try
      {  message.setSubject("Greeting");
         // prepare a MIME part that will default to text/plain
         BodyPart textPart = new MimeBodyPart();
         textPart.setText("Hello EMail World");
         // prepare a MIME part whose content type determined by JAF
         BodyPart filePart = new MimeBodyPart();
         DataSource fileSource = new FileDataSource(filename);
         filePart.setDataHandler(new DataHandler(fileSource));
         filePart.setFileName(filename);
         // prepare a multipart message using the two body parts
         Multipart multipart = new MimeMultipart();
         multipart.addBodyPart(textPart);
         multipart.addBodyPart(filePart);
         message.setContent(multipart);
         // send the message
         Transport.send(message);
         System.out.println("Message sent");
      }
      catch (MessagingException e)
      {  System.err.println("Messaging exception: " + e);
      }
   }
}
\end{verbatim}\end{program}\end{figure*}

\begin{figure*}\begin{program}\begin{verbatim}
/**
   A class that demonstrates the optional JavaMail API for receiving
   email messages from a POP or IMAP mail server
   Note this class needs the optional JavaMail to compile, place the
   JAR file mail.jar in classpath
   @author Andrew Ensor
*/
import java.util.Properties;
import javax.mail.Authenticator;
import javax.mail.Folder;
import javax.mail.Message;
import javax.mail.MessagingException;
import javax.mail.NoSuchProviderException;
import javax.mail.Session;
import javax.mail.Store;

public class EMailReceiver
{
   private static final String HOST = "pop.aut.ac.nz"; // AUT student

   public static void main(String[] args)
   {  Properties properties = System.getProperties();
      properties.put("mail.pop3.host", HOST);
      // obtain shared default mail session
      Authenticator authenticator = new SimpleAuthenticator();
      Session session = Session.getDefaultInstance(properties,
         authenticator);
      // connect to mailbox store
      Store store = null;
      try
      {  store = session.getStore("pop3"); // POP server
      }
      catch (NoSuchProviderException e)
      {  System.err.println("Mail provider not available: " + e);
      }
      Folder folder = null;
      try
      {  store.connect(); // use authenticator for user log in
         System.out.println("Connected to mail inbox");
         // obtain messages in specified folder
         folder = store.getFolder("INBOX");
         folder.open(Folder.READ_ONLY);
         Message[] messages = folder.getMessages();
         System.out.println("Messages currently in inbox:");
         for (int i=0; i<messages.length; i++)
         {  Message message = messages[i];
            System.out.println("Message from " + message.getFrom()[0]
               + " with subject " + message.getSubject());
         }
      }
\end{verbatim}\hfill \emph{cont-}\end{program}\end{figure*}%
\begin{figure*}\begin{program}\emph{-cont}\begin{verbatim}
      catch (MessagingException e)
      {  System.err.println("Messaging exception: " + e);
      }
      finally
      {  try
         {  if (folder != null)
               folder.close(false); // don't expunge deleted folders
            if (store != null)
               store.close();
         }
         catch (MessagingException e)
         {} // ignore
      }
   }
}
\end{verbatim}\end{program}\end{figure*}
\begin{figure*}\begin{program}\begin{verbatim}
/**
   A class that provides a simple Authenticator for EMailReceiver
   Note that this class does not obtain and store the password in
   a secure manner
   @see EMailReceiver.java
*/
import java.util.Scanner; // Java 1.5 equivalent of cs1.Keyboard
import javax.mail.Authenticator;
import javax.mail.PasswordAuthentication;

public class SimpleAuthenticator extends Authenticator
{
   private Scanner keyboardInput;

   public SimpleAuthenticator()
   {  keyboardInput = new Scanner(System.in);
   }

   public PasswordAuthentication getPasswordAuthentication()
   {  // prompt user for name and get name using Scanner
      System.out.print("Please enter user name:");
      String username = keyboardInput.nextLine();
      // prompt user for password and get using Scanner
      System.out.print("Please enter password:");
      String password = keyboardInput.nextLine(); // not secure here
      return new PasswordAuthentication(username, password);
   }
}
\end{verbatim}\end{program}\end{figure*}

\begin{exercise}[Adding a Message Signature]
Modify \begin{code}EMailSender\end{code} so that it adds
a digital signature as a MIME part to an email message.
\end{exercise}


%%%%%%%%%%%%%%%%%%%%%%%%%%%%%%%%%%%%%%%%%%%%%%%%%%%%%%%%%%%%%%%%%%%%%%%%%%%%%%%%%%%%%%%%%
%%%%% Web Security                                                                  %%%%%
%%%%%%%%%%%%%%%%%%%%%%%%%%%%%%%%%%%%%%%%%%%%%%%%%%%%%%%%%%%%%%%%%%%%%%%%%%%%%%%%%%%%%%%%%
\clearpage
\section{Web Security}\label{Se:WebSecurity}

\reading{pp484-498,528-549}

\begin{multicols}{2}
Using an authentication service such as Kerberos, or secure email via PGP or S/MIME,
presumes that the communication in a system is insecure, and so security is
provided at the application level. A complementary approach to ensuring security in a system is to secure the network
communication channels, either at the network protocol or at the transport protocol level.
By implementing security at the IP level, all network communication is made secure
transparently to the applications in the system.

\begin{pspicture}[shift=-1.2](0,-0.25)(6,2.2)
  \rput(3,2.7){\emph{Application Level Security}}
  \psline[fillcolor=yellow,fillstyle=solid](0,0)(0,0.6)(6,0.6)(6,0)(0,0)
  \rput(3,0.3){IP}
  \psline[fillcolor=yellow,fillstyle=solid](0,0.6)(0,1.2)(2,1.2)(2,0.6)(0,0.6)
  \psline[fillcolor=yellow,fillstyle=solid](2,0.6)(2,1.2)(6,1.2)(6,0.6)(2,0.6)
  \rput(1,0.9){UDP}\rput(4,0.9){TCP}
  \psline[fillcolor=lightred,fillstyle=solid](1,1.2)(1,1.8)(2,1.8)(2,1.2)(1,1.2)
  \rput(1.5,1.6){\scriptsize Kerb}\rput(1.5,1.375){\scriptsize -eros}
  \psline[fillcolor=yellow,fillstyle=solid](2,1.2)(2,1.8)(4,1.8)(4,1.2)(2,1.2)
  \psline[fillcolor=yellow,fillstyle=solid](4,1.2)(4,1.8)(6,1.8)(6,1.2)(4,1.2)
  \rput(3,1.5){HTTP}\rput(5,1.5){SMTP}
  \psline[fillcolor=lightred,fillstyle=solid](4,1.8)(4,2.4)(5,2.4)(5,1.8)(4,1.8)
  \psline[fillcolor=lightred,fillstyle=solid](5,1.8)(5,2.4)(6,2.4)(6,1.8)(5,1.8)
  \rput(4.5,2.1){\scriptsize PGP}\rput(5.5,2.1){\scriptsize S\hspace{-0.5mm}/\hspace{-0.5mm}MIME}
\end{pspicture}
\end{multicols}
\begin{multicols}{2}
\defn{IP security} (IPSec) provides authentication, confidentiality, and key management
capabilities for IP packets (optional for IPv4 but mandatory for the newer IPv6).
It is implemented as extension headers that follow the main IP header in each packet.
Encryption is provided by a cipher such as triple DES or Blowfish, and authentication
by HMAC with either MD5 or SHA-1 (but producing a 96 bit MAC by default).
Secret keys are distributed using a version of Diffie-Hellman key exchange known
as \defn{Oakley key exchange}, which provides added security by using cookies to
counter clogging attacks, nonces for replay attacks, and authentication for
man-in-the-middle attacks.\nudgedown{1}

\begin{pspicture}[shift=-1.2](0,-0.25)(6,2.2)
  \rput(3,2.1){\emph{Network Level Security}}
  \psline[fillcolor=lightred,fillstyle=solid](0,0)(0,0.6)(6,0.6)(6,0)(0,0)
  \rput(3,0.3){IP/IPSec}
  \psline[fillcolor=yellow,fillstyle=solid](0,0.6)(0,1.2)(2,1.2)(2,0.6)(0,0.6)
  \psline[fillcolor=yellow,fillstyle=solid](2,0.6)(2,1.2)(6,1.2)(6,0.6)(2,0.6)
  \rput(1,0.9){UDP}\rput(4,0.9){TCP}
  \psline[fillcolor=yellow,fillstyle=solid](2,1.2)(2,1.8)(4,1.8)(4,1.2)(2,1.2)
  \psline[fillcolor=yellow,fillstyle=solid](4,1.2)(4,1.8)(6,1.8)(6,1.2)(4,1.2)
  \rput(3,1.5){HTTP}\rput(5,1.5){SMTP}
\end{pspicture}
\end{multicols}
\begin{multicols}{2}
The \defn{Secure Sockets Layer} (SSL) and its Internet standard version
\defn{Transport Layer Security} (TLS) provide confidentiality and authentication
for applications that use TCP communication.
It establishes an \defn{SSL session} between a client and a server using a set of security parameters, which are shared for several secure
\defn{SSL connections} between the client and server.
The parameters specify a session identifier, the algorithms used for compression
and encryption, and hash functions used for MAC calculations. They also
include a secure 384-bit \defn{master secret} shared by both the client and server.
This value is calculated from a 384-bit \defn{pre-master-secret}
(either by the client and sent to the server encrypted using the RSA public key
of the server, or else via Diffie-Hellman key exchange) using a combination
of MD5 and SHA-1, and is used to determine any secret keys and
initialization vectors required during the session.

\begin{pspicture}[shift=-1.3](0,-0.25)(6,2.4)
  \rput(3,2.7){\emph{Transport Level Security}}
  \psline[fillcolor=yellow,fillstyle=solid](0,0)(0,0.6)(6,0.6)(6,0)(0,0)
  \rput(3,0.3){IP}
  \psline[fillcolor=yellow,fillstyle=solid](0,0.6)(0,1.2)(2,1.2)(2,0.6)(0,0.6)
  \psline[fillcolor=yellow,fillstyle=solid](2,0.6)(2,1.2)(6,1.2)(6,0.6)(2,0.6)
  \rput(1,0.9){UDP}\rput(4,0.9){TCP}
  \psline[fillcolor=lightred,fillstyle=solid](2,1.2)(2,1.8)(6,1.8)(6,1.2)(2,1.2)
  \rput(4,1.5){SSL or TLS}
  \psline[fillcolor=yellow,fillstyle=solid](2,1.8)(2,2.4)(4,2.4)(4,1.8)(2,1.8)
  \psline[fillcolor=yellow,fillstyle=solid](4,1.8)(4,2.4)(6,2.4)(6,1.8)(4,1.8)
  \rput(3,2.1){HTTP}\rput(5,2.1){SMTP}
\end{pspicture}
\end{multicols}



Both SSL and TLS use X.509 certificates to provide authentication.
In order for the server to authenticate itself it must deliver a valid certificate
to the client, for which the client can establish a certificate chain to a certificate that
it already trusts, and then the server must prove that it is indeed the subject
of the certificate by successfully decrypting something (such as the pre-master-secret),
which can only be done by the holder of the private key.
Optionally, the client too can be made to authenticate
itself to the server using its certificate.
The authentication steps are automated by the protocol so long as the server and client
have appropriate certificates.

SSL and TLS process data before they get sent via TCP by breaking the data into fragments
of up to 16384 bytes and applying the following steps to each fragment:
\begin{itemize}
  \item firstly the fragment can optionally be compressed,
  \item a 128-bit or 160-bit MAC is calculated for the compressed data
  (TLS uses HMAC with either MD5 or SHA-1, whereas SSL uses an older variant of HMAC),
  and is appended to the compressed fragment,
  \item the entire fragment (including the MAC) is then encrypted using a block
  cipher such as AES, DES, or triple DES (padding the fragment so that the final
  padding byte gives the number of bytes of padding added),
  or else using the RC4 stream cipher (with no padding required),
  \item a five byte SSL header is prepended to the fragment to indicate the type of SSL fragment,
  the major protocol version (currently \begin{code}3\end{code} for both SSL and TLS),
  the minor protocol version (\begin{code}0\end{code} for SSL and currently
  \begin{code}1\end{code} for TLS), and two bytes that give the length of the fragment.
\end{itemize}
The resulting fragments are then communicated via TCP.
There are four types of SSL fragment that can be communicated:
\begin{description}
  \item[application data] where the data in the fragment has originated from an
  application that is using an application level protocol such as HTTP or SMTP for communication,
  \item[change cipher] where the data is a single fixed byte \begin{code}01\end{code},
  indicating that the cipher suite for the connection is to be updated with
  the next pending values,
  \item[alert] where the data consists of two alert bytes, the first byte a
  \begin{code}01\end{code} (a warning) or \begin{code}02\end{code}
  (a fatal error which results in the connection being terminated),
  and the second byte a code to indicate the specific alert
  (such as a warning alert that a certificate was not provided during
  authentication, or a fatal alert about an incorrect MAC),
  \item[handshake] where the client and server negotiate details for establishing the connection,
  consisting of a byte to indicate one of 10 possible types of message,
  three bytes to indicate the length of parameters, followed by the parameters
  (such as an encrypted key or an X.509 certificate).
\end{description}
The handshaking is used to have the server authenticate itself to
the client (and possibly also the client to the server), create the master secret,
and establish which encryption, compression, and MAC algorithms will be used during
communication.
Handshaking proceeds in four phases:
\begin{description}
  \item[Establish security capabilities]
  The client initiates the handshake by sending a \begin{code}client\_hello\end{code}
  message requesting a protocol version, preferred security algorithms, and a nonce value,
  which the server responds to with a \begin{code}server\_hello\end{code}
  message specifying its choice of algorithms, and its own nonce value.
  \item[Server authentication and key exchange]
  The server then sends a \begin{code}certi\-ficate\end{code} message holding a
  certificate for the server, then (if required by the encryption algorithm) a
  \begin{code}server\_key\_exchange\end{code}
  containing server key parameters and a signature
  using the nonce values, an optional \begin{code}certifi\-cate\_request\end{code}
  message requesting a certificate from the client, followed by a
  \begin{code}server\_done\end{code} message.
  \item[Client authentication and key exchange]
  The client continues by sending a \begin{code}certificate\end{code} message if requested,
  then a compulsory \begin{code}client\_key\_exch\-ange\end{code} containing client key parameters
  and a signature, optionally followed by a \begin{code}certificate\_verify\end{code}
  message that uses the client's private key to encrypt a hash code of the handshake message
  (to prove that the client does possess the private key corresponding to the public key
  provided in the certificate).
  \item[Finish handshake] Finally the client and server notify each other to start using
  the cipher. The client sends a \begin{code}change\_cipher\_spec\end{code}
  message and a \begin{code}finished\end{code} message, and the server responds with
  a \begin{code}change\_cipher\_spec\end{code} message and its own
  \begin{code}finished\end{code} message.
\end{description}

The \defn{Java Secure Socket Extension} (JSSE) is an API included with Java Standard Edition
that provides an implementation of SSL/TLS.
It includes functionality for data encryption, server authentication, message integrity,
and optional client authentication, hiding the protocol details for cipher suite negotiation
and the initial handshaking.
A keystore that holds trusted certificates used to verify received certificates
during handshaking is called a \defn{truststore}.
The keystore and/or truststore to be used by JSSE
can either be specified as command line arguments when the client or server is run:
\begin{quote}\begin{code}
java -Djavax.net.ssl.trustStore=\emph{mytruststore.jks} \\
\trind -Djavax.net.ssl.trustStorePassword=\emph{mytrustpw} \\
\trind -Djavax.net.ssl.keyStore=\emph{mykeystore.jks} \\
\trind -Djavax.net.ssl.keyStorePassword=\emph{mystorepw} \emph{WhateverClass}
\end{code}\end{quote}
or else specified via the \begin{code}System\end{code} method \begin{code}setProperty\end{code}:
\begin{quote}\begin{code}\begin{verbatim}
System.setProperty("javax.net.ssl.trustStore", "...");
System.setProperty("javax.net.ssl.trustStorePassword","...");
System.setProperty("javax.net.ssl.keyStore", "...");
System.setProperty("javax.net.ssl.keyStorePassword", "...");
\end{verbatim}\end{code}\end{quote}
Once JSSE is provided with a truststore an SSL/TLS client can be
implemented similarly to an (insecure) TCP client by creating an \begin{code}SSLSocket\end{code}
via an \begin{code}SSLSocketFactory\end{code}:
\begin{quote}\begin{code}\begin{verbatim}
SSLSocketFactory sslFactory
   = (SSLSocketFactory)SSLSocketFactory.getDefault();
SSLSocket socket
   = (SSLSocket)sslFactory.createSocket(HOST_NAME,HOST_PORT);
socket.startHandshake(); //optional,happens anyway when flush
...
socket.close();
\end{verbatim}\end{code}\end{quote}
Also, once JSSE is provided with a keystore an SSL/TLS server can use an
\begin{code}SSLServerSocketFac-\\tory\end{code} to create an \begin{code}SSLServerSocket\end{code}:
\begin{quote}\begin{code}\begin{verbatim}
SSLServerSocketFactory sslServerFactory
   = (SSLServerSocketFactory)
   SSLServerSocketFactory.getDefault();
SSLServerSocket serverSocket = (SSLServerSocket)
   sslServerFactory.createServerSocket(PORT);
SSLSocket socket = (SSLSocket)serverSocket.accept();
socket.setWantClientAuth(true); // either none, want, or need
...
socket.close();
serverSocket.close();
\end{verbatim}\end{code}\end{quote}
Optionally, the cipher algorithms negotiated during the handshaking
can be restricted by specifying the allowable algorithms to either the
\begin{code}SSLServerSocket\end{code} or \begin{code}SSLSocket\end{code}:
\begin{quote}\begin{code}\begin{verbatim}
String[] enabledSuites = {"TLS_RSA_WITH_AES_128_CBC_SHA"};
socket.setEnabledCipherSuites(enabledSuites);
\end{verbatim}\end{code}\end{quote}

As an example, the classes \begin{code}SSLEchoClient\end{code} and
\begin{code}SSLEchoServer\end{code} demonstrate how SSL/TLS can be used
to ensure confidentiality and authentication of communication between a
TCP client and server.
The server uses a keystore called \begin{code}jssekeystore.jks\end{code}
to select a public/private key pair and send a certificate for the public key.
The client uses a truststore called \begin{code}jssetruststore.jks\end{code}
to check the received certificate against its own trusted certificates.
The following three command line statements can be used to create the keystore and
generate an RSA key pair, export a corresponding self-signed certificate,
and import the certificate to a new truststore
(so that it is immediately verified without actually building a certificate chain
to a CA):
\begin{quote}\begin{code}\begin{verbatim}
keytool -genkey -alias ssltest -keyalg RSA
   -keystore jssekeystore.jks
keytool -export -alias ssltest -file ssltest.cer
   -keystore jssekeystore.jks
keytool -import -alias ssltest -file ssltest.cer
   -keystore jssetruststore.jks
\end{verbatim}\end{code}\end{quote}
The communication between the client and server during handshaking can be displayed by
enabling the \emph{dynamic debug tracing} support, running the server via the
command line statement:
\begin{quote}\begin{code}\begin{verbatim}
java -Djavax.net.debug=SSL,handshake,data,trustmanager
   SSLEchoServer
\end{verbatim}\end{code}\end{quote}

As JSSE uses the cryptographic capabilities of the JCA, it can be configured
to use a cryptographic token such as a smart card or a hardware cryptographic accelerator
via PKCS\#11.
To use a smart card as a keystore or truststore the
\begin{code}javax.net.ssl.keyStoreType\end{code} or
\begin{code}javax.net.\\ssl.trustStoreType\end{code} system property
should be set to \begin{code}"pkcs11"\end{code}, and the
\begin{code}javax.net.ssl.key\\Store\end{code} or
\begin{code}javax.net.ssl.trustStore\end{code} system property to
\begin{code}"NONE"\end{code}.
Also, JSSE has support for Kerberos cipher suites provided that both the
SSL/TLS client and server have accounts with the Kerberos authentication server,
in which case JAAS and a Kerberos \begin{code}LoginModule\end{code} can be used
to obtain the necessary Kerberos credentials.

The \begin{code}java.net\end{code} class \begin{code}URL\end{code} supports
\defn{HTTP over SSL/TLS} (HTTPS), for using HTTP communication over
SSL/TLS for secure communication with web servers (using default port 443 instead
of the HTTP default port 80):
\begin{quote}\begin{code}\begin{verbatim}
		URL pageURL = new URL("https:...");
		HttpsURLConnection conn
		= (HttpsURLConnection)pageURL.openConnection();
		InputStream is = pageURL.openStream();
		\end{verbatim}\end{code}\end{quote}
The class \begin{code}HTTPSExample\end{code} is a simple example of
establishing an HTTPS connection.
Note that besides setting system properties for the truststore
it must also set system proxy properties if it accesses a web server from behind a firewall
(discussed in Section \ref{Se:Firewalls}).

\begin{exercise}[HTTPS With Generic Connection Framework]
	Use the Java Micro Edition class \begin{code}Connector\end{code} to open an
	\begin{code}HttpsConnection\end{code} to a web server and obtain details of
	the server certificate.
\end{exercise}

\begin{figure*}\begin{program}\begin{verbatim}
/**
   A class that represents a client that repeatedly sends any keyboard
   input securely to an SSLEchoServer until the user enters DONE
   @see SSLEchoServer.java
*/
import java.io.BufferedReader;
import java.io.InputStreamReader;
import java.io.IOException;
import java.io.PrintWriter;
import java.util.Scanner; // Java 1.5 equivalent of cs1.Keyboard
import javax.net.ssl.SSLSocket;
import javax.net.ssl.SSLSocketFactory;

public class SSLEchoClient
{
   public static final String HOST_NAME = "localhost";
   public static final int HOST_PORT = 8890; // host port number
   public static final String DONE = "done"; // terminates echo
   ...

   public void startClient()
   {  Scanner keyboardInput = new Scanner(System.in);
      System.setProperty("javax.net.ssl.trustStore",
         "jssetruststore.jks");
      System.setProperty("javax.net.ssl.trustStorePassword",
         "changeit");
      SSLSocketFactory sslFactory
         = (SSLSocketFactory)SSLSocketFactory.getDefault();
      SSLSocket socket = null;
      try
      {  socket
            = (SSLSocket)sslFactory.createSocket(HOST_NAME,HOST_PORT);
         System.out.println("Starting handshake");
         socket.startHandshake(); //optional,happens anyway when flush
      }
      catch (IOException e)
      {  System.err.println("Client could not make connection: " + e);
         System.exit(-1);
      }
      PrintWriter pw = null; // output stream to server
      BufferedReader br = null; // input stream from server
      ...
   }

   public static void main(String[] args)
   {  SSLEchoClient client = new SSLEchoClient();
      client.startClient();
   }
}
\end{verbatim}\end{program}\end{figure*}

\begin{figure*}\begin{program}\begin{verbatim}
/**
   A class that represents a server that continually echos any SSL/TLS
   TCP message securely back to the client until it receives DONE
   @author Andrew Ensor
*/
import java.io.BufferedReader;
import java.io.InputStreamReader;
import java.io.IOException;
import java.io.PrintWriter;
import java.net.InetAddress;
import java.net.Socket;
import java.net.SocketTimeoutException;
import javax.net.ssl.SSLServerSocket;
import javax.net.ssl.SSLServerSocketFactory;
import javax.net.ssl.SSLSocket;

public class SSLEchoServer
{
   private boolean stopRequested;
   public static final int PORT = 8890; // some unused port number
   public static final String DONE = "done"; // terminates echo

   public SSLEchoServer()
   {  stopRequested = false;
   }

   // start the server if not already started and repeatedly listen
   // for client connections until stop requested
   public void startServer()
   {  stopRequested = false;
      System.setProperty("javax.net.ssl.keyStore", "jssekeystore.jks");
      System.setProperty("javax.net.ssl.keyStorePassword", "changeit");
      SSLServerSocketFactory sslServerFactory
         =(SSLServerSocketFactory)SSLServerSocketFactory.getDefault();
      SSLServerSocket serverSocket = null;
      try
      {  serverSocket = (SSLServerSocket)
            sslServerFactory.createServerSocket(PORT);
         serverSocket.setSoTimeout(2000); // timeout for accept
         System.out.println("Server started at "
            + InetAddress.getLocalHost() + " on port " + PORT);
      }
      catch (IOException e)
      {  System.err.println("Server can't listen on port: " + e);
         System.exit(-1);
      }
\end{verbatim}\hfill \emph{cont-}\end{program}\end{figure*}%

\begin{figure*}\begin{program}\emph{-cont}\begin{verbatim}
      while (!stopRequested)
      {  // block until the next client requests a connection
         // or else the server socket timeout is reached
         try
         {  SSLSocket socket = (SSLSocket)serverSocket.accept();
            System.out.println("Secure connection made with "
               + socket.getInetAddress());
            // start an echo with this connection
            EchoConnection echo = new EchoConnection(socket);
            Thread thread = new Thread(echo);
            thread.start();
         }
         catch (SocketTimeoutException e)
         {  // ignore and try again
         }
         catch (IOException e)
         {  System.err.println("Can't accept client connection: " + e);
            stopRequested = true;
         }
      }
      try
      {  serverSocket.close();
      }
      catch (IOException e)
      {  // ignore
      }
      System.out.println("Server finishing");
   }

   // stops server AFTER the next client connection has been made
   // or timeout is reached
   public void requestStop()
   {  stopRequested = true;
   }

   // driver main method to test the class
   // note that it doesn't call requestStop so main does not exit
   public static void main(String[] args)
   {  SSLEchoServer server = new SSLEchoServer();
      server.startServer();
   }

   // inner class that represents a single echo connection
   private class EchoConnection implements Runnable
   {
      ...
   }
}
\end{verbatim}\end{program}\end{figure*}

\begin{figure*}\begin{program}\begin{verbatim}
/**
   A class that demonstrates how to open an HTTPS connection to a
   web server and obtain the response
   @author Andrew Ensor
*/
...
public class HTTPSExample
{
   public static void main(String[] args)
   {  // use the cacerts keystore included with Java Runtime
      Scanner keyboardInput = new Scanner(System.in);
      System.setProperty("javax.net.ssl.trustStore",
         "c:/j2sdk/jre/lib/security/cacerts");
      System.setProperty("javax.net.ssl.trustStorePassword",
         "changeit");
      // set the proxy host and port number for AUT's firewall
      // note this is not needed if no firewall present
      System.setProperty("https.proxySet","true");//true if using proxy
      System.setProperty("https.proxyHost", "cache.aut.ac.nz"); // AUT
      System.setProperty("https.proxyPort", "3128"); // AUT specific
      // obtain the URL for https
      String urlName = null;
      ...
      try
      {  URL pageURL = new URL(urlName);
         // set connection and read timeouts for the connection
         HttpsURLConnection conn
            = (HttpsURLConnection)pageURL.openConnection();
         conn.setConnectTimeout(5000); // 5000ms
         conn.setReadTimeout(5000); // 5000ms
         // create an input stream and read all the lines
         InputStream is = pageURL.openStream();
         BufferedReader br = new BufferedReader
            (new InputStreamReader(is));
         String line = br.readLine();
         while (line != null)
         {  System.out.println("Received: " + line);
            line = br.readLine();
         }
         br.close();
      }
      catch (MalformedURLException e) // not valid protocol
      {  System.err.println("Invalid protocol or URL: " + e);
      }
      catch (IOException e)
      {  System.err.println("IO Exception: " + e);
      }
   }
}
\end{verbatim}\end{program}\end{figure*}


%%%%%%%%%%%%%%%%%%%%%%%%%%%%%%%%%%%%%%%%%%%%%%%%%%%%%%%%%%%%%%%%%%%%%%%%%%%%%%%%%%%%%%%%%
%%%%% XML Web Service Security                                                          %%%%%
%%%%%%%%%%%%%%%%%%%%%%%%%%%%%%%%%%%%%%%%%%%%%%%%%%%%%%%%%%%%%%%%%%%%%%%%%%%%%%%%%%%%%%%%%
%\input{XML}
%
%
%
% <END networksecurity>
