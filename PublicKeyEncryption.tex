% This is the LaTeX file for the chapter:
%   Public-Key Encryption
% for the manual:
%   Highly Secure Systems
% Written by Andrew Ensor
% Last updated 5 June 2006.

\chapter{Public-Key Encryption}

%%%%%%%%%%%%%%%%%%%%%%%%%%%%%%%%%%%%%%%%%%%%%%%%%%%%%%%%%%%%%%%%%%%%%%%%%%%%%%%%%%%%%%%%%
%%%%% Modular Arithmetic                                                            %%%%%
%%%%%%%%%%%%%%%%%%%%%%%%%%%%%%%%%%%%%%%%%%%%%%%%%%%%%%%%%%%%%%%%%%%%%%%%%%%%%%%%%%%%%%%%%

\section{Modular Arithmetic}

\reading{pp235-253}

Besides its importance for AES, the modular arithmetic operations in the ring
$\mathbb{Z}_n$ have an important role in public-key encryption.
Calculating an arithmetic expression modulo $n$ can be simplified by taking the modulo of
each term before performing the arithmetic operations:
\begin{eqnarray*}
  (a+b)\mbox{ mod }n &=& ((a\mbox{ mod }n)+(b\mbox{ mod }n))\mbox{ mod }n \\
  (a-b)\mbox{ mod }n &=& ((a\mbox{ mod }n)-(b\mbox{ mod }n))\mbox{ mod }n \\
  (a\cdot b)\mbox{ mod }n &=& ((a\mbox{ mod }n)\cdot(b\mbox{ mod }n))\mbox{ mod }n.
\end{eqnarray*}
Prime numbers turn out to be very useful when performing modular arithmetic.
The following fundamental result demonstrates the importance of prime numbers in number theory.

\begin{theorem}{Unique Prime Factorization}
Every integer $a>1$ can be factorized in a unique way as a product of prime numbers:
\begin{displaymath}
  a = p_1^{a_1} p_2^{a_2} \dots p_t^{a_t}
\end{displaymath}
where $p_1<p_2<\cdots<p_t$ are prime numbers and the exponents
$a_1,a_2,\dots,a_t$ are positive integers.
\end{theorem}

For example:
\begin{eqnarray*}
  21600 &=& 2^5\cdot3^3\cdot5^2 \\
  49000 &=& 2^3\cdot5^3\cdot7^2 \\
  65535 &=& 3^1\cdot5^1\cdot17^1\cdot257^1 \\
  65536 &=& 2^{16} \\
  65537 &=& 65537^1 \qquad\mbox{ as $65537$ is itself prime}.
\end{eqnarray*}

As a consequence, if a prime $p$ divides a product $ab$ then it must be one of the primes
in the factorization of the product $ab$ and thus also be a prime in the factorization
of either $a$ or of $b$, so must divide either $a$ or $b$ (or both).

\begin{theorem}{Fermat's Theorem}
If $p$ is a prime number and $a$ is a positive integer not divisible by $p$ then
$a^{p-1} \mbox{ mod } p = 1$.
\end{theorem}

As an example, $7$ is a prime number that does not divide $21600$ (as $7$ does not
appear in its prime factorization). Hence $21600^6\mbox{ mod }7=1$.

To see why Fermat's Theorem is true consider the numbers $1, 2, \dots, p-1$, multiply
each number by $a$ and take the remainder modulo $p$ to get the set:
\begin{displaymath}
  \{a\mbox{ mod }p, 2a\mbox{ mod }p, \dots, (p-1)a\mbox{ mod }p\}.
\end{displaymath}
Note that none of the elements in this set are $0$, which can be
seen by using a proof by contradiction (if $ia\mbox{ mod }p=0$
for some $i$ with $1\leq i\leq p-1$ then $p$ divides $ia$, but $p$ does not
divide $a$ so $p$ must divide $i$, which is not possible).
Furthermore, all the elements of this set are distinct from each other,
which can also be seen by using a proof by contradiction
(if $ia\mbox{ mod }p = ja\mbox{ mod }p$  where
$1\leq i<j\leq p-1$ then $p$ must divide $(j-i)a$,
but as $a$ is not divisible by $p$, $p$ must be divide $j-i$,
which is not possible as both $i<p$ and $j<p$).
Hence this set is just a permutation of $\{1, 2, \dots, p-1\}$.
Multiplying all the elements together therefore gives:
\begin{eqnarray*}
  (a\mbox{ mod }p)\cdot(2a\mbox{ mod }p)\cdot\dots\cdot((p-1)a\mbox{ mod }p)
    &=& 1\cdot2\cdot\dots\cdot(p-1) \\
  (a\cdot2a\cdot\dots\cdot(p-1)a)\mbox{ mod }p
    &=& (p-1)!\mbox{ mod }p \\
  a^{p-1} (p-1)!\mbox{ mod }p &=& (p-1)!\mbox{ mod }p
\end{eqnarray*}
So $p$ divides $(a^{p-1}-1)(p-1)!$. As $p$ does not divide $(p-1)!$ it must divide
$a^{p-1}-1$ which verifies Fermat's Theorem.

Now, if $p$ is an odd prime number then $p-1$ is even and so
$p-1=2^k q$ for some $k\geq1$ and odd value of $q$.
Hence for any $a<p$ Fermat's Theorem gives that $a^{2^k q}\mbox{ mod }p=1$.
As a consequence since $p$ is prime, it can be shown that the values:
\begin{displaymath}
  a^q\mbox{ mod }p, a^{2q}\mbox{ mod }p, a^{2^2 q}\mbox{ mod }p,
  a^{2^3 q}\mbox{ mod }p, \dots, a^{2^k q}\mbox{ mod }p=1
\end{displaymath}
where each term is the square of the previous modulo $p$,
are either all $1$ or else one is equal to $-1\mbox{ mod }p=p-1$ and the next is $1$.
Interestingly, for an odd value of $n$ that is not a prime
if a random value of $a<n-1$ is chosen it has probability less than
$\frac{1}{4}$ of satisfying this condition.
This gives a simple $O(\log_2 n)$ technique for determining whether a number is
\defn{probably prime} which can be much more efficient than the brute-force $O(\sqrt{n})$
approach of checking whether any of the numbers
$2, 3, \dots, \sqrt{n}$ divide $n$.
If $\proc{Probably-Prime}(n)$ returns $\const{false}$ then $n$ is definitely
not a prime, whereas if it returns $\const{true}$ then the probability
that it is a prime is at least $\frac{3}{4}$.

\begin{figure*}[htb]
\begin{codebox}
\Procname{$\proc{Probably-Prime}(n)$}
\li \Comment determine whether the odd value of $n>2$ is probably a prime
\li Find integers $k$ and $q$ for which $n-1=2^k q$
\li \Comment pick a random integer between $2$ and $n-1$ exclusive to use in test
\li $a \gets \proc{Random(2, $n-1$)}$
\li $\id{current} \gets a^q\mbox{ mod }n$
\li \If $\id{current}=1$ \Then
\li    \Return \const{true} \`\Comment $n$ is probably a prime
    \End
\li \For $i \gets 1$ \To $k$ \Do
\li    $\id{next} \gets \id{current}^2\mbox{ mod }n$
\li    \If $\id{next}=1$ \Then
\li       \If $\id{current}\neq n-1$ \Then
\li          \Return \const{false} \`\Comment $n$ is definitely not a prime
\li       \Else
\li          \Return \const{true} \`\Comment $n$ is probably a prime
          \End
       \End
\li    $\id{current}\gets\id{next}$
    \End
\li \Comment After end of loop it must be that $\id{current}\neq1$
\li \Return \const{false} \`\Comment $n$ is definitely not a prime
\end{codebox}
\begin{codebox}
\Procname{$\proc{Miller-Rabin}(n,s)$}
\li \If $n=2$ \Then
\li    \Return \const{true}
\li  \ElseIf  $n\mbox{ mod }2=0$ \Then
\li    \Return \const{false} \`\Comment $n$ is even
    \End
\li \Comment test whether $n$ is a prime $s$ times
\li \For $j\gets1$ \To $s$ \Do
\li    \If not $\proc{Probably-Prime}(n)$ \Then
\li       \Return \const{false} \`\Comment $n$ is definitely not a prime
       \End
    \End
\li \Return \const{true} \`\Comment $n$ is probably a prime with prob at least $1-\frac{1}{4^s}$
\end{codebox}
\end{figure*}
The \defn{Miller-Rabin} algorithm simply repeatedly tests $\proc{Probably-Prime}(n)$
$s$ times, and will mistakenly claim that an odd value of $n$ is a prime
with probability less than $\frac{1}{4^s}$. Thus the algorithm can be
used to determine with any desired probability whether a value of $n$ is a
prime, simply by choosing a large enough value of $s$.
In fact, if $n$ is a randomly chosen large number it can be shown that
taking $s=3$ is very unlikely to lead to a false result
(the non-prime \defn{Carmichael numbers} such as $561$, $1105$, $1729$
do always get mistaken by the Miller-Rabin algorithm,
but they are extremely rare with only 255 such numbers less than $100000000$).

If $n$ is a positive integer then the \defn{Euler totient function} $\phi(n)$
is the number of positive integers $m$ less than $n$ that are
\defn{relatively prime} to $n$, meaning that $\mbox{gcd}(m,n)=1$.
By convention one takes $\phi(1)=1$.

\begin{figure*}[htb]\begin{center}
\begin{tabular}{|r|r||r|r||r|r||r|r||r|r|} \hline
  \multicolumn{10}{|l|}{\textit{Euler Totient Function for Small Values of $n$}} \\ \hline\hline
  \textit{$n$} & \textit{$\phi(n)$} & \textit{$n$} & \textit{$\phi(n)$}
  & \textit{$n$} & \textit{$\phi(n)$} & \textit{$n$} & \textit{$\phi(n)$}
  & \textit{$n$} & \textit{$\phi(n)$} \\ \hline
  1 & 1 & 2 & 1 & 3 & 2 & 4 & 2 & 5 & 4 \\
  6 & 2 & 7 & 6 & 8 & 4 & 9 & 6 & 10 & 4 \\
  11 & 10 & 12 & 4 & 13 & 12 & 14 & 6 & 15 & 8 \\
  16 & 8 & 17 & 16 & 18 & 6 & 19 & 18 & 20 & 8 \\
  21 & 12 & 22 & 10 & 23 & 22 & 24 & 8 & 25 & 20 \\
  26 & 12 & 27 & 18 & 28 & 12 & 29 & 28 & 30 & 8 \\\hline
\end{tabular}
\end{center}\end{figure*}

For example, to find $\phi(18)$ one lists the numbers $1$ to $17$ and removes
those that are not relatively prime to $18$:
\begin{displaymath}
  \mathbf{1}, \mathit{2}, \mathit{3}, \mathit{4}, \mathbf{5}, \mathit{6},
  \mathbf{7}, \mathit{8}, \mathit{9}, \mathit{10}, \mathbf{11}, \mathit{12},
  \mathbf{13}, \mathit{14}, \mathit{15}, \mathit{16}, \mathbf{17}.
\end{displaymath}
Only $6$ numbers remain and so $\phi(18)=6$.

If $p$ is a prime then clearly all positive integers less than $p$ are relatively
prime to $p$ and so $\phi(p)=p-1$. Also, it is not difficult to show that
if $p$ and $q$ are distinct primes then $\phi(pq)=(p-1)(q-1)$.
Determining $\phi(n)$ for most other values of $n$ can be shown to be about as difficult
as is determining the unique prime factorization of $n$, which is usually quite difficult
for large values of $n$.

The following result generalizes Fermat's Theorem for the case when arithmetic
is taken modulo a value $n$ that might not be a prime. It can be proved in almost
exactly the same way as Fermat's Theorem, but by starting with the $\phi(n)$ numbers
that are relatively prime to $n$.

\begin{theorem}{Euler's Theorem}
  If $n$ and $a$ are positive integers that are relatively prime then
  $a^{\phi(n)}\mbox{ mod }n = 1$.
\end{theorem}

For example, $18$ and $35$ are relatively prime as $\mbox{gcd}(18,35)=1$
(this can also be seen as their prime factorizations have no primes in common).
As $\phi(18)=6$ Euler's Theorem gives that $35^6\mbox{ mod }18=1$.

The following result is also useful for performing modular arithmetic.
It states that arithmetic in $\mathbb{Z}_n$ can be performed by using (simpler)
arithmetic in $\mathbb{Z}_{n_i}$ for some relatively prime divisors $n_i$ of $n$.

\begin{theorem}{Chinese Remainder Theorem}
  Suppose $n=n_1n_2\dots n_k$ where $n_1, n_2, \dots, n_k$ are all relatively prime,
  and define the function
  $f\colon\mathbb{Z}_n\to\mathbb{Z}_{n_1}\times\mathbb{Z}_{n_2}\times\dots\times\mathbb{Z}_{n_k}$
  by:
  \begin{displaymath}
    f(a) = \left( a\mbox{ mod }n_1, a\mbox{ mod }n_2, \dots, a\mbox{ mod }n_k \right).
  \end{displaymath}
  Then $f$ is a bijection for which:
  \begin{eqnarray*}
    f(a+b) &=& f(a)+f(b) \\
    f(a-b) &=& f(a)-f(b) \\
    f(a\cdot b) &=& f(a)\cdot f(b).
  \end{eqnarray*}
  where arithmetic in $\mathbb{Z}_n$ is taken modulo $n$ and in
  $\mathbb{Z}_{n_1}\times\mathbb{Z}_{n_2}\times\dots\times\mathbb{Z}_{n_k}$ is taken
  modulo $n_i$ in each coordinate.
  The inverse $f^{-1}\colon
  \mathbb{Z}_{n_1}\times\mathbb{Z}_{n_2}\times\dots\times\mathbb{Z}_{n_k}\to\mathbb{Z}_n$
  is given by:
  \begin{displaymath}
    f^{-1}\left( a_1, a_2, \dots, a_k \right)
    = \left( a_1c_1+a_2c_2+\cdots+a_kc_k \right)\mbox{ mod }n,
  \end{displaymath}
  where each $c_i=(n/n_i)\cdot d_i$
  and $d_i$ is the value in $\mathbb{Z}_{n_i}$ for which
  $(n/n_i)\cdot d_i\mbox{ mod }n_i=1$ ($d_i$ can be obtained by the Euclidean algorithm).
\end{theorem}

To illustrate the Chinese Remainder Theorem, suppose $a$ is a value for which
$a\mbox{ mod }5=3$ and $a\mbox{ mod }12=7$. To determine the value of $a$
(which is unique modulo $5\cdot12=60$) one approach would be to make a table of all values
between $0$ and $59$ where the rows have the same remainder modulo $5$ and
the columns have the same remainder $12$. Then the correct value of $a$
can be read off the row with remainder $3$ and column with remainder $7$,
giving $a=43$.
\begin{figure*}[htb]\begin{center}
\begin{tabular}{|c||rrrrrrrrrrrr|} \hline
  \multicolumn{13}{|l|}{\textit{Values of $a$ between $0$ and $59$
    with remainders modulo $5$ and modulo $12$}} \\ \hline\hline
  \textit{Remainders} & \textit{$0$} & \textit{$1$} & \textit{$2$} & \textit{$3$}
  & \textit{$4$} & \textit{$5$} & \textit{$6$} & \textit{$7$}
  & \textit{$8$} & \textit{$9$} & \textit{$10$} & \textit{$11$} \\ \hline\hline
  \textit{$0$} & 0 & 25 & 50 & 15 & 40 & 5 & 30 & 55 & 20 & 45 & 10 & 35 \\
  \textit{$1$} & 36 & 1 & 26 & 51 & 16 & 41 & 6 & 31 & 56 & 21 & 46 & 11 \\
  \textit{$2$} & 12 & 37 & 2 & 27 & 52 & 17 & 42 & 7 & 32 & 57 & 22 & 47 \\
  \textit{$3$} & 48 & 13 & 38 & 3 & 28 & 53 & 18 & 43 & 8 & 33 & 58 & 23 \\
  \textit{$4$} & 24 & 49 & 14 & 39 & 4 & 29 & 54 & 19 & 44 & 9 & 34 & 59 \\\hline
\end{tabular}\end{center}
\end{figure*}
Alternatively, as $n_1=5$ and $n_2=12$ are relatively prime the Chinese Remainder Theorem
can be used to find $a=f^{-1}(3,7)$. Let $n=n_1\cdot n_2=60$,
so that $n/n_1=12$ and $n/n_2=5$. Then $d_1$ is the value for which
$12\cdot d_1\mbox{ mod }5=1$, which by the Euclidean algorithm can be found to be
$d_1=3$. Similarly, $d_2$ is the value for which $5\cdot d_2\mbox{ mod }12=1$,
which is $d_2=5$. Then $c_1=(n/n_1)\cdot d_1=12\cdot3=36$ and
$c_2=(n/n_2)\cdot d_2=5\cdot5=25$.
Now any value of $f^{-1}(a_1,a_2)$ can be found, such as
$a=f^{-1}(3,7)=(3\cdot36+7\cdot25)\mbox{ mod }60=43$.


\begin{exercise}[Random Prime Generator]
  Prepare a program that can generate a cryptographically strong large random prime number
  (for a very large number you might like to make use of the
  \begin{code}BigInteger\end{code} class that is in the \begin{code}java.math\end{code}
  package).
\end{exercise}


%%%%%%%%%%%%%%%%%%%%%%%%%%%%%%%%%%%%%%%%%%%%%%%%%%%%%%%%%%%%%%%%%%%%%%%%%%%%%%%%%%%%%%%%%
%%%%% RSA Algorithm                                                                 %%%%%
%%%%%%%%%%%%%%%%%%%%%%%%%%%%%%%%%%%%%%%%%%%%%%%%%%%%%%%%%%%%%%%%%%%%%%%%%%%%%%%%%%%%%%%%%

\section{RSA Algorithm}\label{Se:RSAAlgorithm}

\reading{pp258-280}

The concept of public-key cryptography was introduced as a way to
distribute secret keys without resorting to physical distribution,
and as a way to authenticate the sender of a message.
Public-key cryptography uses two keys, a \defn{public key} which is
usually publicly available and a \defn{private key} which must
be kept confidential. Encrypting plaintext using one of the keys
results in ciphertext which should only be able to be decrypted
with the other key. Hence the holder of the private key should be the
only one that can decrypt messages encoded by anyone using the public key,
enabling others to encrypt messages that they send to the holder.
The holder should also be the only one that can encrypt messages that
get decrypted by anyone using the public key, enabling others
to authenticate that a message was produced by the holder
(provided they are certain of the origin of the public key).

One of the first public-key ciphers and currently the most-popular
public key block cipher
was developed by Rivest, Shamir, Adleman and known as the \defn{RSA algorithm}.
The public and private keys are calculated with the following steps:
\begin{enumerate}
  \item two large prime numbers $p$ and $q$ are selected at random,
  typically each at least 512 bits,
  \item the product $n=pq$ is calculated, allowing encryption
  of any plaintext block whose value is in $\mathbb{Z}_n$
  (typically at least 1024 bits),
  \item a small odd integer $e$ that is relatively prime to
  $\phi(n)=(p-1)(q-1)$ is selected, a common choice is $e=65537$ so long
  as $\mbox{gcd}(\phi(n),e)=1$,
  \item the multiplicative inverse $d$ of $e$ in $\mathbb{Z}_{\phi(n)}$ is calculated
  using the Euclidean algorithm (which is the value $s$ calculated by
  $\proc{Extended-Euclid}(e, \phi(n))$),
  \item the public key is the pair $(e, n)$ which is published,
  \item the private key is the pair $(d, n)$ which is kept secret.
\end{enumerate}
\begin{figure*}[htb]
\begin{codebox}
\Procname{$\proc{Extended-Euclid}(m,n)$}
\li \Comment determine $d=\mbox{gcd}(m,n)$ and $s$, $t$ for which
   $\mbox{gcd}(m,n)=s\cdot m+t\cdot n$
\li \If $n=0$ \Then
\li    \Return $(m, 1, 0)$
    \End
\li $(d^\prime, s^\prime, t^\prime) \gets \proc{Extended-Euclid}(n,m\mbox{ mod }n)$
\li $q \gets \lfloor m/n\rfloor$ \`\Comment integer division $m/n$
\li $(d, s, t) \gets (d^\prime, t^\prime, s^\prime - q t^\prime)$
\li \Return $(d, s, t)$
\end{codebox}
\end{figure*}

A plaintext block $P$ is considered a value in $\mathbb{Z}_n$.
Encryption of $P$ using the public key into a ciphertext value $C$ is given by:
\begin{displaymath}
  C = P^e\mbox{ mod }n,
\end{displaymath}
whereas for authentication the private key $d$ would instead be used for encryption.
Decryption is then achieved by the same technique but with the other key:
\begin{displaymath}
  P = C^d\mbox{ mod }n.
\end{displaymath}
To verify that encryption and decryption are indeed inverse operations, it must be
shown that $\left(P^e\mbox{ mod }n\right)^d\mbox{ mod }n=P$
for any possible value $P$ in $\mathbb{Z}_n$.
Note that $ed\mbox{ mod }\phi(n)=1$ and $\phi(n)=(p-1)(q-1)$ as $p$ and $q$ are
both primes, so that $ed=1+k(p-1)(q-1)$ for some multiple $k$. Thus:
\begin{displaymath}
  \left(P^e\mbox{ mod }n\right)^d\mbox{ mod }n
  = P^{ed}\mbox{ mod }n
  = P^{1+k(p-1)(q-1)}\mbox{ mod }n.
\end{displaymath}
Now, if $P$ is not divisible by $p$ then $P^{p-1}\mbox{ mod }p=1$ by Fermat's Theorem, so:
\begin{eqnarray*}
  P^{1+k(p-1)(q-1)}\mbox{ mod }p
  &=& P\cdot\left(P^{p-1}\right)^{k(q-1)}\mbox{ mod }p \\
  &=& P\cdot\left(1\right)^{k(q-1)}\mbox{ mod }p \\
  &=& P\mbox{ mod }p.
\end{eqnarray*}
If instead $P$ is divisible by $p$ then too $P^{1+k(p-1)(q-1)}\mbox{ mod }p=0=P\mbox{ mod }p$.
Similarly, $P^{1+k(p-1)(q-1)}\mbox{ mod }q=P\mbox{ mod }q$, so the Chinese Remainder Theorem
can be applied to show that $P^{1+k(p-1)(q-1)}\mbox{ mod }n=P\mbox{ mod }n$.
This verifies that RSA decryption is the reverse of RSA encryption.

The exponentiation operations $P^e$ and $C^d$ used in encryption and decryption
can be performed in $\Theta(\log_2 e)$ and $\Theta(\log_2 d)$ respectively.
For instance $P^e\mbox{ mod }n$ can be found by calculating:
\begin{displaymath}
  P, P^2\mbox{ mod }n, P^4\mbox{ mod }n, P^8\mbox{ mod }n, P^{16}\mbox{ mod }n, \dots
\end{displaymath}
where each term is the square modulo $n$ of the previous term.
The algorithm $\proc{Modular}$-$\proc{Exponentiation}$ demonstrates how this can be achieved
using the binary representation of an exponent $b$.
For example, if $b=22$ which has binary representation $10110$, then
after the first iteration of the loop $b_4=1$ and so $t=a\cdot1=a$.
During the next iteration $b_3=0$ and so $t$ is just squared giving
$t=a^2$. During the third iteration $b_2=1$ and so
$t$ is squared and then multiplied by $a$ giving $t=a^5$.
During the fourth iteration $b_1=1$ and so $t$ is again squared and
multiplied by $a$, resulting in $t=a^{11}$. In the final iteration
$b_0=0$ and so the result is just squared, giving $t=a^{22}$.
\begin{figure*}[htb]
\begin{codebox}
\Procname{$\proc{Modular-Exponentiation}(a,b,n)$}
\li \Comment determine $a^b\mbox{ mod }n$ in order $\Theta(\log_2 b)$
       where $b=b_{k-1} b_{k-2}\dots b_0$ in binary
\li $t \gets 1$
\li \For $i\gets k\!-\!1$ \Downto $0$ \Do
\li    $t \gets t\times t\mbox{ mod }n$
\li    \If $b_i=1$ \Then
\li       $t \gets (t\times a)\mbox{ mod }n$
       \End
    \End
\li \Return $t$
\end{codebox}
\end{figure*}
Actually, since the exponent $d$ is probably very large a more efficient
way of calculating $C^d$ during decryption can be advantageous.
Using Fermat's Theorem, $C^d\mbox{ mod }p=C^{d\mbox{ \scriptsize mod }(p-1)}\mbox{ mod }p$
and $C^d\mbox{ mod }q=C^{d\mbox{ \scriptsize mod }(q-1)}\mbox{ mod }q$.
So $C^d\mbox{ mod }p$ and $C^d\mbox{ mod }q$ can be found relatively quickly.
Then as $n=pq$ and $p$ and $q$ are relatively prime,
the Chinese Remainder Theorem can be applied to obtain $C^d\mbox{ mod }n$.
This way is about four times faster than simply using $\proc{Modular-Exponentiation}$
to calculate $C^d\mbox{ mod }n$.

It may appear surprising that $n$ can be included as part of the public key,
as if a cryptanalyst were to find the primes $p$ and $q$ with $n=pq$, or even just
calculate $\phi(n)$ then the private key would be obtained.
However, obtaining the prime factorization for large values of $n$ is well-known
to be difficult, with no efficient algorithm yet discovered.
Also, it has been shown that calculating $\phi(n)$ is equivalent to
factorizing $n=pq$.
Hence the security of the RSA algorithm is based on the difficulty
of factorizing a sufficiently large value of $n$.

Cryptanalysis attacks on RSA based on factorizing $n$ have improved since the
release of RSA. When RSA was first published in 1977 a challenge was made
to break a particular ciphertext encrypted using a 428 bit value for $n$,
which at the time was presumed would take 40 quadrillion years by brute force.
But in 1994, after only eight months work the ciphertext was successfully
broken, and since then factorization techniques have improved substantially,
so that by 2005 a 663 bit value of $n$ was no longer considered secure.
Hence larger values (currently between 1024 bit and 2048 bit) are now used for
RSA encryption. However, the drawback is that encryption and decryption
calculations are longer, slowing the algorithm.

There are several other interesting cryptanalysis attacks on RSA.
If a particularly small value of $e$ is chosen and if the same plaintext is encrypted
for that number of users each with different values of $n$ then the plaintext
can be easily obtained. For instance, if $e=3$ (which was once commonly used)
and the same plaintext $P$ is encrypted with $n_1$, $n_2$, $n_3$ and intercepted by a
cryptanalyst, then the values
$P^3\mbox{ mod }n_1$, $P^3\mbox{ mod }n_1$, $P^3\mbox{ mod }n_1$
are obtained. As $n_1$, $n_2$, $n_3$ are probably relatively prime
with each other, the Chinese Remainder Theorem can then be used to obtain
$P^3\mbox{ mod }n_1n_2n_3$. But $P<n_1$, $P<n_2$, $P<n_3$, so $P^3$
has been found. Taking its cube root then gives the original plaintext $P$.

Another approach that demonstrates the ingenuity of cryptanalysis attacks
is known as a \defn{timing attack}, which relies on the time taken by an
algorithm such as RSA to decrypt ciphertext.
As the time taken by each iteration of the $\proc{Modular-Exponentiation}$
algorithm depends on whether the bit $d_i$ of the private key $d$ is $0$
or $1$, by carefully choosing ciphertext, a cryptanalyst can
determine bit by bit the value of $d$. To foil a timing attack,
RSA algorithms usually include \defn{blinding}, where a secret random number
$r$ is chosen, encrypted and multiplied by the ciphertext $C$ before it
gets decrypted. This ensures that the ciphertext actually being decrypted is not
known to the cryptanalyst. The decrypted result is then multiplied by
the multiplicative inverse of $r$ in $\mathbb{Z}_n$ to give the original
plaintext.

RSA has further properties that can be exploited by a cryptanalyst.
It is easily seen that the ciphertext for $P_1\cdot P_2$ is the same
as the product of the ciphertext for $P_1$ with the ciphertext for $P_2$.
As a consequence, if a cryptanalyst intercepts a ciphertext $C$ and
is then allowed to obtain the plaintext for certain chosen ciphertext, then
the plaintext for the intercepted ciphertext $C$ can be quickly obtained.
To counter this attack, RSA algorithms use \defn{optimal asymmetric encryption padding}
(OAEP) where all plaintext are randomly padded before they get encrypted.

Due to the length of the key needed to ensure security of RSA, and the resulting
long calculations, RSA is not nearly as efficient as symmetric cipher algorithms
such as AES for encrypting and decrypting large quantities of data.
For this reason RSA is typically used only for the initial exchange of a
secret key after which a symmetric cipher is used for further encryption and
decryption.

The class \begin{code}RSADecrypter\end{code} demonstrates how a pair
of keys can be generated in Java using the \begin{code}KeyPairGenerator\end{code}
class:
\begin{quote}\begin{code}\begin{verbatim}
KeyPairGenerator kpg = KeyPairGenerator.getInstance("RSA");
kpg.initialize(1024); // size of key in bits
KeyPair keyPair = kpg.generateKeyPair();
PublicKey publicKey = keyPair.getPublic();
PrivateKey privateKey = keyPair.getPrivate();
\end{verbatim}\end{code}\end{quote}
From the two keys their modulus $n$ and exponent $e$ or $d$ can be obtained
by typecasting them to \begin{code}RSAPublicKey\end{code} and
\begin{code}RSAPrivateKey\end{code} respectively.
Once the key pair has been generated the example sends the public key
via a TCP socket connection to another class called \begin{code}RSAEncrypter\end{code}
(either a Java or a C\# class) that is listening for socket connections.
The \begin{code}RSAEncrypter\end{code} obtains the modulus $n$ and the public
exponent $e$ and uses them to encrypt a short plaintext using RSA.
The encrypted ciphertext is then passed back to \begin{code}RSADecrypter\end{code}
where it is decrypted using the private key.
In a more-realistic application this plaintext might typically be a secret key
for symmetric encryption, and the public key would be passed using a standard format
such as the X.509 standard (discussed in Section \ref{Se:CertificatesKeyStorage}).
Furthermore, this example does not use OAEP which is supported in both Java and C\#.

RSA encryption is performed in C\# using the \begin{code}RSACryptoServiceProvider\end{code}
class (instead of the usual C\# \begin{code}CipherStream\end{code}).
Note if a key of length exactly a power of two is used then the
\begin{code}RSACryptoServiceProvider\end{code} cipher appears to add an
extra byte to the encrypted ciphertext, which is inconsistent
with the length of the key,
resulting in a \begin{code}BadPaddingException\end{code} in \begin{code}RSADecrypter\end{code}.
This small incompatibility can be avoided by not using keys of lengths
exactly 512 or 1024 in the C\# version.
\begin{figure*}\begin{program}\begin{verbatim}
/**
   A class that demonstrates how RSA encryption can be used to obtain
   RSA encrypted information from an RSAEncrypter server.
   To run first start RSAEncrypter.
   @author Andrew Ensor
*/
...
public class RSADecrypter
{
   public static final String HOST_NAME = "localhost";
   public static final int HOST_PORT = 8888; // host port number
   private KeyPair keyPair;

   public RSADecrypter()
   {  // generate an RSA public and private key pair
      try
      {  KeyPairGenerator kpg = KeyPairGenerator.getInstance("RSA");
         kpg.initialize(1024); // size of key in bits
         keyPair = kpg.generateKeyPair();
      }
      catch (NoSuchAlgorithmException e)
      {  System.err.println("Encryption algorithm not available: "+e);
      }
   }

   private void obtainBytes()
   {  // open TCP connection to RSAEncrypter server on specified port
      Socket socket = null;
      OutputStream os = null;
      InputStream is = null;
      try
      {  socket = new Socket(HOST_NAME, HOST_PORT);
         os = socket.getOutputStream();
         // send the public key modulus and public exponent to host
         RSAPublicKey publicKey = (RSAPublicKey) keyPair.getPublic();
         System.out.println("Sending public key");
         byte[] modulus = publicKey.getModulus().toByteArray();
         byte[] exponent = publicKey.getPublicExponent().toByteArray();
         os.write(modulus.length); // write number of bytes in modulus
         os.write(modulus);
         os.write(exponent.length);//write number of bytes in exponent
         os.write(exponent);
         os.flush();
         // obtain the encrypted bytes
         is = socket.getInputStream();
         ByteArrayOutputStream baos = new ByteArrayOutputStream();
         int data = is.read();
         while (data != -1) // end of stream encountered
         {  baos.write(data);
            data = is.read();
         }
\end{verbatim}\hfill \emph{cont-}\end{program}\end{figure*}%
\begin{figure*}\begin{program}\emph{-cont}\begin{verbatim}
         byte[] ciphertext = baos.toByteArray();
         // decrypt the bytes using the private key
         PrivateKey privateKey = keyPair.getPrivate();
         Cipher cipher = Cipher.getInstance("RSA");
         cipher.init(Cipher.DECRYPT_MODE, privateKey);
         byte[] plaintext = cipher.doFinal(ciphertext);
         // display the plaintext
         System.out.print("Deciphered plaintext:");
         for (int i=0; i<plaintext.length; i++)
           System.out.print(" " + plaintext[i]);
         System.out.println();
      }
      catch (IOException e)
      {  System.err.println("IOException in decrypter: " + e);
      }
      catch (NoSuchAlgorithmException e)
      {  System.err.println("Encryption algorithm not available: "+e);
      }
      catch (NoSuchPaddingException e)
      {  System.err.println("Padding scheme not available: " + e);
      }
      catch (InvalidKeyException e)
      {  System.err.println("Invalid key: " + e);
      }
      catch (IllegalBlockSizeException e)
      {  System.err.println("Cannot pad plaintext: " + e);
      }
      catch (BadPaddingException e)
      {  System.err.println("Exception with padding: " + e);
      }
      finally
      {  try
         {  if (os != null) os.close();
            if (is != null) is.close();
            if (socket != null) socket.close();
         }
         catch (IOException e)
         {  System.err.println("Failed to close streams: " + e);
         }
      }
   }

   public static void main(String[] args)
   {  RSADecrypter decrypter = new RSADecrypter();
      decrypter.obtainBytes();
   }
}
\end{verbatim}\end{program}\end{figure*}
\begin{figure*}\begin{program}\begin{verbatim}
/**
   A class that demonstrates how RSA encryption can be used to encrypt
   RSA encrypted information to an RSADecrypter client
   @see RSADecrypter.java
*/
...
public class RSAEncrypter
{
   public static final int PORT = 8888; // some unused port number
   private final byte[] plaintext = {1, 2, 3, 4, 5, 6, 7}; // secret
   ...
   // inner class that represents a single connection handler
   private class ConnectionHandler implements Runnable
   {
      private Socket socket; // socket for client/server communication

      public ConnectionHandler(Socket socket)
      {  this.socket = socket;
      }

      public void run()
      {  // handle the TCP connection with an RSADecrypter client
         InputStream is = null;
         OutputStream os = null;
         try
         {  // obtain public key modulus and public exponent from client
            is = socket.getInputStream();
            int modulusLength = is.read(); // get num bytes in modulus
            byte[] modulus = new byte[modulusLength];
            for (int i=0; i<modulusLength; i++)
               modulus[i] = (byte)is.read();
            int exponentLength = is.read();//get num bytes in exponent
            byte[] exponent = new byte[exponentLength];
            for (int i=0; i<exponentLength; i++)
               exponent[i] = (byte)is.read();
            RSAPublicKeySpec publicKeySpec = new RSAPublicKeySpec
               (new BigInteger(modulus), new BigInteger(exponent));
            // create public key from the modulus and public exponent
            KeyFactory keyFactory = KeyFactory.getInstance("RSA");
            PublicKey publicKey
               = keyFactory.generatePublic(publicKeySpec);
            // use the public key to encrypt the information
            Cipher cipher = Cipher.getInstance("RSA");
            cipher.init(Cipher.ENCRYPT_MODE, publicKey);
            byte[] ciphertext = cipher.doFinal(plaintext);
            // send the encrypted bytes
            os = socket.getOutputStream();
            os.write(ciphertext);
         }
         ...
      }
   }
}
\end{verbatim}\end{program}\end{figure*}
\begin{figure*}\begin{program}\begin{verbatim}
/*
   A C# class that demonstrates how RSA encryption can be used to
   obtain RSA encrypted information from an RSAEncrypter server.
   Note this example presumes that the modulus length can be
   represented as a byte (an int would be preferable but then the
   endian order needs to be considered), and that the modulus is a
   positive integer in twos-complement notation (so most significant
   bit is zero).
*/
...
public class RSADecrypter
{
   public const String HOST_NAME = "localhost";
   public const int HOST_PORT = 8888; // host port number
   private RSAParameters keyPair;

   public RSADecrypter()
   {  // generate an RSA public and private key pair using 1024 bit key
      RSACryptoServiceProvider cipher
         = new RSACryptoServiceProvider(1024);
      keyPair = cipher.ExportParameters(true);
   }

   private void ObtainBytes()
   {  // open TCP connection to RSAEncrypter server on specified port
      TcpClient client = null;
      try
      {  client = new TcpClient();
         client.Connect(HOST_NAME, HOST_PORT);
      }
      catch (SocketException e)
      {  Console.WriteLine("Client could not make connection: " + e);
         System.Environment.Exit(System.Environment.ExitCode);
      }
      NetworkStream stream = null;
      BinaryWriter bw = null;
      BinaryReader br = null;
      try
      {  stream = client.GetStream();
         bw = new BinaryWriter(stream);
         // send the public key modulus and public exponent to host
         Console.WriteLine("Sending public key");
         byte[] modulus = keyPair.Modulus;
         byte[] exponent = keyPair.Exponent;
         if ((modulus[0] & 0x80) != 0)
         {  // modulus needs a 00 prepended to put into twos complement
            byte[] twosComplement = new byte[modulus.Length+1];
            twosComplement[0] = 0;
            Array.Copy(modulus, 0, twosComplement, 1, modulus.Length);
            modulus = twosComplement;
         }
\end{verbatim}\hfill \emph{cont-}\end{program}\end{figure*}%
\begin{figure*}\begin{program}\emph{-cont}\begin{verbatim}
         bw.Write((byte)modulus.Length); // write num of modulus bytes
         bw.Write(modulus);
         bw.Write((byte)exponent.Length);//write num of exponent bytes
         bw.Write(exponent);
         bw.Flush();
         // obtain the encrypted bytes
         br = new BinaryReader(stream);
         MemoryStream memoryStream = new MemoryStream();
         BinaryWriter msbw = new BinaryWriter(memoryStream);
         byte[] data = br.ReadBytes(0);
         while (data.Length>0) // end of stream encountered
         {  msbw.Write(data);
            data = br.ReadBytes(0);
         }
         stream.CopyTo(memoryStream);
         byte[] ciphertext = memoryStream.ToArray();
         // decrypt the bytes using private key without OAEP padding
         RSACryptoServiceProvider cipher
            = new RSACryptoServiceProvider();
         cipher.ImportParameters(keyPair);
         byte[] plaintext = cipher.Decrypt(ciphertext, true);
         // display the plaintext
         Console.Write("Deciphered plaintext:");
         for (int i=0; i<plaintext.Length; i++)
            Console.Write(" " + plaintext[i]);
         Console.WriteLine();
         cipher.Dispose();

      }
      catch (SocketException e)
      {  Console.WriteLine("Client error: " + e);
      }
      finally
      {  try
         {  if (br != null) br.Close();
            if (bw != null) bw.Close();
            if (stream != null) stream.Close();
            if (client != null) client.Close();
         }
         catch (SocketException e)
         {  Console.WriteLine("Failed to close streams: " + e);
         }
      }
   }

   public static void Main(string[] args)
   {  RSADecrypter decrypter = new RSADecrypter();
      decrypter.ObtainBytes();
   }
}
\end{verbatim}\end{program}\end{figure*}
\begin{figure*}\begin{program}\begin{verbatim}
/*
   A C# class that demonstrates how RSA encryption can be used in .NET
   to encrypt RSA encrypted information to an RSADecrypter client.
*/
...
public class RSAEncrypter
{
   ...
   // inner class that represents a single connection handler
   private class ConnectionHandler
   {
      ...
      public void Run() // can call this method anything in C#
      {  // handle the TCP connection with an RSADecrypter client
         NetworkStream stream = new NetworkStream(socket);
         BinaryReader br = null;
         BinaryWriter bw = null;
         try
         {  // obtain public key modulus and public expon from client
            br = new BinaryReader(stream);
            int modulusLength=br.ReadByte();//get num bytes in modulus
            byte[] modulus = new byte[modulusLength];
            for (int i=0; i<modulusLength; i++)
               modulus[i] = br.ReadByte();
            if (modulus[0] == 0)
            {  // an additional 00 byte has probably been appended
               // to make modulus twos complement
               byte[] unsigned = new byte[modulus.Length-1];
               Array.Copy(modulus, 1, unsigned, 0, modulus.Length-1);
               modulus = unsigned;
            }
            int exponentLength=br.ReadByte();//get num bytes in expon
            byte[] exponent = new byte[exponentLength];
            for (int i=0; i<exponentLength; i++)
               exponent[i] = br.ReadByte();
            RSAParameters publicKeySpec = new RSAParameters();
            publicKeySpec.Modulus = modulus;
            publicKeySpec.Exponent = exponent;
            // create public key from the modulus and public exponent
            RSACryptoServiceProvider cipher
               = new RSACryptoServiceProvider();
            cipher.ImportParameters(publicKeySpec);
            // use the public key to encrypt info without OAEP padding
            byte[] ciphertext = cipher.Encrypt(plaintext, true);
            // send the encrypted bytes
            bw = new BinaryWriter(stream);
            bw.Write(ciphertext);
            cipher.Dispose();
         }
         ...
      }
   }
}
\end{verbatim}\end{program}\end{figure*}


\begin{exercise}[Implementation of RSA]
Prepare a program that provides an implementation of the RSA public-key block cipher.
\end{exercise}


%%%%%%%%%%%%%%%%%%%%%%%%%%%%%%%%%%%%%%%%%%%%%%%%%%%%%%%%%%%%%%%%%%%%%%%%%%%%%%%%%%%%%%%%%
%%%%% Key Exchange                                                                  %%%%%
%%%%%%%%%%%%%%%%%%%%%%%%%%%%%%%%%%%%%%%%%%%%%%%%%%%%%%%%%%%%%%%%%%%%%%%%%%%%%%%%%%%%%%%%%

\section{Key Exchange}

\reading{pp290-301}

The performance of public-key encryption algorithms
can be up to 1000 times slower than symmetric encryption,
which limits their suitability to the exchange of
brief messages such as secret keys for symmetric encryption.
The classes \begin{code}RSADecrypter\end{code} and
\begin{code}RSAEncrypter\end{code} from Section \ref{Se:RSAAlgorithm}
demonstrate how an asymmetric cipher such as RSA can be used to pass a
secret key safely encrypted between two processes, which can then be
used for the encryption of further messages.

The \defn{Diffie-Hellman key exchange} algorithm was the first published
public-key algorithm and was devised so that two processes could
use insecure communication to determine a common secret key.
This algorithm starts with two public values that are pre-computed,
a large prime $p$ and an integer value $a$ with $1<a<p$ for which:
\begin{displaymath}
  a, a^2\mbox{ mod }p, a^3\mbox{ mod }p, \dots, a^{p-1}\mbox{ mod }p=1.
\end{displaymath}
are all distinct (so they must just be a permutation of $\{1,2,3,\dots, p-1\}$)
in which case $a$ is called a \defn{primitive root} of $p$.
Once a primitive root $a$ of a prime $p$ has been found,
for any integer $b$ there is a unique power $i<p$
for which $b\mbox{ mod }p=a^i\mbox{ mod }p$, and $i$ is
called the \defn{discrete logarithm} or \defn{index} of $b$ modulo $p$ for
the base $a$.

For example, taking $p=7$ the value $a=2$ is not a primitive root of $7$ since in
$\mathbb{Z}_7$ one has:
\begin{displaymath}
  2^1=2, 2^2=4, 2^3=1, 2^4=2, 2^5=4, 2^6=1,
\end{displaymath}
which are not all distinct, whereas the value $a=3$ is a primitive root as:
\begin{displaymath}
  3^1=3, 3^2=2, 3^3=6, 3^4=4, 3^5=5, 3^6=1,
\end{displaymath}
which are all distinct.
Thus the discrete logarithm of an integer such as $b=1763$ modulo $p=7$ for the
base $a=3$ is found by $b\mbox{ mod }7=6=3^3$, and so $i=3$.

The security of the Diffie-Hellman key exchange relies on the
difficulty of computing discrete logarithms for large primes $p$,
much as RSA relies on the difficulty of computing prime
factorizations for large integers $n$. When two processes with the
same base $a$ and prime $p$ want to exchange a secret key each
selects its own secure random number $x<p$ as its private key and
calculates a public key $y=a^x\mbox{ mod }p$, which is
communicated to the other process. When each process receives the
public key $y^\prime=a^{x^\prime}\mbox{ mod }p$ of the other
process (but not knowing the other private key $x^\prime$), it
calculates the secret key $K$:
\begin{displaymath}
  K = \left(y^\prime\right)^x\mbox{ mod }p.
\end{displaymath}
Note that this is the same key as that calculated by the other process as:
\begin{eqnarray*}
  \left(y^\prime\right)^x\mbox{ mod }p
  &=& \left(a^{x^\prime}\mbox{ mod }p\right)^x\mbox{ mod }p \\
  &=& a^{x\cdot x^\prime}\mbox{ mod }p \\
  &=& \left(a^x\mbox{ mod }p\right)^{x^\prime}\mbox{ mod }p \\
  &=& y^{x^\prime}\mbox{ mod }p,
\end{eqnarray*}
but a cryptanalyst cannot calculate $K$ without knowing one of the values $x$ or $x^\prime$.
Obtaining these values would require calculating the discrete logarithm of
an intercepted $y$ or $y^\prime$ value modulo $p$ to the base $a$, which is
computationally difficult.

JCA supports Diffie-Hellman key exchange using the algorithm name \begin{code}DH\end{code}
and the following steps:
\begin{description}
  \item[Generate the Public Prime and Primitive Root]
  New values of $p$ and $a$ can be generated for a Diffie-Hellman key exchange
  using an \begin{code}Algorithm\-ParameterGenerator\end{code}:
\begin{quote}\begin{code}\begin{verbatim}
AlgorithmParameterGenerator apg
   = AlgorithmParameterGenerator.getInstance("DH");
apg.init(512); // 512-bit prime
DHParameterSpec dhSpec = (DHParameterSpec)
   apg.generateParameters().getParameterSpec
   (DHParameterSpec.class);
\end{verbatim}\end{code}\end{quote}
  or else existing values of $p$ and $a$ can be passed directly:
\begin{quote}\begin{code}\begin{verbatim}
DHParameterSpec dhSpec = new DHParameterSpec(p, a);
\end{verbatim}\end{code}\end{quote}

  \item[Create Public and Private Keys]
  Values of $x$ and $y$ can be generated by each process
  by passing a \begin{code}KeyPairGenerator\end{code}
  the $p$ and $a$ parameters:
\begin{quote}\begin{code}\begin{verbatim}
KeyPairGenerator kpg=KeyPairGenerator.getInstance("DH");
kpg.initialize(dhSpec);
KeyPair keyPair = kpg.generateKeyPair();
PublicKey publicKey = keyPair.getPublic();
PrivateKey privateKey = keyPair.getPrivate();
\end{verbatim}\end{code}\end{quote}
  From the two keys their public value $y$ or private value $x$ can be obtained
  by typecasting the keys to \begin{code}DHPublicKey\end{code} or
  \begin{code}DHPrivateKey\end{code} respectively.

  \item[Swap Public Keys]
  The processes involved in the key exchange swap their public values of $y$.
  This can be accomplished by passing the raw values of $y$
  (along with $p$ and $a$ if not already publicly available):
\begin{quote}\begin{code}\begin{verbatim}
BigInteger yValue = ((DHPublicKey)publicKey).getY();
... // send yValue
\end{verbatim}\end{code}\end{quote}
  and using a \begin{code}DHPublicKeySpec\end{code} with a \begin{code}KeyFactory\end{code}
  to reassemble the public key at the other end:
\begin{quote}\begin{code}\begin{verbatim}
BigInteger otherYValue = ...; // receive
DHPublicKeySpec otherPublicKeySpec
   = new DHPublicKeySpec(otherYValue, pValue, aValue);
KeyFactory keyFactory = KeyFactory.getInstance("DH");
PublicKey otherPublicKey
   = keyFactory.generatePublic(otherPublicKeySpec);
\end{verbatim}\end{code}\end{quote}
  A preferred way to accomplish this is to send the public key in the
  standardized X.509 key exchange format:
\begin{quote}\begin{code}\begin{verbatim}
byte[] publicKeyBytes = publicKey.getEncoded();
... // send publicKeyBytes
\end{verbatim}\end{code}\end{quote}
  and use an \begin{code}X509EncodedKeySpec\end{code} with a \begin{code}KeyFactory\end{code}
  to reassemble the public key at the other end:
\begin{quote}\begin{code}\begin{verbatim}
byte[] otherPublicKeyBytes = ...; // receive
X509EncodedKeySpec otherPublicKeySpec
   = new X509EncodedKeySpec(otherPublicKeyBytes);
KeyFactory keyFactory = KeyFactory.getInstance("DH");
PublicKey otherPublicKey
   = keyFactory.generatePublic(otherPublicKeySpec);
\end{verbatim}\end{code}\end{quote}

  \item[Generate Secret Key]
  The private key is used to initialize a \begin{code}KeyAgreement\end{code}
  and then its \begin{code}doPhase\end{code} method is passed the swapped public key(s)
  (with a boolean to indicate which is the last public key obtained):
\begin{quote}\begin{code}\begin{verbatim}
KeyAgreement ka = KeyAgreement.getInstance("DH");
ka.init(privateKey); // initialize with own private key
ka.doPhase(otherPublicKey, true);
\end{verbatim}\end{code}\end{quote}
  Then the \begin{code}KeyAgreement\end{code} method \begin{code}generateSecret\end{code}
  can be used to generate bytes for a secret key (or for a specific
  algorithm such as \begin{code}DES\end{code} or \begin{code}DESede\end{code}):
\begin{quote}\begin{code}\begin{verbatim}
byte[] key = ka.generateSecret();
\end{verbatim}\end{code}\end{quote}

\end{description}

However, if the communication channel between two processes
can be compromised so that messages can be modified then
the Diffie-Hellman key exchange is susceptible to a \defn{man-in-the-middle attack}.
With this attack an adversary generates its own key pair and
intercepts all messages between the two processes.
The attacker starts by replacing the public key
of each with its own public key during the key exchange.
Then each process will generate a secret key in common
with the attacker rather than with each other.
The attacker then decrypts messages from either process using one secret
key and can re-encrypt them (possibly modified) for the other process using
the other secret key.
If both ends are unaware that their messages have been modified then their
encrypted communication will be compromised.

To counter a man-in-the-middle attack it is important that the recipient
of a public key be certain that it is an authentic key for the intended
process. Merely broadcasting a public key is not secure as anyone could
forge a broadcast, sending a public key generated by themselves and
claiming to be another process. The forger would then receive messages
encrypted for it that were intended for the genuine process until the
forgery were detected and all other processes notified.
Instead, the public keys could be made globally available from a
secure public directory, which would require a man-in-the-middle attack to
intercept messages to and from the directory (which might be able to be made
physically secure) as well as eavesdrop on communication between
processes.

A more secure approach is to use a \defn{public-key authority} which
controls access to the public keys in a system.
A central authority holds a public key for each process, which is
delivered to the authority via some secure means (such as physically),
and in exchange the process is given a public key for its communication
with the authority.
When an initiating process wants to exchange its public key with that
of another process it can use the following steps:
\begin{enumerate}
	\begin{multicols}{2}
  \item
     The initiator sends a request encrypted using the public key of the
    authority, requesting the public key for the receiving process, and
    including a timestamp with the request.
    The public-key authority responds with a message encrypted using its private key,
    including the public key for the receiver and the original request with the timestamp.
    \columnbreak Once the response has been decrypted by the initiator using the public key of the authority
    it is assured that the message was encrypted by the authority and that
    its original request was received by the authority untampered.
 \begin{center}
 	 \begin{pspicture}[shift=-1.5](0,0)(5,1.5)
    \rput(0.8,0.25){\ovalnode{I}{\makebox[10mm]{\scriptsize Initiator}}}
    \rput(2.5,1.25){\ovalnode{K}{\makebox[10mm]{\scriptsize Authority}}}
    \rput(4.2,0.25){\ovalnode{R}{\makebox[10mm]{\scriptsize Receiver}}}
    \nccurve[arrows=->,linecolor=red,linewidth=1mm,angleA=75,angleB=195]{I}{K}
    \nccurve[arrows=->,linecolor=red,linewidth=1mm,angleA=255,angleB=15]{K}{I}
    \rput(0.75,1.2){\tiny\begin{tabular}{c}public key\\encrypted\\request\end{tabular}}
    \rput(2.75,0.3){\tiny\begin{tabular}{c}private key\\encrypted\\reply\end{tabular}}
  \end{pspicture}
\end{center}
\end{multicols}
\begin{multicols}{2}
  \item
    The initiator then uses the public key for the receiver to encrypt an identifier
    for itself and a nonce, which it sends to the receiver, requesting an exchange.
    It is assured that only the receiver can decrypt the request to obtain
    its nonce.
  \begin{pspicture}[shift=-1.5](0,0)(5,1.5)
    \rput(0.8,0.25){\ovalnode{I}{\makebox[10mm]{\scriptsize Initiator}}}
    \rput(2.5,1.25){\ovalnode{K}{\makebox[10mm]{\scriptsize Authority}}}
    \rput(4.2,0.25){\ovalnode{R}{\makebox[10mm]{\scriptsize Receiver}}}
    \ncline[arrows=->,linecolor=red,linewidth=1mm]{I}{R}
    \rput(2.5,0.55){\tiny\begin{tabular}{c}receiver's public key\\encrypted request\end{tabular}}
  \end{pspicture}
\end{multicols}
\begin{multicols}{2} 
	 \item
    Upon receiving the request for an exchange, the receiver repeats the same process
    with the public-key authority to obtain a public key for the initiator. At this stage the initiator and the receiver have securely obtained
    each other's public key. To assure both ends that they are communicating with
    each other two further steps are included.
 \begin{pspicture}[shift=-1.5](0,0)(5,1.5)
    \rput(0.8,0.25){\ovalnode{I}{\makebox[10mm]{\scriptsize Initiator}}}
    \rput(2.5,1.25){\ovalnode{K}{\makebox[10mm]{\scriptsize Authority}}}
    \rput(4.2,0.25){\ovalnode{R}{\makebox[10mm]{\scriptsize Receiver}}}
    \nccurve[arrows=->,linecolor=red,linewidth=1mm,angleA=105,angleB=345]{R}{K}
    \nccurve[arrows=->,linecolor=red,linewidth=1mm,angleA=285,angleB=165]{K}{R}
    \rput(4.25,1.2){\tiny\begin{tabular}{c}public key\\encrypted\\request\end{tabular}}
    \rput(2.25,0.3){\tiny\begin{tabular}{c}private key\\encrypted\\reply\end{tabular}}
  \end{pspicture}
\end{multicols}
\begin{multicols}{2} 
  \item
    The receiver sends a response to the initiator encrypted using the initiator's
    public key, including the initiator's nonce as well as its own nonce.
    When the initiator decrypts the response and finds its own nonce it is 
    assured that it is actually communicating with the intended receiver.

  \begin{pspicture}[shift=-1.5](0,0)(5,1.5)
    \rput(0.8,0.25){\ovalnode{I}{\makebox[10mm]{\scriptsize Initiator}}}
    \rput(2.5,1.25){\ovalnode{K}{\makebox[10mm]{\scriptsize Authority}}}
    \rput(4.2,0.25){\ovalnode{R}{\makebox[10mm]{\scriptsize Receiver}}}
    \ncline[arrows=->,linecolor=red,linewidth=1mm]{R}{I}
    \rput(2.5,0.55){\tiny\begin{tabular}{c}initiator's public key\\encrypted response\end{tabular}}
  \end{pspicture}

\end{multicols}
\begin{multicols}{2} 
  \item
    The initiator acknowledges the response from the receiver by sending an
    acknowledgement message encrypted using the public key of the receiver,
    including the receiver's nonce.
When the receiver decrypts the acknowledgement and finds its own nonce it is assured that it is actually
communicating with the intended initiator.
 
  \begin{pspicture}[shift=-1.5](0,0)(5,1.5)
    \rput(0.8,0.25){\ovalnode{I}{\makebox[10mm]{\scriptsize Initiator}}}
    \rput(2.5,1.25){\ovalnode{K}{\makebox[10mm]{\scriptsize Authority}}}
    \rput(4.2,0.25){\ovalnode{R}{\makebox[10mm]{\scriptsize Receiver}}}
    \ncline[arrows=->,linecolor=red,linewidth=1mm]{I}{R}
    \rput(2.5,0.55){\tiny\begin{tabular}{c}receivers's public key\\encrypted acknow\end{tabular}}
  \end{pspicture}
\end{multicols}
 
\end{enumerate}

One drawback of using a central public-key authority is that it can become
a bottleneck in a large system, hindering scalability. Also, if the
authority were to temporarily fail then no keys would be available.
An alternative approach is to use encrypted certificates.
A \defn{certificate} holds an identifier for a process and a public key offered
by it, prepared by a trusted \defn{certificate authority}
and encrypted by the authority using a private key.
The certificate authority makes its corresponding public key available
so that any process can decrypt the certificate and obtain the public key
of the process, but since its private key is confidential no one else
can forge a certificate by that authority.
So long as each process trusts the certificate authority and is certain
that it has the correct public key for the authority, then two processes
can exchange public keys simply by sending each other a certificate
for itself issued by the authority.
Thus certificates can be forwarded around the system even when the authority
might be temporarily unavailable.
A timestamp is usually included as part of the encrypted certificate so that
obsolete or expired certificates can be ignored.
Certificates are discussed further in Section \ref{Se:CertificatesKeyStorage}.

Once two processes have successfully exchanged public keys
(such as via a public-key authority or by exchanging certificates)
the following steps can be followed to protect against attacks
before a secret key is exchanged:
\begin{enumerate}
  \item First, the initiator and receiver assure that they are communicating
  with each other by exchanging nonce values. The initiator starts by sending
  a request encrypted using the public key of the receiver,
  including an identifier for itself and a nonce.

  \item The receiver replies with a response encrypted using the public key
  of the initiator, including the nonce of the initiator as well as
  its own nonce. This assures the initiator that it is communicating with
  the receiver.

  \item The initiator then sends an acknowledgement encrypted using the
  public key of the receiver, including the nonce of the receiver.
  This assures the receiver that it is communicating with the initiator.

  \item One end then generates a secret key. Rather than just encrypting it once
  using the public key of the other process (so that only that process can decrypt
  it), the encrypted key is also further encrypted using the private key of the
  process that generated it. This assures the other end that this process
  was the process that created the key, and so is not susceptible to a
  man-in-the-middle attack.
\end{enumerate}


\begin{exercise}[Diffie-Hellman Key Exchange]
Prepare a program that uses the Diffie-Hellman key exchange algorithm over a
TCP connection to exchange a secret key.
\end{exercise}


%%%%%%%%%%%%%%%%%%%%%%%%%%%%%%%%%%%%%%%%%%%%%%%%%%%%%%%%%%%%%%%%%%%%%%%%%%%%%%%%%%%%%%%%%
%%%%% Elliptic Curve Cryptography                                                   %%%%%
%%%%%%%%%%%%%%%%%%%%%%%%%%%%%%%%%%%%%%%%%%%%%%%%%%%%%%%%%%%%%%%%%%%%%%%%%%%%%%%%%%%%%%%%%

\section{Elliptic Curve Cryptography}\label{Se:EllipticCurveCryptography}

\reading{pp301-313}

Due to recent progress in calculating prime factorizations RSA public keys
need to be at least 1024 bits to provide adequate security.
\defn{Elliptic Curve Cryptography} (ECC) is a promising alternative
to RSA for public-key encryption, allowing a much shorter key to be used
with far less computational overhead, yet providing the same level of security
as RSA against a cryptanalysis attack.

\noindent
\begin{minipage}{95mm}
  For instance, in order to provide roughly the same level of security as a
  128-bit AES key, RSA requires a 3072-bit key, which places quite a computational
  burden on any devices using RSA. Worse still, to be equivalent to
  a 256-bit AES key, RSA requires a 15360-bit key, which is infeasible on limited
  devices such as mobile phones. ECC however requires just double the number of bits
  than an AES key, making it particularly attractive for public-key encryption on limited devices. It has begun to challenge
  RSA and Diffie-Hellman key exchange as the
\end{minipage}
\hfill
\begin{tabular}{|r|r|r|} \hline
  \multicolumn{3}{|l|}{\textit{Comparable Key Sizes}} \\ \hline\hline
  \textit{AES} & \textit{RSA} & \textit{ECC} \\ \hline
  --- & 512 & 112 \\
  --- & 1024 & 160 \\
  --- & 2048 & 224 \\
  128 & 3072 & 256 \\
  192 & 7680 & 384 \\
  256 & 15360 & 512 \\\hline
\end{tabular}
preferred public-key cryptographic algorithm, since the difficulty of cryptanalysis against ECC gets harder for longer keys much faster than for either RSA or Diffie-Hellman.

An \defn{elliptic curve} over a field consists of the set of points $(x,y)$
where $x$ and $y$ are elements of the field that satisfy an equation of the form:
\begin{displaymath}
  y^2 + a_1xy+a_3y = x^3+a_2x^2+a_4x+a_6,
\end{displaymath}
where the coefficients $a_1, a_3, a_2, a_4, a_6$ are also elements of the field.

If the number of elements in the field is not divisible by $2$ nor $3$
(such as the field $\mathbb{R}$ of all real numbers
or the finite field $\mathbb{Z}_p$ for a prime $p>3$)
then using a suitable affine transformation results in the simpler equation:
\begin{displaymath}
  y^2 = x^3+ax+b.
\end{displaymath}
If instead the number of elements is divisible by $2$ but not by $3$
(such as the field $GF(2^m)$) then the equation is more
conveniently expressed in the form $y^2+xy=x^3+ax^2+b$.

The \defn{discriminant} $\Delta=-16\left(4a^3+27b^2\right)$ for an elliptic curve
$y^2 = x^3+ax+b$
is a quantity analogous to $\Delta=b^2-4ac$ for a quadratic curve $y=ax^2+bx+c$,
that determines the number of values of $x$ for which $y=0$.
If $\Delta\neq0$ then there are three distinct (possibly complex) solutions.
For example, the elliptic curves $y^2=x^3-x$, $y^2=x^3-x+1$, $y^2=x^3+x$
all have non-zero discriminant, and so $x^3+ax+b=0$ has three distinct solutions
for $x$ (the first curve has roots $x=-1, 0, 1$, the second has roots
$x\approx-1.3247, 0.6624+0.5623i, 0.6624-0.5623i$, and the third has
$x=0, i, -i$).

\noindent
\begin{pspicture}*(-1.5,-3.2)(2.5,2.5)
  % draw the x and y axes
  \psaxes[arrows=->,linecolor=red](0,0)(-1.5,-2.5)(2.5,2.5)
  \rput(2.3,0.2){$x$}\rput(0.15,2.2){$y$}
  \psset{linecolor=blue,linewidth=0.5mm}
  \psplot{-1}{0}{x 3 exp x sub sqrt}
  \psplot{-1}{0}{x 3 exp x sub sqrt -1 mul}
  \psplot{1}{2}{x 3 exp x sub sqrt}
  \psplot{1}{2}{x 3 exp x sub sqrt -1 mul}
  \rput(0.5,-2.7){\small$y^2=x^3-x$}
\end{pspicture}
\hfill
\begin{pspicture}*(-1.5,-3.2)(2.5,2.5)
  % draw the x and y axes
  \psaxes[arrows=->,linecolor=red](0,0)(-1.5,-2.5)(2.5,2.5)
  \rput(2.3,0.2){$x$}\rput(0.15,2.2){$y$}
  \psset{linecolor=blue,linewidth=0.5mm}
  \psplot{-1.3247}{2}{x 3 exp x sub 1 add sqrt}
  \psplot{-1.3247}{2}{x 3 exp x sub 1 add sqrt -1 mul}
  \rput(0.5,-2.7){\small$y^2=x^3-x+1$}
\end{pspicture}
\hfill
\begin{pspicture}*(-0.5,-3.2)(2.5,2.5)
  % draw the x and y axes
  \psaxes[arrows=->,linecolor=red](0,0)(-0.5,-2.5)(2.5,2.5)
  \rput(2.3,0.2){$x$}\rput(0.15,2.2){$y$}
  \psset{linecolor=blue,linewidth=0.5mm}
  \psplot{0}{2}{x 3 exp x add sqrt}
  \psplot{0}{1.7}{x 3 exp x add sqrt -1 mul}
  \rput(0.5,-2.7){\small$y^2=x^3+x$}
\end{pspicture}

So long as the discriminant is not zero it can be shown that any elliptic curve
given by $y^2=x^3+ax+b$ can be used to define an abelian group $E(a,b)$
consisting of all points $(x,y)$ on the curve together with another element denoted by $O$.
In this group the binary operation $+$ is defined for two points $P$ and $Q$
depending on four possible cases:
\begin{enumerate}
  \item If $P=O$ or $Q=O$ then $P+Q$ is just taken to be $P$ (if $Q=O$)
  or else $Q$ (if $P=O$), so that $O$ acts as the identity for the group.

 
   \item If $P=(x, y)$ and $Q=(x, -y)$, so that $Q$ is the mirror image of $P$ in the $x$-axis
    then $P+Q$ is taken to be $O$, so that $Q=-P$ is the inverse of $P$ in the group.
    Algebraically:
    \begin{displaymath}
      -\left(x,y\right) = \left(x, -y\right).
    \end{displaymath}

  \begin{pspicture}*[shift=2.5](-1,-1.25)(1.25,1.25)
    \psset{unit=0.5}
    % draw the x and y axes
    \psaxes[arrows=->,linecolor=red,labels=none,ticks=none](0,0)(-2,-2.5)(2.5,2.5)
    \psset{linecolor=blue,linewidth=0.5mm}
    \psplot{-1.3247}{2}{x 3 exp x sub 1 add sqrt}
    \psplot{-1.3247}{2}{x 3 exp x sub 1 add sqrt -1 mul}
    \psset{linecolor=black}
    \dotnode(1.5,1.696){P}\rput(1.9,1.5){\footnotesize$P$}
    \dotnode(1.5,-1.696){MP}\rput(2,-1.5){\footnotesize$-\!P$}
    \psline[linecolor=red,linestyle=dashed,linewidth=0.2mm](1.5,1.696)(1.5,-1.696)
  \end{pspicture}
  \item

    If $P=(x, y)$ and $Q=(x^\prime, y^\prime)$ with $x\neq x^\prime$ then $P+Q$ is taken
    to be $-R$ where $R$ is the point where the line through $P$ and $Q$ meets the
    elliptic curve a third time.
    Algebraically, the slope of the line through $P$ and $Q$ is given by
    $m=\frac{y^\prime-y}{x^\prime-x}$
    and the following equation can be derived:
    \begin{displaymath}
      \left(x,y\right)+\left(x^\prime,y^\prime\right)
      = \left(m^2\!-\!x\!-\!x^\prime, 2mx\!+\!mx^\prime\!-\!m^3\!-\!y\right).
    \end{displaymath}


  \begin{pspicture}*[shift=-2.5](-1,-1.25)(1.25,1.25)
    \psset{unit=0.5}
    % draw the x and y axes
    \psaxes[arrows=->,linecolor=red,labels=none,ticks=none](0,0)(-2,-2.5)(2.5,2.5)
    \psset{linecolor=blue,linewidth=0.5mm}
    \psplot{-1.3247}{2}{x 3 exp x sub 1 add sqrt}
    \psplot{-1.3247}{2}{x 3 exp x sub 1 add sqrt -1 mul}
    \psset{linecolor=black}
    \dotnode(1.5,1.696){P}\rput(1.9,1.5){\footnotesize$P$}
    \dotnode(0.588,0.785){Q}\rput(0.9,0.4){\footnotesize$Q$}
    \psline[linecolor=red,linewidth=0.2mm](-2,-1.804)(2.304,2.5)
    \psline[linecolor=red,linestyle=dashed,linewidth=0.2mm](-1.089,-0.893)(-1.089,0.893)
    \dotnode(-1.089,-0.893){R}\rput(-1.1,-1.4){\footnotesize$R$}
    \dotnode(-1.089,0.893){PQ}\rput(-1.45,1.4){\footnotesize$P\!\!+\!\!Q$}
  \end{pspicture}

  \item

    If $P=Q$ then $P+Q=2 P$ is taken to be $-R$ where $R$ is the point
    where the tangent line through $P$ meets the
    elliptic curve again.
    Algebraically, the slope of the tangent is given by $m=\frac{3x^2+a}{2y}$
    and the following equation can be derived:
    \begin{displaymath}
      2\left(x,y\right) = \left(m^2-2x, 3mx-m^3-y\right).
    \end{displaymath}
 
 
  \begin{pspicture}*[shift=-2.5](-1,-1.25)(1.25,1.25)
    \psset{unit=0.5}
    % draw the x and y axes
    \psaxes[arrows=->,linecolor=red,labels=none,ticks=none](0,0)(-2,-2.5)(2.5,2.5)
    \psset{linecolor=blue,linewidth=0.5mm}
    \psplot{-1.3247}{2}{x 3 exp x sub 1 add sqrt}
    \psplot{-1.3247}{2}{x 3 exp x sub 1 add sqrt -1 mul}
    \psset{linecolor=black}
    \dotnode(1.5,1.696){P}\rput(1.9,1.5){\footnotesize$P$}
    \psline[linecolor=red,linewidth=0.2mm](-1.12,-2.5)(2,2.5)
    \psline[linecolor=red,linestyle=dashed,linewidth=0.2mm](-0.256,-1.113)(-0.256,1.113)
    \dotnode(-0.256,-1.113){R}\rput(-0.25,-1.6){\footnotesize$R$}
    \dotnode(-0.256,1.113){PP}\rput(-0.45,1.6){\footnotesize$2P$}
  \end{pspicture}
\end{enumerate}

In Elliptic Curve Cryptography a finite field is used instead of $\mathbb{R}$.
Typically if ECC is implemented in software then the field
$\mathbb{Z}_p$ where $p$ is a prime is chosen and an elliptic curve
$y^2=x^3+ax+b$ with $4a^3+27b^2\neq0$ is used (called a \defn{prime curve}),
and addition calculations are performed using the previous algebraic equations:
\begin{eqnarray*}
  \left(x,y\right)+\left(x^\prime,y^\prime\right)
    &=& \left(m^2\!-\!x\!-\!x^\prime, 2mx\!+\!mx^\prime\!-\!m^3\!-\!y\right) \\
  2\left(x,y\right) &=& \left(m^2-2x, 3mx-m^3-y\right).
\end{eqnarray*}
The number of elements in the resulting abelian group can be shown to be
somewhere between $p+1-2\sqrt{p}$ and $p+1+2\sqrt{p}$.
If instead ECC is implemented in hardware (where rapid bitwise operations are
advantageous) then the field $GF(2^m)$ where $m\geq1$ is chosen and
an elliptic curve $y^2+xy=x^3+ax^2+b$ with $b\neq0$ is used
(called a \defn{binary curve}).

\clearpage
\begin{multicols}{2}
As an example, the elliptic curve $y^2=x^3-x+1$ over the field $\mathbb{Z}_7$
has $11$ solutions, the points $(0,1)$, $(0,6)$, $(1,1)$, $(1,6)$,
$(2,0)$, $(3,2)$, $(3,5)$, $(5,3)$, $(5,4)$, $(6,1)$, $(6,6)$.
Including the identity $O$ results in a group $E(-1,1)$ with $12$ elements.
Note that for any point $(x, y)$ in the abelian group
its inverse is given by
$-(x,y)=(x, -y)=(x, 7-y)$ resulting in symmetry in the
illustrated grid. So for example $-(5,3)=(5,4)$ and $-(2,0)=(2,0)$.\columnbreak

\begin{pspicture}[shift=1.75](-0.4,-0.7)(3.1,2.8)
  \psset{unit=0.5}
  \multirput*(0,0)(1,0){7}{\psline[linestyle=dotted,linewidth=0.3mm](0,0)(0,6)}
  \rput(0,-0.4){\footnotesize$0$}\rput(1,-0.4){\footnotesize$1$}
  \rput(2,-0.4){\footnotesize$2$}\rput(3,-0.4){\footnotesize$3$}
  \rput(4,-0.4){\footnotesize$4$}\rput(5,-0.4){\footnotesize$5$}
  \rput(6,-0.4){\footnotesize$6$}
  \multirput*(0,0)(0,1){7}{\psline[linestyle=dotted,linewidth=0.3mm](0,0)(6,0)}
  \rput(-0.4,0){\footnotesize$0$}\rput(-0.4,1){\footnotesize$1$}
  \rput(-0.4,2){\footnotesize$2$}\rput(-0.4,3){\footnotesize$3$}
  \rput(-0.4,4){\footnotesize$4$}\rput(-0.4,5){\footnotesize$5$}
  \rput(-0.4,6){\footnotesize$6$}
  \psset{dotsize=1.5mm}
  \dotnode(0,1){e1}\dotnode(0,6){e2}\dotnode(1,1){e3}\dotnode(1,6){e4}
  \dotnode(2,0){e5}\dotnode(3,2){e6}\dotnode(3,5){e7}\dotnode(5,3){e8}
  \dotnode(5,4){e9}\dotnode(6,1){e10}\dotnode(6,6){e11}
\end{pspicture}
\end{multicols}
\noindent
Taking $P=(5,3)$ the equation for $2P$ gives that
$m=\frac{3\cdot5^2-1}{2\cdot3}=4\cdot6^{-1}=3$
(using the fact that $6^{-1}=6$ in the field $\mathbb{Z}_7$) and so:
\begin{displaymath}
  2P = \left(3^2-2\cdot5,3\cdot3\cdot5-3^3-3\right) = (6,1).
\end{displaymath}
Larger multiples $nP$ of a point $P$ can be found by repeated doubling of the previous value,
obtaining $P, 2P, 4P, 8P, 16P, \dots$ in $\Theta(\log_2 n)$. For some particular
curves and values of $p$ there are more efficient algorithms
(such as the NIST P192 curve defined using the prime $p=2^{192}-2^{64}+1$
where $nP$ can be found using up to 38 addition and 192 doubling operations,
quite feasible for a limited device).
However, for a cryptanalyst to determine $n$ given $P$ and $nP$ by brute force
would require checking $P, 2P, 3P, 4P, 5P, \dots$ in $\Theta(n)$, which is infeasible for
large values of $n$
(a brute force attack on the NIST P192 curve would require on average approximately
$\frac{1}{2}p\approx3\times10^{57}$ additions, completely infeasible on any computer).
Hence the security of ECC relies on the difficulty of computing the integer
$n$ given the points $P$ and $nP$,
which is named the \defn{elliptic curve discrete logarithm problem}.

ECC can be utilized for key exchange by making public the
chosen field ($p$ or $2^m$), the curve ($a$ and $b$ values in the field),
and a point $G$ in the resulting abelian group for which there are many multiples
$G, 2G, 3G, \dots, nG$ that are all distinct
(eventually the multiples start to repeat when for some integer value of $n$ called the
\defn{order} of $G$ one finds that $nG=O$).
When two processes want to exchange a secret key each selects its own secure
random number $s<n$ as its private key and calculates a public key
$Q=sG$, which is communicated to the other process.
When each process receives the public key $Q^\prime=s^\prime G$ of the other process
(but not knowing the other private key $s^\prime$), it calculates the secret key $K$:
\begin{displaymath}
  K = s Q^\prime.
\end{displaymath}
Note that as in Diffie-Hellman key exchange this gives the same key for
both processes as:
\begin{displaymath}
  s^\prime Q
  = \underbrace{Q+Q+\cdots+Q}_{s^\prime\mbox{ times}}
  = \underbrace{G+G+\cdots+G}_{s\cdot s^\prime\mbox{ times}}
  = \underbrace{Q^\prime+Q^\prime+\cdots+Q^\prime}_{s\mbox{ times}}
  = s Q^\prime.
\end{displaymath}

JCA provides support for Elliptic Curve Cryptography, including
the interface \begin{code}ECField\end{code}
with implementing classes \begin{code}ECFieldFp\end{code} and \begin{code}ECFieldF2m\end{code}
for representing finite fields of order $p$ and $2^m$ respectively,
the interface \begin{code}ECKey\end{code} with subinterfaces
\begin{code}ECPublicKey\end{code}, \begin{code}ECPrivateKey\end{code}
and implementing classes \begin{code}ECPublic\-KeySpec\end{code},
\begin{code}ECPrivateKeySpec\end{code}, and classes
\begin{code}EllipticCurve\end{code}, \begin{code}ECParameterSpec\end{code},
\begin{code}ECGenParameterSpec\end{code} for creating elliptic curves and suitable
parameters for the cipher. The cipher for ECC is named \begin{code}ECIES\end{code}
and the key agreement is named \begin{code}ECDH\end{code}.
For example, the Eclipse project \begin{code}AndroidECKeyExchangeDemo\end{code}
has the activity \begin{code}AndroidECKeyExchangeActivity\end{code}
that demonstrates ECC key exchange using the fixed elliptic curve secp256r1.


The Bouncy Castle lightweight API also supports Elliptic Curve Cryptography,
including the class \begin{code}ECCurve\end{code} with subclasses
\begin{code}ECCurve.Fp\end{code} and \begin{code}ECCurve.F2m\end{code}
for representing prime and binary curves,
\begin{code}ECPoint\end{code} for representing a point on an elliptic curve,
and \begin{code}ECDomainParameters\end{code} for suitable parameters for the cipher.
The MIDlet \begin{code}ECKeyExchangeMIDlet\end{code} demonstrates how ECC can be used for
key exchange. It creates an \begin{code}ECCurve\end{code} for the fixed elliptic curve secp256r1.
Using the parameters for the curve and a point
(\begin{code}GX\end{code},\begin{code}GY\end{code}) on the curve
an \begin{code}AsymmetricCipher\-KeyPair\end{code}
is generated and its public key made available to the other process.
Once the public key is received from the other process
a \begin{code}BasicAgreement\end{code} is used to calculate the
secret key.
Note that the generation of the key pair is very computationally intensive,
in practice this step would need to be performed in a background thread.

Besides the shorter key size required for ECC as compared to other public-key encryption
systems, ECC encryption is not vulnerable to \defn{index calculus} attacks
that are used against RSA and Diffie-Hellman, which make use of certain
special properties of the groups created by RSA and Diffie-Hellman,
properties which are not found in the ECC groups $E(a,b)$.
The best known current attack against ECC is believed to be the \defn{Pollard $\rho$ attack},
one of a class of cryptanalysis techniques known as a \defn{collision search}.
However, even its computational requirements grow exponentially with the size of the field,
quickly becoming infeasible for moderate values of $p$.


\begin{exercise}[ECC Key Exchange]\label{Ex:AESEncryptionMobile}
	Use either the Android example\\
	 \begin{code}Andr\-oidECKeyExchangeActivity\end{code}
	or the Java ME Bouncy Castle lightweight example\\ \begin{code}ECKeyExchangeMIDlet\end{code}
	to prepare a mobile application that uses SMS communication to perform an ECC key exchange.
\end{exercise}
\begin{figure*}\begin{program}\begin{verbatim}
		/**
		* Android Activity that demonstrates ECC key exchange using the fixed
		* elliptic curve secp256r1
		* @author Andrew Ensor
		*/
		package aut.hss;
		...
		import java.math.BigInteger;
		import java.security.InvalidAlgorithmParameterException;
		import java.security.InvalidKeyException;
		import java.security.KeyPair;
		import java.security.KeyPairGenerator;
		import java.security.NoSuchAlgorithmException;
		import java.security.interfaces.ECPrivateKey;
		import java.security.interfaces.ECPublicKey;
		import java.security.spec.ECField;
		import java.security.spec.ECFieldFp;
		import java.security.spec.ECParameterSpec;
		import java.security.spec.ECPoint;
		import java.security.spec.EllipticCurve;
		import java.util.Arrays;
		import javax.crypto.KeyAgreement;
		...
		public class AndroidECKeyExchangeActivity extends Activity implements
		OnClickListener
		{
		private Button exchangeButton;
		private TextView statusTextView;
		// well-known elliptic curve secp256r1 (256 bit prime, 32 byte)
		private static final BigInteger P = new BigInteger
		("FFFFFFFF00000001000000000000000000000000FFFFFFFFFFFFF...", 16);
		private static final BigInteger A = new BigInteger
		("FFFFFFFF00000001000000000000000000000000FFFFFFFFFFFFF...", 16);
		private static final BigInteger B = new BigInteger
		("5AC635D8AA3A93E7B3EBBD55769886BC651D06B0CC53B0F63BCE3...", 16);
		private static final BigInteger GX = new BigInteger
		("6B17D1F2E12C4247F8BCE6E563A440F277037D812DEB33A0F4A13...", 16);
		private static final BigInteger GY = new BigInteger
		("4FE342E2FE1A7F9B8EE7EB4A7C0F9E162BCE33576B315ECECBB64...", 16);
		private static final BigInteger N = new BigInteger
		("FFFFFFFF00000000FFFFFFFFFFFFFFFFBCE6FAADA7179E84F3B9C...", 16);
		private static final int H = 1; // cofactor
		
		// inner class that represents one end of a key exchange
		private static class KeyExchangeProcess
		{
		private ECPublicKey publicKey;
		private ECPrivateKey privateKey;
		\end{verbatim}\hfill \emph{cont-}\end{program}\end{figure*}%
\begin{figure*}\begin{program}\emph{-cont}\begin{verbatim}
		public KeyExchangeProcess()
		{  // create the elliptic curve
		ECField field = new ECFieldFp(P);
		EllipticCurve curve = new EllipticCurve(field, A, B);
		// use existing precalculated values of G and N
		ECPoint G = new ECPoint(GX, GY);
		ECParameterSpec ecSpec = new ECParameterSpec(curve, G, N, H);
		// create public and private keys
		try
		{  KeyPairGenerator kpg = KeyPairGenerator.getInstance("EC");
		kpg.initialize(ecSpec); // size of key in bits
		System.out.println("About to generate a key pair");
		KeyPair keyPair = kpg.generateKeyPair();
		System.out.println("Finished generating a key pair");
		privateKey = (ECPrivateKey)keyPair.getPrivate();
		publicKey = (ECPublicKey)keyPair.getPublic();
		
		}
		catch (NoSuchAlgorithmException e)
		{  System.err.println
		("Encryption algorithm not available: " + e);
		}
		catch (InvalidAlgorithmParameterException e)
		{  System.err.println
		("Invalid Parameter in Encryption algorithm: " + e);
		}
		}
		
		public ECPublicKey getPublicKey()
		{  return publicKey;
		}
		
		public byte[] calculateSecretKey(ECPublicKey otherPublicKey)
		{  // generate secret key using Elliptic Curve Diffie-Hellman
		byte[] key = null;
		try
		{  KeyAgreement ka = KeyAgreement.getInstance("ECDH");
		ka.init(privateKey); // initialize with own private key
		ka.doPhase(otherPublicKey, true);
		key = ka.generateSecret();
		}
		catch (NoSuchAlgorithmException e)
		{  System.err.println
		("Key agreement algorithm not available: " + e);
		}
		catch (InvalidKeyException e)
		{  System.err.println
		("Invalid key in key agreement algorithm: " + e);
		}
		return key;
		}
		}
		}
		\end{verbatim}\end{program}\end{figure*}
\begin{figure*}\begin{program}\begin{verbatim}
/**
   A MIDlet that demonstrates ECC key exchange via the Bouncy Castle
   lightweight API
   @author Andrew Ensor
*/
...
public class ECKeyExchangeMIDlet extends MIDlet
   implements CommandListener
{
   private Display display;
   private Form form;
   private Command exitCommand, exchangeCommand;
   // well-known elliptic curve secp256r1 (256 bit prime, 32 byte)
   private static final BigInteger P = new BigInteger
      ("FFFFFFFF00000001000000000000000000000000FFFFFFFFFFFFF...", 16);
   private static final BigInteger A = new BigInteger
      ("FFFFFFFF00000001000000000000000000000000FFFFFFFFFFFFF...", 16);
   private static final BigInteger B = new BigInteger
      ("5AC635D8AA3A93E7B3EBBD55769886BC651D06B0CC53B0F63BCE3...", 16);
   private static final BigInteger GX = new BigInteger
      ("6B17D1F2E12C4247F8BCE6E563A440F277037D812DEB33A0F4A13...", 16);
   private static final BigInteger GY = new BigInteger
      ("4FE342E2FE1A7F9B8EE7EB4A7C0F9E162BCE33576B315ECECBB64...", 16);
   private static final BigInteger N = new BigInteger
      ("FFFFFFFF00000000FFFFFFFFFFFFFFFFBCE6FAADA7179E84F3B9C...", 16);
   private static final int H = 1; // cofactor

   ...
   // implementation of the CommandListener interface
   public void commandAction(Command c, Displayable s)
   {  if (c == exitCommand)
         exitMIDlet();
      else if (c == exchangeCommand)
      {  KeyExchangeProcess processA = new KeyExchangeProcess();
         KeyExchangeProcess processB = new KeyExchangeProcess();
         CipherParameters publicKeyA = processA.getPublicKey();
         CipherParameters publicKeyB = processB.getPublicKey();
         BigInteger secretKeyA=processA.calculateSecretKey(publicKeyB);
         BigInteger secretKeyB=processB.calculateSecretKey(publicKeyA);
         if (secretKeyA.equals(secretKeyB))
         {  form.append("Secret keys do agree and have length "
               + (secretKeyA.bitLength()+7)/8 + " bytes\n");
            form.append("Secret key: ".concat(secretKeyA.toString()));
         }
         else
         {  form.append("Error: secret keys do not agree!");
            form.append("Secret key: ".concat(secretKeyA.toString()));
            form.append("Secret key: ".concat(secretKeyB.toString()));
         }
      }
   }
\end{verbatim}\hfill \emph{cont-}\end{program}\end{figure*}%
\begin{figure*}\begin{program}\emph{-cont}\begin{verbatim}
   // inner class that represents one end of a key exchange
   private class KeyExchangeProcess
   {
      private CipherParameters publicKey, privateKey;

      public KeyExchangeProcess()
      {  // create the elliptic curve
         ECCurve.Fp curve = new ECCurve.Fp(P, A, B);
         // use existing precalculated values of G and N
         ECPoint G = new ECPoint.Fp(curve,
            new ECFieldElement.Fp(P,GX), new ECFieldElement.Fp(P,GY));
         ECDomainParameters params = new ECDomainParameters(curve, G,
            N);
         // create public and private keys
         SecureRandom generator = new SecureRandom();
         ECKeyGenerationParameters ecSpec
            = new ECKeyGenerationParameters(params, generator);
         ECKeyPairGenerator kpg = new ECKeyPairGenerator();
         kpg.init(ecSpec);
         System.out.println("About to generate a key pair");
         AsymmetricCipherKeyPair keyPair=kpg.generateKeyPair(); //slow
         System.out.println("Finished generating a key pair");
         publicKey = keyPair.getPublic();
         privateKey = keyPair.getPrivate();
      }

      public CipherParameters getPublicKey()
      {  return publicKey;
      }

      public BigInteger calculateSecretKey
         (CipherParameters otherPublicKey)
      {  // generate secret key using
         BasicAgreement ba = new ECDHBasicAgreement();
         ba.init(privateKey); // initialize with own private key
         BigInteger key = ba.calculateAgreement(otherPublicKey);
         return key;
      }
   }
}
\end{verbatim}\end{program}\end{figure*}


