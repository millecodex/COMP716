% This is the LaTeX file for the chapter:
%   Symmetric Ciphers
% for the manual:
%   Highly Secure Systems
% Written by Andrew Ensor
% Last updated 4 December 2018 by Jeff Nijsse

\chapter{Symmetric Ciphers}

%%%%%%%%%%%%%%%%%%%%%%%%%%%%%%%%%%%%%%%%%%%%%%%%%%%%%%%%%%%%%%%%%%%%%%%%%%%%%%%%%%%%%%%%%
%%%%% Classical Encryption Techniques                                               %%%%%
%%%%%%%%%%%%%%%%%%%%%%%%%%%%%%%%%%%%%%%%%%%%%%%%%%%%%%%%%%%%%%%%%%%%%%%%%%%%%%%%%%%%%%%%%

\section{Classical Encryption Techniques}

\reading{pp29-54}

An \defn{encryption} algorithm is an algorithm which takes a message,
called \defn{plaintext}, and converts it into a coded message, called
\defn{ciphertext}, that is not readily intelligible without special knowledge.
For encryption to be useful it must be possible to reverse the algorithm,
decrypting the ciphertext and obtaining the original plaintext.
The study of encryption is known as \defn{cryptography}, whereas
\defn{cryptanalysis} uses techniques to decipher ciphertext
without being provided with full knowledge of the enciphering details,
thus \defn{breaking} the code and obtaining the original plaintext.

Encryption algorithms are usually built using combinations of two simple techniques:
\begin{description}
  \item[substitution] where elements in the plaintext are replaced (substituted)
  with other elements,
  \item[transposition] where elements in the plaintext are rearranged (transposed)
  with each other.
\end{description}
An encryption algorithm makes use of a \defn{key} that helps determine the
resulting ciphertext, with the aim of making it difficult to decipher ciphertext
without prior knowledge of the correct key.

One of the earliest and simplest substitution ciphers is known as the
\defn{Caesar cipher} as it was used by Julius Caesar during military campaigns.
Each letter of plaintext is replaced by a letter some fixed number of positions
further down the alphabet, wrapping around beyond the letter \begin{code}z\end{code}
back to the letter \begin{code}a\end{code}.
For example, using a key that gives a rotation of three places and denoting
plaintext in lowercase and ciphertext in uppercase gives:
\begin{quote}\begin{tabular}{ll}
  \emph{Plaintext:} & \begin{code}secure and reliable systems\end{code} \\
  \emph{Ciphertext:}& \begin{code}VHFXUH DQG UHOLDEOH VBVWHPV\end{code}
\end{tabular}\end{quote}
where spaces have been included for clarity.
Decryption is achieved simply by rotating each letter of the ciphertext back three places.
Note that it is essential for the security of the Caesar cipher that this algorithm
be unknown to anyone intercepting the ciphertext, as by brute-force the
ciphertext could then be decrypted by trying all 26 possible cases of a key,
presuming the original plaintext would be recognizable when found.

A \defn{monoalphabetic substitution cipher} generalizes and improves on the Caesar cipher by
using substitutions determined by a random permutation of the letters of the alphabet,
replacing the letter \begin{code}a\end{code} by the first letter in the permutation,
\begin{code}b\end{code} by the second, and so forth.
For example, using \begin{code}\emph{qwertyuiopasdfghjklzxcvbnm}\end{code} as the key,
so that the letter \begin{code}a\end{code} is substituted by \begin{code}Q\end{code}
and \begin{code}s\end{code} by \begin{code}L\end{code} gives:
\begin{quote}\begin{tabular}{ll}
  \emph{Plaintext:} & \begin{code}secure and reliable systems\end{code} \\
  \emph{Ciphertext:}& \begin{code}LTEXKT QFR KTSOQWST LNLZTDL\end{code}
\end{tabular}\end{quote}
(in practice spaces too would be encoded if included as part of
the plaintext). A brute-force approach would require checking on
average half of the $26!\approx4\times10^{26}$ possible keys,
which is totally infeasible. However, like any simple substitution
cipher this cipher has a fatal flaw in that it preserves much of
the structure of the plaintext. If the type of information in the
plaintext is known, such as English text, then a \defn{frequency
analysis} can be performed on the elements of the ciphertext,
comparing the frequencies of each ciphertext element with expected
frequencies for that type of text. For example, if a cryptanalyst
suspects that \begin{code}LTEXKTQFRKTSOQWSTLNLZTDL\end{code} is
encoded from English plaintext then the fact that
\begin{code}T\end{code} occurs the most times in the ciphertext
suggests that it is probably the ciphertext for the letter
\begin{code}e\end{code}, and the other frequencies give further
insight into the key. Provided with a longer ciphertext or
multiple ciphertexts encrypted with the same key a cryptanalyst
has no problems in breaking the code.
If parts of the plaintext are known (for example, many file
formats such as PDF start with a well-known fixed header),
known to contains a particular sequence repeated (such as
\begin{code}"the"\end{code}), or worse still if a cryptanalyst has
access to a sample of both plaintext and its corresponding
ciphertext, the code can be easily broken.
\begin{figure*}[htb]
\begin{tabular}{|c|r|c|r|c|r|} \hline
  \multicolumn{6}{|l|}{\textit{Relative Frequency of Letters in English Text}} \\ \hline\hline
  \textit{Letter} & \textit{Frequency} & \textit{Letter} & \textit{Frequency}
  & \textit{Letter} & \textit{Frequency} \\ \hline
  a & 8.167\% &
  b & 1.492\% &
  c & 2.782\% \\
  d & 4.253\% &
  e & 12.702\% &
  f & 2.228\% \\
  g & 2.015\% &
  h & 6.094\% &
  i & 6.996\% \\
  j & 0.153\% &
  k & 0.772\% &
  l & 4.025\% \\
  m & 2.406\% &
  n & 6.749\% &
  o & 7.507\% \\
  p & 1.929\% &
  q & 0.095\% &
  r & 5.987\% \\
  s & 6.327\% &
  t & 9.056\% &
  u & 2.758\% \\
  v & 0.978\% &
  w & 2.360\% &
  x & 0.150\% \\
  y & 1.974\% &
  z & 0.074\% & & \\ \hline
\end{tabular}
\end{figure*}

One attempt to improve on the monoalphabetic substitution cipher is to assign
multiple ciphertext keys to common letters such as \begin{code}e\end{code} to ensure
the ciphertext does not contain any particular letters with noticeable frequency.
However, this is still susceptible to a frequency analysis since certain combinations
of pairs of letters, called \defn{bigrams}, occur more frequently than others.
Given a suitably large ciphertext the code could still be successfully broken using
a frequency analysis of bigrams.
\begin{figure*}[htb]
\begin{tabular}{|c|r|c|r|c|r|} \hline
  \multicolumn{6}{|l|}{\textit{Relative Frequency of Bigrams in English Text}} \\ \hline\hline
  \textit{Bigram} & \textit{Frequency} & \textit{Bigram} & \textit{Frequency}
  & \textit{Bigram} & \textit{Frequency} \\ \hline
  th & 3.883\% &
  he & 3.681\% &
  in & 2.284\% \\
  er & 2.178\% &
  an & 2.140\% &
  re & 1.749\% \\
  nd & 1.572\% &
  on & 1.418\% &
  en & 1.383\% \\ \hline
\end{tabular}
\end{figure*}

A \defn{Hill cipher} encrypts blocks of $m$ elements at a time using an
$m\times m$ matrix $K=\left(k_{ij}\right)$ as the key.
A plaintext message is broken into blocks
each with $m$ elements (possibly requiring the end of the plaintext to be padded
so its length is a multiple of $m$), then each block $p_1p_2\dots p_m$ is encoded
to a block $c_1c_2\dots c_m$ of ciphertext using matrix multiplication modulo $26$:
\begin{displaymath}
  \left(\begin{array}{c}
    c_1 \\ c_2 \\ \vdots \\ c_m
  \end{array}\right)
  =
  \left(\begin{array}{cccc}
    k_{11} & k_{12} & \cdots & k_{1m} \\
    k_{21} & k_{22} & \cdots & k_{2m} \\
    \vdots & \vdots &        & \vdots \\
    k_{m1} & k_{m2} & \cdots & k_{mm} \\
  \end{array}\right)
  \left(\begin{array}{c}
    p_1 \\ p_2 \\ \vdots \\ p_m
  \end{array}\right)
  \mbox{ mod $26$}
\end{displaymath}
where each letter is assigned a numerical value, \begin{code}a\end{code} as $0$,
\begin{code}b\end{code} as $1$, \dots, \begin{code}z\end{code} as $25$, and multiplication
and addition are performed modulo $26$.
The matrix $K$ must be invertible (modulo $26$) to ensure that the ciphertext can be reversed
using the inverse matrix $K^{-1}$.

As an example, the matrix
$K=\left(\begin{array}{ccc}8&21&21\\5&8&12\\10&21&8\end{array}\right)$ can be used to
encrypt the plaintext \begin{code}secureandreliablesystems\end{code} three characters
at a time, where the first block \begin{code}sec\end{code} would be encoded by:
\begin{displaymath}
  \left(\begin{array}{ccc}8&21&21\\5&8&12\\10&21&8\end{array}\right)
  \left(\begin{array}{c}18\\4\\2\end{array}\right)
  \mbox{ mod $26$} =
  \left(\begin{array}{c}10\\16\\20\end{array}\right)
\end{displaymath}
corresponding to the ciphertext \begin{code}KQU\end{code}.
The entire plaintext would be encoded as \begin{code}KQU DYR YKL JPE HAK ERA HYQ MUU\end{code}.
As a $3\times3$ matrix has been used it is not susceptible to a frequency analysis
of bigrams, and the larger the size of matrix the more of the structure of the plaintext
that is hidden. However, the Hill cipher is particularly vulnerable to a
\defn{plaintext attack}, where some plaintext and its corresponding ciphertext are
known. Possibly with just as few as $m$ blocks of plaintext and its ciphertext the
entries in the matrix $K$ could be determined by solving a system of $m$ linear equations,
and so too the inverse matrix $K^{-1}$ found for decryption.

The \defn{Vigen\`ere cipher} was originally
described in 1553 and is an example of a \defn{polyalphabetic substitution cipher},
where there are several monoalphabetic substitution ciphers available and the one chosen
for each element of plaintext depends on the key.
The Vigen\`ere cipher simply uses the 26 possible Caesar ciphers and cycles through
the letters of a key to determine which rotation to use for each element.
For example, encrypting \begin{code}secureandreliablesystems\end{code}
using the key \begin{code}\emph{emily}\end{code} gives:
\begin{quote}\begin{tabular}{ll}
  \emph{Key:}       & \begin{code}\emph{emilye mil yemilyem ilyemil}\end{code} \\
  \emph{Plaintext:} & \begin{code}secure and reliable systems\end{code} \\
  \emph{Ciphertext:}& \begin{code}WQKFPI MVO PIXQLZPQ AJQXQUD\end{code}
\end{tabular}\end{quote}
Note that one strength of the Vigen\`ere cipher is that a frequent letter
such as \begin{code}e\end{code} gets encoded to different ciphertext elements
depending on its position in the plaintext.
The first successful cryptanalyst attack on the Vigen\`ere cipher
was made in 1854 by Charles Babbage, but is known as the \defn{Kasiski examination}.
It works by first attempting to determine the length $m$ of the key.
Since occurrences of common words such as \begin{code}"the"\end{code} have a $\frac{1}{m}$
chance of being encrypted using the same key letters,
if the plaintext is many times longer than the key then repeated patterns will
result in the ciphertext
(such as the repeated pattern \begin{code}PI\end{code} in the above ciphertext example).
These repeated patterns would occur at multiples of the key length, allowing
guesses at the probable value of $m$. Once $m$ is determined the Vigen\`ere cipher
is quickly broken as the ciphertext is next split into $m$ sections, one per letter
of the key and up to $26$ brute-force attacks or a frequency analysis used to break
each of the $m$ separate Caesar ciphers.

The \defn{one-time pad} is a modification of the Vigen\`ere cipher where a random key
is used that is at least as long as the message to be transmitted, and then the key
gets discarded. Although this can be shown to give a theoretically secure cipher,
for most applications it is not practical to first distribute a secret key for
each individual message.

Instead, \defn{rotor machines} such as the famous German \defn{Enigma machine}
provided a practical implementation of a polyalphabetic substitution cipher that
were widely used during the period 1930-1950 for encryption.
A rotor machine is an electro-mechanical device consisting of a series of rotors,
each which has connections hard-wired to perform a simple monoalphabetic substitution.
After each element is encoded the rotor advances one position and changes the substitution.
Using just a single rotor would result in a key of length $26$, so in order to confuse
attempts at cryptanalysis the output from one rotor would be used as input to another
which would rotate at a different rate. In practice, at least three rotors were used
ensuring that there is no repetition in the pattern for $26^3=17576$ letters.
However, there were successful cryptanalysis attacks during the period 1932-1945
on all but the most complex rotor machines,
particularly by the Polish and British using algebraic techniques.

Transposition techniques rely on permuting the order of elements in the plaintext.
For instance, a \defn{rail fence cipher} writes plaintext elements row by row
using say six columns then reads the elements column by column:
\begin{quote}
  \begin{code}secure\end{code} \\
  \begin{code}andrel\end{code} \\
  \begin{code}iables\end{code} \\
  \begin{code}ystems\end{code} \\
  \emph{Ciphertext:} \begin{code}SAIYENASCBDTURLEREEMELSS\end{code}
\end{quote}
This can be later decoded by writing the ciphertext column by column and reading off row by row.
This algorithm could be improved by using a key that is a permutation of
$(1,2,3,4,5,6)$ to rearrange the columns before the ciphertext is read off:\nudgedown{1}

\noindent
\begin{minipage}{32mm}
\begin{quote}
  \begin{code}secure\end{code} \\
  \begin{code}andrel\end{code} \\
  \begin{code}iables\end{code} \\
  \begin{code}ystems\end{code} \\
\end{quote}
\end{minipage}
\hfill \shortstack{permute\\ $(3,1,4,5,2,6)$\\ $\Longrightarrow$} \hfill
\begin{minipage}{83mm}
\begin{quote}
  \begin{code}csuree\end{code} \\
  \begin{code}darenl\end{code} \\
  \begin{code}bileas\end{code} \\
  \begin{code}tyemss\end{code} \\
 \emph{Ciphertext:} \begin{code}CDBTSAIYURLEREEMENASELSS\end{code}
\end{quote}
\end{minipage}

\noindent
If the frequency of ciphertext elements were found to be similar to the expected frequencies
in plaintext then a pure transposition cipher would be suspected by a cryptanalyst.
If only a single transposition were used then it could be broken by
guessing the number $m$ of columns and testing their various $m!$ permutations.
The encryption can however be made much harder to break by repeating the process
multiple times using the same or several keys.


\begin{exercise}[Steganography]
Rather than making a message unintelligible \defn{steganography} instead
tries to hide plaintext embedded inside an innocuous message.
Prepare a program that can embed or extract a text message in an image by
using the least-significant bit of the alpha (transparency) values for the image
to store the plaintext bit by bit.
\end{exercise}


%%%%%%%%%%%%%%%%%%%%%%%%%%%%%%%%%%%%%%%%%%%%%%%%%%%%%%%%%%%%%%%%%%%%%%%%%%%%%%%%%%%%%%%%%
%%%%% Data Encryption Standard                                                      %%%%%
%%%%%%%%%%%%%%%%%%%%%%%%%%%%%%%%%%%%%%%%%%%%%%%%%%%%%%%%%%%%%%%%%%%%%%%%%%%%%%%%%%%%%%%%%

\section{Data Encryption Standard}\label{Se:DataEncryptionStandard}

\reading{pp63-89, 181-187}

Contrary to some classical encryption techniques, usually no attempt is made to
keep modern cryptography algorithms secret, most are widely known and so
well tested for their security against known cryptanalysis attacks.
If the encryption and decryption processes both use the same key then the
encryption is known as \defn{symmetric} or \defn{single-key} encryption.
If instead the encryption and decryption use different keys then the
encryption is known as \defn{asymmetric} or \defn{public-key} encryption.
Examples of symmetric ciphers include DES
(discussed in this section),
AES (Section \ref{Se:AdvancedEncryptionStandard}),
as well as triple DES and RC4 (Section \ref{Se:OtherSymmetricCiphers}),
whereas RSA (Section \ref{Se:RSAAlgorithm}) and
ECC (Section \ref{Se:EllipticCurveCryptography}) are both public-key ciphers.
Encryption algorithms are also classified either as \defn{block ciphers}
if they process plaintext an entire block at a time, or else as \defn{stream ciphers}
if they can process plaintext one element at a time as the plaintext is read.
For example, DES is a block cipher that typically processes 64-bit blocks of plaintext
at a time, AES is another block cipher designed for 128-bit blocks, whereas
RC4 is a stream cipher that processes the plaintext one byte at a time.

The \defn{Data Encryption Standard} (DES) is the most widely-used symmetric cipher,
although it is planned to eventually be replaced by more modern algorithms
such as AES and triple-DES, which are presumed to be more resistant to cryptographic attacks.
DES operates on 64-bit (8 byte) blocks of plaintext $p_1p_2p_3\dots p_{63}p_{64}$ at a time
to produce 64-bit $c_1c_2c_3\dots c_{63}c_{64}$ ciphertext.
The ciphertext is formed using a 64-bit key (of which only 56 bits are actually utilized).
Firstly, the plaintext bits are permuted using a standard \defn{initial permutation}
giving bits $p_{58}p_{50}p_{42}\dots p_{15}p_7$, which is then split into two
32-bit halves $L_0$ and $R_0$.
These two halves then pass through 16 rounds of substitutions and transpositions
to progressively give $L_1$, $R_1$, then $L_2$, $R_2$, etc until $L_{16}$, $R_{16}$ are obtained.
At the start of each round the key is used to obtain a \defn{subkey} $K_i$ of 48 bits,
by using various rotations and permutations of the bits of the key, and $L_i$, $R_i$
are given by the recurrence:
\begin{eqnarray*}
  L_i &=& R_{i-1} \\
  R_i &=& L_{i-1}\oplus F\left(R_{i-1},K_i\right)
\end{eqnarray*}
where $\oplus$ denotes the exclusive-OR (XOR) bitwise operation
(with $0\oplus0=0=1\oplus1$ and $0\oplus1=1=1\oplus0$) and
$F$ is a specific function for DES.
Once $L_{16}$, $R_{16}$ are obtained their order is swapped before they are
combined back together to give 64 bits and the inverse
of the initial permutation applied to finally give $c_1c_2c_3\dots c_{63}c_{64}$.

\begin{center}
\begin{pspicture}(-4.5,-1.5)(4.5,12)
  \rput(2,12.25){\rnode{DESin}{$p_1p_2p_3\dots p_{63}p_{64}$}}
  \rput(2,11.25){\rnode{IP}{\psframebox{\makebox[48mm]{Initial Permutation}}}}
  \ncline[arrows=->]{DESin}{IP}\naput{\scriptsize{64 bit}}
  \pnode(1,11){L0S} \pnode(3,11){R0S}
  \pnode(1,10){L0E} \pnode(3,10){R0E}
  \ncline{L0S}{L0E}\nbput{$L_0$} \ncline{R0S}{R0E}\naput{$R_0$}

  \rput(3,9){\rnode{F1}{\psframebox{F}}} \rput(4.25,9){\rnode{K1}{$K_1$}}
  \pnode(1,8){L1S} \rput(3,8){\rnode{R1S}{$\bigoplus$}}
  \pnode(1,7){L1E} \pnode(3,7){R1E}
  \ncline[arrows=->]{L0E}{R1S} \ncline{R0E}{L1S}
  \ncline[arrows=->]{R0E}{F1} \ncline[arrows=->]{K1}{F1}
  \ncline[arrows=->]{F1}{R1S}
  \ncline{L1S}{L1E}\nbput{$L_1$} \ncline{R1S}{R1E}\naput{$R_1$}

  \rput(3,6){\rnode{F2}{\psframebox{F}}} \rput(4.25,6){\rnode{K2}{$K_2$}}
  \pnode(1,5){L2S} \rput(3,5){\rnode{R2S}{$\bigoplus$}}
  \pnode(1,4){L2E} \pnode(3,4){R2E}
  \ncline[arrows=->]{L1E}{R2S} \ncline{R1E}{L2S}
  \ncline[arrows=->]{R1E}{F2} \ncline[arrows=->]{K2}{F2}
  \ncline[arrows=->]{F2}{R2S}
  \ncline[arrows=->]{L2S}{L2E}\nbput{$L_2$} \ncline[arrows=->]{R2S}{R2E}\naput{$R_2$}

  \rput(1,3.5){\vdots} \rput(3,3.5){\vdots}

  \pnode(1,3){L16S} \pnode(3,3){R16S}
  \pnode(1,2){L16E} \pnode(3,2){R16E}
  \ncline{L16S}{L16E}\nbput{$L_{16}$} \ncline{R16S}{R16E}\naput{$R_{16}$}

  \pnode(1,1){Lend} \pnode(3,1){Rend}
  \ncline[arrows=->]{L16E}{Rend} \ncline[arrows=->]{R16E}{Lend}

  \rput(2,0.75){\rnode{IIP}{\psframebox{\makebox[48mm]{Inverse of Initial Permutation}}}}
  \rput(2,-0.25){\rnode{DESout}{$c_1c_2c_3\dots c_{63}c_{64}$}}
  \ncline[arrows=->]{IIP}{DESout}\naput{\scriptsize{64 bit}}

  \rput(2,-1){DES Encryption Algorithm}
\end{pspicture}
\end{center}

Since the initial permutation and its inverse are well-known they do not really
contribute to the security of the algorithm, but the 16 rounds help diffuse
any statistical structure that might be in the plaintext across the entire block
and tries to confuse attempts at cryptanalysis.

In each round the function $F$ has as input a 32-bit
$R_{i-1}=r_1r_2r_3\dots r_{31}r_{32}$
partially encrypted half of the block and a 48-bit subkey $K_i$.
First it expands $R_{i-1}$ by duplicating some bits to give a 48-bit
$E\left(R_{i-1}\right)=r_{32}r_1r_2r_3\dots r_{31}r_{32}r_1$.
This is then passed through a bitwise XOR with the subkey
to give a 48-bit result.
Next, the function $F$ treats the 48 bits as eight groups of six bits,
and for each group uses a special \defn{S-box} to replace the six bits
by four bits, effectively reducing the previous 48 bits to eight groups of four bits,
or a 32-bit result.
For example, if $000000$ is in the first group it is replaced with $1110$ from
the S-box $S_1$, whereas if it is in the second group it is replaced
with $1111$ from the S-box $S_2$.
Finally, these 32 bits are passed through a permutation $P$
to give $F\left(R_{i-1},K_i\right)$.
\setlength{\columnsep}{1cm}

\begin{multicols}{2}
\noindent Although the eight S-boxes are well-known they have been a source of contention
amongst cryptographers as their original design criteria were kept confidential.
Hence suspicions arose as to whether they were designed with a weakness so that the
US National Security Agency could decipher messages without knowledge of the key.
Also of concern was the length of the key, only 56 bits, instead of the
original IBM proposal of using a more-secure 128-bit key. Although this still allows $2^{56}\approx7\times10^{16}$ possible keys, by 1998 DES-cracker devices were available that could break DES ciphertext via brute-force in less than three days, and by 2012 the keyspace could be exhausted in 26 hours with custom-built hardware. \\

\begin{pspicture}[shift=*](0.5,-0.75)(5.5,5)
  \psline(0.5,0)(0.5,4.75)(4.5,4.75)(4.5,0)(0.5,0)

  \rput(2.5,5.5){\rnode{Fin}{$R_{i-1}$}}

  \pnode(2.5,4.25){Ein}
  \ncline[arrows=->]{Fin}{Ein}\naput[npos=0.8]{\scriptsize{32 bit}}
  \rput(2.5,4){\rnode{S}{Expansion $E$}}
  \psline(1.5,4.25)(1,3.75)(4,3.75)(3.5,4.25)(1.5,4.25)
  \pnode(2.5,3.75){Eout}

  \rput(2.5,3){\rnode{XOR}{$\bigoplus$}}
  \ncline[arrows=->]{Eout}{XOR}\naput{\scriptsize{48 bit}}
  \rput(5,3){\rnode{K}{$K_i$}}
  \ncline[arrows=->]{K}{XOR}\nbput{\scriptsize{48 bit}}

  \rput(1.1,2){\rnode{S1}{\rule[-1.5mm]{0mm}{5mm}$S_1$}} % note: rule is strut
  \nccurve[angleA=270,angleB=90,arrows=->]{XOR}{S1}
  \rput(1.6,2.7){\scriptsize{6 bit}}
  \psline(0.9,1.75)(0.7,2.25)(1.5,2.25)(1.3,1.75)(0.9,1.75)
  \rput(2.1,2){\rnode{S2}{\rule[-1.5mm]{0mm}{5mm}$S_2$}}
  \nccurve[angleA=270,angleB=90,arrows=->]{XOR}{S2}
  \psline(1.9,1.75)(1.7,2.25)(2.5,2.25)(2.3,1.75)(1.9,1.75)
  \rput(3,2){\dots}
  \rput(3.9,2){\rnode{S8}{\rule[-1.5mm]{0mm}{5mm}$S_8$}}
  \nccurve[angleA=270,angleB=90,arrows=->]{XOR}{S8}
  \rput(3.4,2.7){\scriptsize{6 bit}}
  \psline(3.7,1.75)(3.5,2.25)(4.3,2.25)(4.1,1.75)(3.7,1.75)

  \pnode(2.5,1){Pin}
  \nccurve[angleA=270,angleB=90,arrows=->]{S1}{Pin}
  \rput(1.6,1.25){\scriptsize{4 bit}}
  \nccurve[angleA=270,angleB=90,arrows=->]{S2}{Pin}
  \nccurve[angleA=270,angleB=90,arrows=->]{S8}{Pin}
  \rput(3.4,1.25){\scriptsize{4 bit}}
  \rput(2.5,0.75){\rnode{P}{\psframebox{\makebox[25mm]{Permutation $P$}}}}
  \pnode(2.5,0.5){Pout}
  \rput(2.5,-0.75){\rnode{Fout}{$F\!\left(R_{i-1},K_i\right)$}}
  \ncline[arrows=->]{Pout}{Fout}\naput[npos=0.2]{\scriptsize{32 bit}}
\end{pspicture}
\end{multicols}

As DES is a symmetric cipher the same key is used to decrypt ciphertext
as well as encrypt plaintext. Decryption is achieved by simply reversing
the steps of the encryption, applying the subkeys in the reverse order
and recalculating the $F\left(R_{i-1},K_i\right)$.
Hence if DES is implemented in hardware, the same hardware can be used
for encryption as for decryption.

Any block cipher such as DES can be used with plaintext so long as
its length is an exact multiple of the block size. The simplest
way of encrypting multiple blocks of plaintext is known as
\defn{electronic code book} (ECB) mode, which simply encrypts each
block separately (using the same key for the encryption of each
block):

\noindent
\begin{center}
\begin{pspicture}(0,-0.25)(11.5,2.5)
  \psline(2,2)(2,2.5)(10,2.5)(10,2)(2,2)
  \psline[linestyle=dotted](4,2)(4,2.5)
  \psline[linestyle=dotted](6,2)(6,2.5)
  \psline[linestyle=dotted](8,2)(8,2.5)
  \psline[linestyle=dotted](10,2)(10,2.5)
  \rput(3,2.25){\rnode{P1}{\rule[-1.5mm]{0mm}{5mm}block}}
  \rput(5,2.25){\rnode{P2}{\rule[-1.5mm]{0mm}{5mm}block}}
  \rput(7,2.25){\rnode{P3}{\rule[-1.5mm]{0mm}{5mm}block}}
  \rput(9,2.25){\rnode{P4}{\rule[-1.5mm]{0mm}{5mm}block}}

  \rput(3,1.25){\rnode{E1}{\psframebox[fillstyle=solid,fillcolor=yellow]{encrypt}}}
  \rput(5,1.25){\rnode{E2}{\psframebox[fillstyle=solid,fillcolor=yellow]{encrypt}}}
  \rput(7,1.25){\rnode{E3}{\psframebox[fillstyle=solid,fillcolor=yellow]{encrypt}}}
  \rput(9,1.25){\rnode{E4}{\psframebox[fillstyle=solid,fillcolor=yellow]{encrypt}}}

  \psline(2,0)(2,0.5)(10,0.5)(10,0)(2,0)
  \psline[linestyle=dotted](4,0)(4,0.5)
  \psline[linestyle=dotted](6,0)(6,0.5)
  \psline[linestyle=dotted](8,0)(8,0.5)
  \psline[linestyle=dotted](10,0)(10,0.5)
  \rput(3,0.25){\rnode{C1}{\rule[-1.5mm]{0mm}{5mm}block}}
  \rput(5,0.25){\rnode{C2}{\rule[-1.5mm]{0mm}{5mm}block}}
  \rput(7,0.25){\rnode{C3}{\rule[-1.5mm]{0mm}{5mm}block}}
  \rput(9,0.25){\rnode{C4}{\rule[-1.5mm]{0mm}{5mm}block}}

  \psset{arrows=->}
  \ncline{P1}{E1}\ncline{E1}{C1}
  \ncline{P2}{E2}\ncline{E2}{C2}
  \ncline{P3}{E3}\ncline{E3}{C3}
  \ncline{P4}{E4}\ncline{E4}{C4}
\end{pspicture}
\end{center}
However, this has the disadvantage that identical
plaintext blocks are encrypted as identical ciphertext blocks, which could
reveal some patterns in the plaintext, particularly for longer plaintext.
To remove this vulnerability, the \defn{cipher block chaining} (CBC) mode
takes a bitwise XOR of each plaintext block with the previous block's
ciphertext before the plaintext block gets encrypted.
An \defn{initialization vector} of length one block is chosen and used for the bitwise
XOR with the first block of plaintext. For added security the initialization
vector should only be known to the sender and intended receiver (and might
initially be sent to the receiver using ECB mode).

\noindent
\begin{center}\begin{pspicture}(0,-0.25)(11.5,4)
  \psline(2,3.5)(2,4)(10,4)(10,3.5)(2,3.5)
  \psline[linestyle=dotted](4,3.5)(4,4)
  \psline[linestyle=dotted](6,3.5)(6,4)
  \psline[linestyle=dotted](8,3.5)(8,4)
  \psline[linestyle=dotted](10,3.5)(10,4)
  \rput(3,3.75){\rnode{P1}{\rule[-1.5mm]{0mm}{5mm}block}}
  \rput(5,3.75){\rnode{P2}{\rule[-1.5mm]{0mm}{5mm}block}}
  \rput(7,3.75){\rnode{P3}{\rule[-1.5mm]{0mm}{5mm}block}}
  \rput(9,3.75){\rnode{P4}{\rule[-1.5mm]{0mm}{5mm}block}}

  \rput(1,2.75){\rnode{IV}{\scriptsize{\begin{tabular}{c}initialization\\vector\end{tabular}}}}
  \rput(3,2.75){\rnode{X1}{$\bigoplus$}}
  \rput(5,2.75){\rnode{X2}{$\bigoplus$}}
  \rput(7,2.75){\rnode{X3}{$\bigoplus$}}
  \rput(9,2.75){\rnode{X4}{$\bigoplus$}}

  \rput(3,1.75){\rnode{E1}{\psframebox[fillstyle=solid,fillcolor=yellow]{encrypt}}}
  \rput(5,1.75){\rnode{E2}{\psframebox[fillstyle=solid,fillcolor=yellow]{encrypt}}}
  \rput(7,1.75){\rnode{E3}{\psframebox[fillstyle=solid,fillcolor=yellow]{encrypt}}}
  \rput(9,1.75){\rnode{E4}{\psframebox[fillstyle=solid,fillcolor=yellow]{encrypt}}}

  \pnode(3,1){B1}
  \pnode(5,1){B2}
  \pnode(7,1){B3}

  \psline(2,0)(2,0.5)(10,0.5)(10,0)(2,0)
  \psline[linestyle=dotted](4,0)(4,0.5)
  \psline[linestyle=dotted](6,0)(6,0.5)
  \psline[linestyle=dotted](8,0)(8,0.5)
  \psline[linestyle=dotted](10,0)(10,0.5)
  \rput(3,0.25){\rnode{C1}{\rule[-1.5mm]{0mm}{5mm}block}}
  \rput(5,0.25){\rnode{C2}{\rule[-1.5mm]{0mm}{5mm}block}}
  \rput(7,0.25){\rnode{C3}{\rule[-1.5mm]{0mm}{5mm}block}}
  \rput(9,0.25){\rnode{C4}{\rule[-1.5mm]{0mm}{5mm}block}}

  \psset{arrows=->}
  \ncline{IV}{X1}
  \ncline{P1}{X1}\ncline{X1}{E1}\ncline{E1}{C1}\ncangle[angleA=0,angleB=180,arm=8mm]{B1}{X2}
  \ncline{P2}{X2}\ncline{X2}{E2}\ncline{E2}{C2}\ncangle[angleA=0,angleB=180,arm=8mm]{B2}{X3}
  \ncline{P3}{X3}\ncline{X3}{E3}\ncline{E3}{C3}\ncangle[angleA=0,angleB=180,arm=8mm]{B3}{X4}
  \ncline{P4}{X4}\ncline{X4}{E4}\ncline{E4}{C4}
\end{pspicture}\end{center}

\noindent
As the input for the encryption of each successive plaintext block is modified depending
on the result of the previous encryption
CBC ensures that identical plaintext blocks probably get encrypted to
distinct ciphertext blocks.

The \defn{cipher feedback} (CFB) and the \defn{output feedback}
(OFB) modes can each be used to convert a block cipher such as DES so
that it can be used as a stream cipher that processes $s$ bits at a time.
Both these modes start with an initialization vector of length one block,
encrypt it and select the $s$ most-significant bits of the output.
These $s$ bits are passed through a bitwise XOR with the first $s$ bits
of the plaintext stream to produce the first $s$-bit ciphertext. Then the
bits in the initialization vector are shifted left by $s$ bits.
With CFB the $s$-bit ciphertext output is used as the new least significant bits
of the modified vector, whereas with OFB the $s$ most-significant bits of the
encryption output are instead used (before they get passed to the XOR with the plaintext
rather than after).
Then the process is repeated using the modified vector to
produce $s$-bit ciphertext for the next $s$ bits of plaintext.
Interestingly, decryption via CFB or OFB mode does not need to reverse the
encryption algorithm, as it can just use the same process as for encryption to obtain the
required $s$-bit outputs.
One potential disadvantage of CFB is that any errors during transmission
get propagated through the rest of the decryption process as each $s$-bit
ciphertext received gets used in the following decryptions.
Instead, with OFB a transmission error
of one $s$-bit ciphertext affects only its corresponding $s$-bit plaintext,
not any further plaintext, making it suitable for security over unreliable
communication channels.

If the length of plaintext is not an exact multiple of the block size
for a block cipher then \defn{padding} needs to be added
to fill out the remainder of the final block, which is removed when the
ciphertext is later decrypted.
Several padding schemes exist, such as simply appending zero bytes,
or instead the more complicated and popular padding scheme \defn{PKCS\#7}
which appends repeatedly with the number of bytes of padding
and helps complicate an attack that might try to
take advantage of the last block of ciphertext.

The \defn{Java Cryptography Architecture} (JCA) is included as part of Java Standard
Edition and include API in the packages \begin{code}java.security\end{code},
\begin{code}javax.crypto\end{code} and their subpackages
that support encryption, decryption, and key generation.
They allow for pluggable cryptographic security \defn{providers} which provide
implementations of various cryptographic algorithms, and include
a default provider called \begin{code}SunJCE\end{code}.
The following steps are used for a symmetric block cipher:
\begin{description}
  \item[Generate Key]
  A key for symmetric encryption can be randomly generated by creating a suitable
  \begin{code}KeyGenerator\end{code} for the encryption algorithm
  and using its \begin{code}generateKey\end{code} method:
\begin{quote}\begin{code}\begin{verbatim}
KeyGenerator kg = KeyGenerator.getInstance("DES");
SecretKey key = kg.generateKey();
\end{verbatim}\end{code}\end{quote}

  \item[Create Cipher]
  For either encryption or decryption a \begin{code}Cipher\end{code} object must
  first be created, specifying the name of the algorithm, and optionally
  its mode for handling multiple blocks and how it applies padding, such as:
\begin{quote}\begin{code}\begin{verbatim}
Cipher cipher=Cipher.getInstance("DES/ECB/PKCS5Padding");
\end{verbatim}\end{code}\end{quote}
  JCA supports modes \begin{code}ECB\end{code} for electronic code book,
  \begin{code}CBC\end{code} for cipher block chaining,
  \begin{code}CFB\end{code} for cipher feedback,
  \begin{code}OFB\end{code} for output feedback,
  as well as \begin{code}PCBC\end{code} for propagating cipher block chaining,
  and includes \begin{code}PKCS5Padding\end{code} and
  \begin{code}NoPadding\end{code} for padding.

  \item[Initialize Cipher]
  The cipher is then initialized either for encryption:
\begin{quote}\begin{code}\begin{verbatim}
cipher.init(Cipher.ENCRYPT_MODE, key);
\end{verbatim}\end{code}\end{quote}
  or else for decryption:
\begin{quote}\begin{code}\begin{verbatim}
cipher.init(Cipher.DECRYPT_MODE, key);
\end{verbatim}\end{code}\end{quote}

  \item[Encrypt or Decrypt]
  The \begin{code}Cipher\end{code} method \begin{code}doFinal\end{code}
  is used to encrypt or decrypt a \begin{code}byte[]\end{code} array:
\begin{quote}\begin{code}\begin{verbatim}
byte[] input = ...;
byte[] output = cipher.doFinal(input);
\end{verbatim}\end{code}\end{quote}
\end{description}

For example, the simple class \begin{code}DESExample\end{code} demonstrates
how DES encryption can be achieved using the JCA API.

\begin{figure*}\begin{program}\begin{verbatim}
/**
   A simple class that demonstrates how DES encryption is achieved
   using the Java Cryptography Architecture
   @author Andrew Ensor
*/
...
public class DESExample
{
   public static void main(String[] args)
   {  try
      {  // generate a secret key for DES
         KeyGenerator kg = KeyGenerator.getInstance("DES");
         SecretKey key = kg.generateKey();
         // create a cipher
         Cipher cipher = Cipher.getInstance("DES/ECB/PKCS5Padding");
         // obtain the plaintext
         Scanner keyboardInput = new Scanner(System.in);
         System.out.print("Please enter some plaintext:");
         byte[] plaintext = keyboardInput.nextLine().getBytes();
         // initialize cipher for encryption
         cipher.init(Cipher.ENCRYPT_MODE, key);
         // encrypt the plaintext
         byte[] ciphertext = cipher.doFinal(plaintext);
         System.out.print("Ciphertext is:");
         for (int i=0; i<ciphertext.length; i++)
         {  int unsignedByte = ciphertext[i] & 0xFF;
            System.out.print(Integer.toHexString(unsignedByte)+" ");
         }
         System.out.println();
         // decrypt the ciphertext
         cipher.init(Cipher.DECRYPT_MODE, key);
         byte[] deciphertext = cipher.doFinal(ciphertext);
         System.out.println("Deciphered is:"+new String(deciphertext));
      }
      catch (NoSuchAlgorithmException e)
      {  System.err.println("Encryption algorithm not available: "+e);
      }
      catch (NoSuchPaddingException e)
      {  System.err.println("Padding scheme not available: "+e);
      }
      catch (InvalidKeyException e)
      {  System.err.println("Invalid key: "+e);
      }
      catch (IllegalBlockSizeException e)
      {  System.err.println("Cannot pad plaintext: "+e);
      }
      catch (BadPaddingException e)
      {  System.err.println("Exception with padding: "+e);
      }
   }
}
\end{verbatim}\end{program}\end{figure*}

\begin{exercise}[Comparing ECB and CBC modes]
Use DES in ECB mode and then instead in CBC mode to encrypt the bytes in an image
and display the plaintext image as well as both ciphertext images.
Test with an image that has a uniform background.
\end{exercise}


%%%%%%%%%%%%%%%%%%%%%%%%%%%%%%%%%%%%%%%%%%%%%%%%%%%%%%%%%%%%%%%%%%%%%%%%%%%%%%%%%%%%%%%%%
%%%%% Finite Fields                                                                 %%%%%
%%%%%%%%%%%%%%%%%%%%%%%%%%%%%%%%%%%%%%%%%%%%%%%%%%%%%%%%%%%%%%%%%%%%%%%%%%%%%%%%%%%%%%%%%

\section{Finite Fields}

\reading{pp96-129}

The set $\left\{0, 1, 2, \dots, n-1\right\}$ of the first $n$ non-negative
integers is usually denoted by $\mathbb{Z}_n$ and plays an interesting role in
cryptography. Since adding together two integers each less than $n$ might result in
one that is greater than $n-1$ addition in $\mathbb{Z}_n$ is usually taken modulo $n$.
For example $\mathbb{Z}_4 = \{0,1,2,3\}$ has the following addition table:

\begin{tabular}{|c|cccc|}\hline
  $+$ & $0$ & $1$ & $2$ & $3$ \\ \hline
  $0$ & $0$ & $1$ & $2$ & $3$ \\
  $1$ & $1$ & $2$ & $3$ & $0$ \\
  $2$ & $2$ & $3$ & $0$ & $1$ \\
  $3$ & $3$ & $0$ & $1$ & $2$ \\ \hline
\end{tabular}
\qquad\qquad
$\underbrace{a+b}_{\mbox{add in }\mathbb{Z}_n}
= \underbrace{(\mbox{(int)}a + \mbox{(int)}b)}_{\mbox{add in }\mathbb{Z}} \mbox{ mod } n$.

\noindent With this form of addition modulo $n$, $\mathbb{Z}_n$ gives an important example
of what is known as an abelian group.

A \defn{group} $(G, \bullet)$ is a set $G$ together with a binary function
$\bullet\colon G\times G\rightarrow G$ that obeys the following laws:
\begin{description}
  \item[Associativity] for all elements $a$, $b$, $c$ in $G$ one has
  $(a\bullet b)\bullet c = a\bullet(b\bullet c)$,
  \item[Identity] there is an element $e$ in $G$ for which $a\bullet e=a=e\bullet a$
  for every element $a$ in $G$,
  \item[Inverse] for every element $a$ in $G$ there is an element $a^\prime$ in $G$
  for which $a\bullet a^\prime=e=a^\prime\bullet a$.
\end{description}
If a group also satisfies:
\begin{description}
  \item[Commutativity] for all elements $a$, $b$ in $G$ one has $a\bullet b=b\bullet a$,
\end{description}
then the group is called \defn{abelian} or \defn{commutative}.

In cases where the group operation $\bullet$ is denoted by $+$, such as in
the abelian group $\mathbb{Z}_n$ with addition modulo $n$,
the identity $e$ is usually denoted by $0$
and the inverse $a^\prime$ of an element $a$ by $-a$.
Then a \defn{subtraction} operation can be defined by $a-b=a+(-b)$.

Multiplication in $\mathbb{Z}_n$ can also be performed modulo $n$.
For example $\mathbb{Z}_4$ has the following multiplication table:

\begin{tabular}{|c|cccc|}\hline
  $\cdot$ & $0$ & $1$ & $2$ & $3$ \\ \hline
  $0$ & $0$ & $0$ & $0$ & $0$ \\
  $1$ & $0$ & $1$ & $2$ & $3$ \\
  $2$ & $0$ & $2$ & $0$ & $2$ \\
  $3$ & $0$ & $3$ & $2$ & $1$ \\ \hline
\end{tabular}
\qquad\qquad
$\underbrace{a\cdot b}_{\mbox{mult in }\mathbb{Z}_n}
= \underbrace{(\mbox{(int)}a\cdot\mbox{(int)}b)}_{\mbox{mult in }\mathbb{Z}} \mbox{ mod } n$.

\noindent With both addition and multiplication taken modulo $n$,
$\mathbb{Z}_n$ is an example of what is known as a commutative ring.

A \defn{ring} $(R,+,\cdot)$ is an abelian group $(R,+)$ together with a second binary function
$\cdot\colon R\times R\rightarrow R$ that obeys the following laws:
\begin{description}
  \item[Associativity] for all elements $a$, $b$, $c$ in $R$ one has
  $(a\cdot b)\cdot c = a\cdot(b\cdot c)$,
  \item[Distributivity] for all elements $a$, $b$, $c$ in $R$ one has
  $a\cdot(b+c)=a\cdot b + a\cdot c$ and $(a+b)\cdot c=a\cdot c+b\cdot c$.
\end{description}
If a ring also satisfies:
\begin{description}
  \item[Commutativity] for all elements $a$, $b$ in $R$ one has $a\cdot b=b\cdot a$,
\end{description}
then the ring is said to be \defn{commutative}.

The \defn{greatest common divisor} of two positive integers $m$ and $n$
is the greatest positive number $\mbox{gcd}(m,n)$ that divides both $m$ and $n$.
For example, the divisors of $m=12$ are $1$, $2$, $3$, $4$, $6$, $12$,
whereas the divisors of $n=28$ are $1$, $2$, $4$, $7$, $14$, $28$.
Hence, the common divisors are $1$, $2$, $4$, so $\mbox{gcd}(12,28)=4$.
The Euclidean algorithm provides a simple procedure for determining the
greatest common divisor using just the integer modulo operation.

\begin{theorem}{Euclidean Algorithm for Positive Integers}
  Suppose $m$ and $n$ are positive integers. Then
  $\mbox{gcd}(m,n)=\mbox{gcd}(n \mbox{ mod } m, m)$,
  and so $\mbox{gcd}(m,n)$ can be calculated by taking repeated remainders
  $r_1>r_2>\dots>r_n=\mbox{gcd}(m,n)$ until one of the remainders $r_n$ is a divisor
  of the previous $r_{n-1}$:
  \begin{eqnarray*}
    r_1 &=& n \mbox{ mod } m \qquad\mbox{where $r_1<m$} \\
    r_2 &=& m \mbox{ mod } r_1 \qquad\mbox{where $r_2<r_1$} \\
    r_3 &=& r_1 \mbox{ mod } r_2 \qquad\mbox{where $r_3<r_2$} \\
    r_4 &=& r_2 \mbox{ mod } r_3 \qquad\mbox{where $r_4<r_3$} \\
    \vdots && \vdots \\
    r_n &=& r_{n-2} \mbox{ mod } r_{n-1} \qquad\mbox{where $r_n<r_{n-1}$} \\
    0 &=& r_{n-1} \mbox{ mod } r_n.
  \end{eqnarray*}
  Furthermore, there are integers $s$ and $t$ for which $\mbox{gcd}(m,n)=s\cdot m+t\cdot n$.
\end{theorem}

For example, $\mbox{gcd}(945,2415)$ can be found as follows:
\begin{eqnarray*}
  2415 &=& 2\cdot945+525 \qquad\mbox{so $r_1=525$} \\
  945 &=& 1\cdot525+420 \qquad\mbox{so $r_2=420$} \\
  525 &=& 1\cdot420+105 \qquad\mbox{so $r_3=105$} \\
  420 &=& 4\cdot105+0 \qquad\mbox{so $r_4=0$}.
\end{eqnarray*}
Hence $\mbox{gcd}(945,2415)=105$.
The values of $s$ and $t$ can then be found from these equations using them in reverse order:
\begin{eqnarray*}
  105 &=& 525 + (-1)\cdot420 \\
  &=& 525 + (-1)\cdot(945+(-1)\cdot525) \\
  &=& 2\cdot525+(-1)\cdot945 \\
  &=& 2\cdot(2415+(-2)\cdot945)+(-1)\cdot945 \\
  &=& (-5)\cdot945+2\cdot2415.
\end{eqnarray*}

Division in a ring can be more problematic than addition, subtraction, and multiplication.
For instance since $2\cdot2=0$ in $\mathbb{Z}_4$, it is not possible to
have a meaningful way to divide elements by $2$ (otherwise one could divide both
sides of the equation $2\cdot2=0$ by $2$ to obtain that $2=0$ which is a contradiction).
This problem arises whenever there are elements $a\neq0$ and
$b\neq0$ in the ring for which $a\cdot b=0$.
In the case of the ring $(\mathbb{Z}_n,+,\cdot)$ this problem can occur if and only if
$n$ is not a prime number.

If a commutative ring also satisfies:
\begin{description}
  \item[Identity] there is an element $1$ in $R$ for which $a\cdot1=a=1\cdot a$
  for every element $a$ in $R$,
  \item[Inverse] for every element $a\neq0$ in $R$ there is an element $a^{-1}$
  for which $a\cdot a^{-1}=1=a^{-1}\cdot a$.
\end{description}
then the ring is called a \defn{field}. Essentially, a field is a ring in which division
makes sense and can be defined by $a/b=a\cdot b^{-1}$ when $b\neq0$.

From the previous comments one can conclude that $\mathbb{Z}_n$ with addition and
multiplication modulo $n$ is a field if and only if $n$ is a prime number.
For example, since $5$ is a prime $\mathbb{Z}_5$ is a field.
This can be seen from its multiplication table as for $a\neq0$ and $b\neq0$
one has $a\cdot b\neq0$ in $\mathbb{Z}_5$:

\begin{tabular}{|c|ccccc|}\hline
  $+$ & $0$ & $1$ & $2$ & $3$ & $4$ \\ \hline
  $0$ & $0$ & $1$ & $2$ & $3$ & $4$ \\
  $1$ & $1$ & $2$ & $3$ & $4$ & $0$ \\
  $2$ & $2$ & $3$ & $4$ & $0$ & $1$ \\
  $3$ & $3$ & $4$ & $0$ & $1$ & $2$ \\
  $4$ & $4$ & $0$ & $1$ & $2$ & $3$ \\ \hline
\end{tabular}
\hfill
\begin{tabular}{|c|c|}\hline
  $a$ & $-a$ \\ \hline
  $0$ & $0$ \\
  $1$ & $4$ \\
  $2$ & $3$ \\
  $3$ & $2$ \\
  $4$ & $1$ \\ \hline
\end{tabular}
\hfill
\begin{tabular}{|c|ccccc|}\hline
  $\cdot$ & $0$ & $1$ & $2$ & $3$ & $4$ \\ \hline
  $0$ & $0$ & $0$ & $0$ & $0$ & $0$ \\
  $1$ & $0$ & $1$ & $2$ & $3$ & $4$ \\
  $2$ & $0$ & $2$ & $4$ & $1$ & $3$ \\
  $3$ & $0$ & $3$ & $1$ & $4$ & $2$ \\
  $4$ & $0$ & $4$ & $3$ & $2$ & $1$ \\ \hline
\end{tabular}
\hfill
\begin{tabular}{|c|c|}\hline
  $a$ & $a^{-1}$ \\ \hline
  $0$ & none \\
  $1$ & $1$ \\
  $2$ & $3$ \\
  $3$ & $2$ \\
  $4$ & $4$ \\ \hline
\end{tabular}

The inverse $a^{-1}$ can be found from the row of the multiplication table for the element
$a$ by locating the column that has the identity element $1$.
However, it can also be calculated directly in $\mathbb{Z}_p$
without first finding the multiplication
table by using the Euclidean algorithm.
Since $\mbox{gcd}(a,p)=1$, and the values $s$ and $t$ satisfy
$\mbox{gcd}(a,p)=s\cdot a+t\cdot p$, it follows that $s\mbox{ mod }p=a^{-1}$.
For example, ${12}^{-1}$ can be found in $\mathbb{Z}_{17}$ by:
\begin{eqnarray*}
  17 &=& 1\cdot12+5 \qquad\mbox{so $r_1=5$} \\
  12 &=& 2\cdot5+2 \qquad\mbox{so $r_2=2$} \\
  5 &=& 2\cdot2+1 \qquad\mbox{so $r_3=1$} \\
  2 &=& 2\cdot1+0 \qquad\mbox{so $r_4=0$}.
\end{eqnarray*}
Hence $\mbox{gcd}(12,17)=1$ as expected, and $s$ and $t$ are found by:
\begin{eqnarray*}
  1 &=& 5 + (-2)\cdot2 \\
  &=& 5 + (-2)\cdot(12+(-2)\cdot5) \\
  &=& 5\cdot5+(-2)\cdot12 \\
  &=& 5\cdot(17+(-1)\cdot12)+(-2)\cdot12 \\
  &=& (-7)\cdot12+5\cdot17.
\end{eqnarray*}
Hence ${12}^{-1}=(-7)\mbox{ mod }17=10$ in $\mathbb{Z}_{17}$
(and indeed $12\cdot10\mbox{ mod }17=1$).

Fields with a finite set of elements are important in cryptography,
particularly finite fields where there are $2^m$ elements rather than
$p$ elements for some prime $p$.
Fortunately, for every prime $p$ and positive integer $m$ there is a field denoted by
$GF(p^m)$ with $p^m$ elements,
and called the \defn{Galios field} of order $p^m$.

One of the most intuitive ways of understanding the addition $+$ and
multiplication $\cdot$ operations in the field $GF(p^m)$
is to consider each of the elements
$a=\left(a_0,a_1,a_2,\ldots,a_{m-1}\right)$ in $GF(p^m)$ as a polynomial
$f(x)=a_{m-1}x^{m-1}+\cdots+a_2x^2+a_1x+a_0$,
where $a_0$, $a_1$, \dots, $a_{m-1}$ are elements in $\mathbb{Z}_p$.
Then addition in $GF(p^m)$ is just the addition of two polynomials
whose coefficients are added together modulo $p$.
For example, in the field $GF(5^4)$ the polynomials
$f(x)=3x^3+1x^2+0x+4$ and $g(x)=4x^3+2x^2+3x+2$
can be considered as elements whose sum is
$f(x)+g(x)=2x^3+3x^2+3x+1$.

Multiplication in $GF(p^m)$ is slightly more complicated as the product of
two polynomials in $GF(p^m)$ might produce a polynomial of degree greater
than $m-1$, but simply discarding higher terms would not result in a field.
For example $\left(x^3+0x^2+0x+0\right)\cdot\left(x+0\right)=x^4$
which should not be considered the zero element $0x^3+0x^2+0x+0$ in $GF(5^4)$.
What is needed when a product results in a polynomial of degree greater than $m-1$
is to take the remainder when the product is divided by a
polynomial that acts much like a prime number.
A polynomial of degree $m$ is called \defn{reducible} in $GF(p^m)$ if it
factorizes as a product $f(x)\cdot g(x)$ of two
polynomials that are in $GF(p^m)$, otherwise it is called \defn{irreducible} in $GF(p^m)$.
For example, $x^4+x^2+1$ is reducible in $GF(5^4)$ as
$x^4+x^2+1=\left(x^2+x+1\right)\cdot\left(x^2+4x+1\right)$.
Similarly, $x^4$ is reducible as $x^4=x^3\cdot x$,
but $x^4+2$ can be shown to be irreducible in $GF(5^4)$.
Once an irreducible polynomial $q(x)$ has been found multiplication in $GF(p^m)$
can be defined by:
\begin{displaymath}
  \underbrace{f(x)\cdot g(x)}_{\mbox{mult in }GF(p^m)} =
  \underbrace{f(x)\cdot g(x)}_{\mbox{mult as polys}} \mbox{ mod } q(x).
\end{displaymath}
Choosing a different irreducible polynomial results in a slightly different
multiplication operation, but surprisingly gives essentially the same field
but with the elements permuted.
In fact it can be shown that the fields $GF(p^m)$
are the only finite fields that are possible
(mathematically, any finite field must be \defn{isomorphic} to $GF(p^m)$
for some prime $p$ and $m\geq1$, meaning that they are essentially the
same apart from what the elements are called).

Note that the field $\mathbb{Z}_p$ is just a special case of $GF(p^m)$
where $m=1$.
Also, inverses of elements in $GF(p^m)$ can be calculated directly using
a polynomial equivalent of the Euclidean Algorithm.

The field $GF(2^8)$ is often chosen for cryptography applications
as it has $2^8=256$ elements, one element for each possible byte value.
The elements can be considered as polynomials
$a_7x^7+a_6x^6+a_5x^5+a_4x^4+a_3x^3+a_2x^2+a_1x+a_0$ where
$a_0$,$a_1$,$a_2$,$a_3$,$a_4$,$a_5$,$a_6$,$a_7$ are elements of $\mathbb{Z}_2$.
Addition in $\mathbb{Z}_2$ is simply XOR so addition of two polynomials in
$GF(2^8)$ is simply just a bitwise XOR operation of the corresponding coefficients.
For example, adding together $x^7+x^4+x$ and $x^5+x^4+x+1$ results in
$x^7+x^5+1$.

It can be shown that there are 30 possible irreducible polynomials of degree $8$
in $GF(2^8)$, such as the polynomial $q(x)=x^8+x^4+x^3+x+1$
which is the polynomial chosen for the AES cipher (discussed in Section
\ref{Se:AdvancedEncryptionStandard}).
Using this irreducible polynomial, the product of $x^7+x^4+x$ and $x^5+x^4+x+1$
in $GF(2^8)$ is given by:
\begin{eqnarray*}
  \overbrace{\left(x^7+x^4+x\right)\cdot\left(x^5+x^4+x+1\right)}^{\mbox{mult in }GF(2^8)}
  &=&  \left( x^{12}+x^{11}+x^8+x^7\right. \\
  &&  \quad+x^9+x^8+x^5+x^4 \\
  &&  \quad\left.+x^6+x^5+x^2+x \right) \mbox{ mod } q(x) \\
  &=& \left( x^{12}+x^{11}+x^9+x^7 \right. \\
  &&  \quad\left.+x^6+x^4+x^2+x \right) \mbox{ mod } q(x) \\
  &=& x^7+x^4+x+1,
\end{eqnarray*}
where the remainder of dividing $x^{12}+x^{11}+x^9+x^7+x^6+x^4+x^2+x$
by $q(x)=x^8+x^4+x^3+x+1$ can be found using long division.


\begin{exercise}[The Finite Field $GF(2^2)$]
  Determine the addition, negative, multiplication, and inverse tables
  for the finite field $GF(2^2)$ which consists of the four elements
  $0$, $1$, $x$, $x+1$, using the irreducible polynomial
  $q(x)=x^2+x+1$ for the multiplication operation.
\end{exercise}


%%%%%%%%%%%%%%%%%%%%%%%%%%%%%%%%%%%%%%%%%%%%%%%%%%%%%%%%%%%%%%%%%%%%%%%%%%%%%%%%%%%%%%%%%
%%%%% Advanced Encryption Standard                                                  %%%%%
%%%%%%%%%%%%%%%%%%%%%%%%%%%%%%%%%%%%%%%%%%%%%%%%%%%%%%%%%%%%%%%%%%%%%%%%%%%%%%%%%%%%%%%%%

\section{Advanced Encryption Standard}\label{Se:AdvancedEncryptionStandard}

\reading{pp135-160}

As computing power increased the DES cipher with a key length of only 56 bits
became more and more vulnerable to a brute-force attack,
so a competition was held where new algorithms were suggested for its eventual
replacement. During the competition the cipher algorithms
could be openly scrutinized for susceptibility to cryptanalysis
attacks and their speed and ease of implementation compared.
In 2000 a symmetric block cipher algorithm known as \defn{Rijndael} was
selected as the new \defn{Advanced Encryption Standard} (AES) cipher.
This was considered a landmark for security as it demonstrated that
open scrutiny of an encryption algorithm could result in better security
than a secretly developed algorithm.

AES operates on blocks of length 128 bits (16 bytes), double that of DES,
although the Rijndael algorithm also allows for block sizes of 192 or 256 bits.
Keys can be of length 128, 192, or 256 bits.
It treats a 16-byte block of plaintext as a $4\times4$ matrix of bytes arranged
column by column,
so that the bytes \begin{code}73 65 63 75 72 65 20 61 6E 64 20 72 65 6C 69 61\end{code}
(for the ASCII string \begin{code}"secure and relia"\end{code})
are represented by the matrix:
\begin{displaymath}
\left(\begin{array}{cccc}
  \mbox{\begin{code}73\end{code}} & \mbox{\begin{code}72\end{code}}
    & \mbox{\begin{code}6E\end{code}} & \mbox{\begin{code}65\end{code}} \\
  \mbox{\begin{code}65\end{code}} & \mbox{\begin{code}65\end{code}}
    & \mbox{\begin{code}64\end{code}} & \mbox{\begin{code}6C\end{code}} \\
  \mbox{\begin{code}63\end{code}} & \mbox{\begin{code}20\end{code}}
    & \mbox{\begin{code}20\end{code}} & \mbox{\begin{code}69\end{code}} \\
  \mbox{\begin{code}75\end{code}} & \mbox{\begin{code}61\end{code}}
    & \mbox{\begin{code}72\end{code}} & \mbox{\begin{code}61\end{code}}
\end{array}\right).
\end{displaymath}
Firstly, the matrix is passed through a bitwise XOR with the 128-bit key $K$
(also arranged as a $4\times4$ byte matrix).
Then the matrix passes through 10 rounds, each consisting of three
operations (except the final tenth round which has just two operations),
each followed by a bitwise XOR using a key $K_i$ that is obtained
from the original key $K$ and which gets modified with each round.

The first operation in each round performs a simple substitution of each byte
in the matrix using an S-box:
\begin{center}
\nudgedown{1}
\begin{tabular}{|c|c|} \hline
  \multicolumn{2}{|l|}{\textit{AES S-box substitutions}} \\ \hline\hline
  \textit{Byte} & \textit{Substitution} \\ \hline
  \begin{code}00\end{code} & \begin{code}63\end{code} \\
  \begin{code}01\end{code} & \begin{code}7C\end{code} \\
  \begin{code}02\end{code} & \begin{code}77\end{code} \\
  \begin{code}03\end{code} & \begin{code}7B\end{code} \\
  \begin{code}04\end{code} & \begin{code}F2\end{code} \\
  \vdots                   & \vdots \\
  \begin{code}FF\end{code} & \begin{code}16\end{code} \\ \hline
\end{tabular}\end{center}
\nudgedown{1}

Unlike the obscure and secretive approach of DES S-boxes, it is well-known
how the AES S-box was designed. Firstly, each byte $a_7a_6a_5a_4a_3a_2a_1a_0$
is considered as a polynomial element
$a_7x^7+a_6x^6+a_5x^5+a_4x^4+a_3x^3+a_2x^2+a_1x+a_0$
in the finite field $GF(2^8)$. For all but the zero byte, its inverse
in $GF(2^8)$ is found, giving another byte $b_7b_6b_5b_4b_3b_2b_1b_0$ which
is then fed into the following equation using $\mathbb{Z}_2$ arithmetic
to give the S-box substitution byte:
\begin{displaymath}\begin{scriptsize}
  \left(\begin{array}{cccccccc}
    1&0&0&0&1&1&1&1\\
    1&1&0&0&0&1&1&1\\
    1&1&1&0&0&0&1&1\\
    1&1&1&1&0&0&0&1\\
    1&1&1&1&1&0&0&0\\
    0&1&1&1&1&1&0&0\\
    0&0&1&1&1&1&1&0\\
    0&0&0&1&1&1&1&1
  \end{array}\right)
  \left(\begin{array}{c}
    b_0\\b_1\\b_2\\b_3\\b_4\\b_5\\b_6\\b_7
  \end{array}\right)
  +
  \left(\begin{array}{c}
    1\\1\\0\\0\\0\\1\\1\\0
  \end{array}\right).
\end{scriptsize}\end{displaymath}
\begin{figure*}\begin{center}
\begin{pspicture}(-3.75,-1)(4.25,13.5)
  \rput(2,13){\rnode{AESin}{\scriptsize$\left(\!\!\begin{array}{cccc}
      p_1 & p_5 & p_9    & p_{13} \\
      p_2 & p_6 & p_{10} & p_{14} \\
      p_3 & p_7 & p_{11} & p_{15} \\
      p_4 & p_8 & p_{12} & p_{16}
    \end{array}\!\!\right)$}}

  \rput(2,11.5){\rnode{X0}{$\bigoplus$}}
  \rput(3.85,11.5){\rnode{K0}{$K$}}
  \ncline[arrows=->]{AESin}{X0}\naput{\scriptsize{128 bit}}
  \ncline[arrows=->]{K0}{X0}

  \rput(2,10.75){\rnode{S1}{\psframebox{\makebox[30mm]{S-box substitution}}}}
  \rput(2,10){\rnode{R1}{\psframebox{\makebox[30mm]{Rotate each row}}}}
  \rput(2,9.25){\rnode{M1}{\psframebox{\makebox[30mm]{Mix columns}}}}
  \rput(2,8.5){\rnode{X1}{$\bigoplus$}}
  \rput(3.85,8.5){\rnode{K1}{$K_1$}}
  \ncline[arrows=->]{X0}{S1}
  \ncline[arrows=->]{S1}{R1}
  \ncline[arrows=->]{R1}{M1}
  \ncline[arrows=->]{M1}{X1}
  \ncline[arrows=->]{K1}{X1}

  \pnode(2,7.95){S2}
  \rput(2,7.55){\rnode{DOTS}{\vdots}}
  \pnode(2,7.05){X8}
  \ncline[arrows=->]{X1}{S2}

  \rput(2,6.5){\rnode{S9}{\psframebox{\makebox[30mm]{S-box substitution}}}}
  \rput(2,5.75){\rnode{R9}{\psframebox{\makebox[30mm]{Rotate each row}}}}
  \rput(2,5){\rnode{M9}{\psframebox{\makebox[30mm]{Mix columns}}}}
  \rput(2,4.25){\rnode{X9}{$\bigoplus$}}
  \rput(3.85,4.25){\rnode{K9}{$K_9$}}
  \ncline[arrows=->]{X8}{S9}
  \ncline[arrows=->]{S9}{R9}
  \ncline[arrows=->]{R9}{M9}
  \ncline[arrows=->]{M9}{X9}
  \ncline[arrows=->]{K9}{X9}

  \rput(2,3.5){\rnode{S10}{\psframebox{\makebox[30mm]{S-box substitution}}}}
  \rput(2,2.75){\rnode{R10}{\psframebox{\makebox[30mm]{Rotate each row}}}}
  \rput(2,2){\rnode{X10}{$\bigoplus$}}
  \rput(3.85,2){\rnode{K10}{$K_{\!10}$}}
  \ncline[arrows=->]{X9}{S10}
  \ncline[arrows=->]{S10}{R10}
  \ncline[arrows=->]{R10}{X10}
  \ncline[arrows=->]{K10}{X10}

  \rput(2,0.5){\rnode{AESout}{\scriptsize$\left(\!\!\begin{array}{cccc}
      c_1 & c_5 & c_9    & c_{13} \\
      c_2 & c_6 & c_{10} & c_{14} \\
      c_3 & c_7 & c_{11} & c_{15} \\
      c_4 & c_8 & c_{12} & c_{16}
    \end{array}\!\!\right)$}}
  \ncline[arrows=->]{X10}{AESout}\naput{\scriptsize{128 bit}}

  \rput(2,-0.5){AES Encryption Algorithm}
\end{pspicture}
\end{center}\end{figure*}
The AES S-box gives an invertible permutation of the bytes designed to be
resistant to currently known cryptanalysis attacks and optimized to
ensure a low correlation between each byte and its substituted byte.
For performance, the S-box and its inverse (which is used for AES decryption)
are usually stored in memory rather than being recalculated each time
for the algorithm.

The second operation in each round rotates the bytes row by row in
the matrix. The first row is left unchanged, whereas the second row is rotated
one entry to the left, the third row by two entries to the left,
and the fourth row by three entries to the left, resulting in a transposition
of the bytes in the block:

\begin{displaymath}
  \left(\begin{array}{cccc}
    m_1 & m_5 & m_9    & m_{13} \\
    m_2 & m_6 & m_{10} & m_{14} \\
    m_3 & m_7 & m_{11} & m_{15} \\
    m_4 & m_8 & m_{12} & m_{16}
  \end{array}\right)
  \hfill \shortstack{Rotate\\ $\Longrightarrow$} \hfill
  \left(\begin{array}{cccc}
    m_1    & m_5    & m_9    & m_{13} \\
    m_6    & m_{10} & m_{14} & m_2 \\
    m_{11} & m_{15} & m_3    & m_7 \\
    m_{16} & m_4    & m_8    & m_{12}
  \end{array}\right).
\end{displaymath}

The third operation that is used in the first nine rounds multiplies
the matrix on the left by the invertible matrix:
\begin{displaymath}
  \left(\begin{array}{cccc}
    02 & 03 & 01 & 01 \\
    01 & 02 & 03 & 01 \\
    01 & 01 & 02 & 03 \\
    01 & 01 & 01 & 02
  \end{array}\right)
\end{displaymath}
where again each byte $a_7a_6a_5a_4a_3a_2a_1a_0$
is considered as a polynomial element
$a_7x^7+a_6x^6+a_5x^5+a_4x^4+a_3x^3+a_2x^2+a_1x+a_0$
in the finite field $GF(2^8)$ for the purpose of addition and
matrix multiplication (so $01$ is the constant polynomial $1$,
$02$ is the polynomial $x$, and $03$ is the polynomial $x+1$).
These particular entries in the matrix were chosen since
multiplication in $GF(2^8)$ by the element $02$ or $03$
can be implemented using at most a bit-shift and an XOR
(the entries in its inverse require more sophisticated
implementations, but in practice encryption is often performed more frequently
than decryption, such as when CFB or OFB cipher modes are used).
Also, the entries ensure that after a few rounds of row rotations and mix columns
all the output bits depend on all the input bits of the plaintext block.

If the key used for a round is the matrix:
\begin{displaymath}
  \left(\begin{array}{cccc}
    k_1 & k_5 & k_9    & k_{13} \\
    k_2 & k_6 & k_{10} & k_{14} \\
    k_3 & k_7 & k_{11} & k_{15} \\
    k_4 & k_8 & k_{12} & k_{16}
  \end{array}\right)
\end{displaymath}
(starting with the original key $K$) then the key used for the next round is given by:
\begin{displaymath}
  \left(\begin{array}{cccc}
    k_1^\prime & k_5^\prime & k_9^\prime    & k_{13}^\prime \\
    k_2^\prime & k_6^\prime & k_{10}^\prime & k_{14}^\prime \\
    k_3^\prime & k_7^\prime & k_{11}^\prime & k_{15}^\prime \\
    k_4^\prime & k_8^\prime & k_{12}^\prime & k_{16}^\prime
  \end{array}\right)
\end{displaymath}
where for $i\geq5$ the entries are given by
$k_i^\prime=k_{i-4}^\prime\oplus k_i$ (each entry is the XOR of the entry to its
left with the corresponding entry in the previous key).
For the first column of the new key a more complicated procedure is used to ensure
the key changes round by round:
\begin{eqnarray*}
  k_1^\prime &=& \mbox{Sub}(k_{14}) \oplus R \oplus k_1 \\
  k_2^\prime &=& \mbox{Sub}(k_{15}) \oplus R \oplus k_2 \\
  k_3^\prime &=& \mbox{Sub}(k_{16}) \oplus R \oplus k_3 \\
  k_4^\prime &=& \mbox{Sub}(k_{13}) \oplus R \oplus k_4 \\
\end{eqnarray*}
where $\mbox{Sub}$ denotes the S-box substitution and
$R$ is a byte that depends on the round.
\begin{figure*}[h]\begin{center}
\begin{tabular}{|l|c|c|c|c|c|c|c|c|c|c|} \hline
  \multicolumn{11}{|l|}{\textit{R byte values for calculating new key each round}}
    \\ \hline\hline
  \textit{Round} & 1 & 2 & 3 & 4 & 5 & 6 & 7 & 8 & 9 & 10 \\
  \textit{R byte} & \begin{code}01\end{code} & \begin{code}02\end{code}
    & \begin{code}04\end{code} & \begin{code}08\end{code}
    & \begin{code}10\end{code} & \begin{code}20\end{code}
    & \begin{code}40\end{code} & \begin{code}80\end{code}
    & \begin{code}1B\end{code} & \begin{code}36\end{code} \\ \hline
\end{tabular}
\end{center}\end{figure*}

Although the decryption algorithm is just the reverse of the encryption
algorithm, it must undo the operations in the reverse order to the
encryption algorithm, and so has a separate implementation
from encryption. However, one feature of AES is that by slightly changing
how the keys are calculated for each round during decryption the steps
can be reversed using the same order in which they were used for encryption,
enabling encryption and decryption to share much of their implementation
details (this is actually made possible by having the last round of
encryption not apply the mix columns operation).

The default JCA cryptography provider SunJCE supports AES, and the example from Section
\ref{Se:DataEncryptionStandard} can be modified to use AES simply by replacing
\begin{code}DES\end{code} by \begin{code}AES\end{code} in the
\begin{code}getInstance\end{code} method of \begin{code}KeyGenerator\end{code}
and of \begin{code}Cipher\end{code}.
As AES supports different key lengths some cryptographic security providers
might support 192-bit or 256-bit keys as well as the default 128-bit keys.
Different length keys can be produced by \begin{code}KeyGenerator\end{code}
by including the statement:
\begin{quote}\begin{code}\begin{verbatim}
kg.init(128);
\end{verbatim}\end{code}\end{quote}
before the key is generated.
The class \begin{code}AESExample\end{code} uses AES encryption
but in CBC mode rather than the default ECB mode, and so
requires an initialization vector.
An initialization vector is created using an \begin{code}IvParameterSpec\end{code}
and specifying a \begin{code}byte[]\end{code} array:
\begin{quote}\begin{code}\begin{verbatim}
byte[] ivBytes = ...; // array of 16 bytes
IvParameterSpec initVector = new IvParameterSpec(ivBytes);
\end{verbatim}\end{code}\end{quote}
The \begin{code}IvParameterSpec\end{code} can then be included as a parameter
to the \begin{code}Cipher\end{code} method \begin{code}init\end{code} whenever
the cipher is initialized for encryption or for
decryption:
\begin{quote}\begin{code}\begin{verbatim}
cipher.init(Cipher.ENCRYPT_MODE, key, initVector);
\end{verbatim}\end{code}\end{quote}
If an initialization vector is not specified to the \begin{code}Cipher\end{code}
method \begin{code}init\end{code} and it is being initialized for encryption
then it generates a suitable random vector itself, whose \begin{code}byte[]\end{code} array
can then be obtained from the \begin{code}Cipher\end{code}
method \begin{code}getIV\end{code}. If however it is being initialized for
decryption then it will throw an \begin{code}InvalidKeyException\end{code}
if an initialization vector is required but not specified.

\begin{figure*}\begin{program}\begin{verbatim}
/**
   A simple class that demonstrates how AES encryption with CBC mode
   is achieved using the Java Cryptography Architecture
   @author Andrew Ensor
*/
...
public class AESExample
{
   public static void main(String[] args)
   {  try
      {  // generate a secret key for AES
         KeyGenerator kg = KeyGenerator.getInstance("AES");
         kg.init(128); // 128-bit key used for AES
         SecretKey key = kg.generateKey();
         // create a cipher
         Cipher cipher = Cipher.getInstance("AES/CBC/PKCS5Padding");
         // obtain the plaintext
         ...
         // create an initialization vector (required for CBC)
         byte[] ivBytes = {51, 50, 7, -19, 120, 111, -110, 52, 9, -21,
            -6, -15, -95, 117, 36, -89}; // random array of 16 bytes
         IvParameterSpec initVector = new IvParameterSpec(ivBytes);
         // initialize cipher for encryption
         cipher.init(Cipher.ENCRYPT_MODE, key, initVector);
         // encrypt the plaintext
         byte[] ciphertext = cipher.doFinal(plaintext);
         ...
         // decrypt the ciphertext
         cipher.init(Cipher.DECRYPT_MODE, key, initVector);
         byte[] deciphertext = cipher.doFinal(ciphertext);
         System.out.println("Deciphered is:"+new String(deciphertext));
      }
      catch (NoSuchAlgorithmException e)
      {  System.err.println("Encryption algorithm not available: "+e);
      }
      catch (NoSuchPaddingException e)
      {  System.err.println("Padding scheme not available: "+e);
      }
      catch (InvalidKeyException e)
      {  System.err.println("Invalid key: "+e);
      }
      catch (InvalidAlgorithmParameterException e)
      {  System.err.println("Invalid algorithm parameter: "+e);
      }
      catch (IllegalBlockSizeException e)
      {  System.err.println("Cannot pad plaintext: "+e);
      }
      catch (BadPaddingException e)
      {  System.err.println("Exception with padding: "+e);
      }
   }
}
\end{verbatim}\end{program}\end{figure*}


\begin{exercise}[Performance of DES and AES]
Prepare a program that can be used to compare the performance of the DES cipher
with the AES cipher for different length plaintext and plot a graph of
time versus length.
\end{exercise}


%%%%%%%%%%%%%%%%%%%%%%%%%%%%%%%%%%%%%%%%%%%%%%%%%%%%%%%%%%%%%%%%%%%%%%%%%%%%%%%%%%%%%%%%%
%%%%% Other Symmetric Ciphers                                                       %%%%%
%%%%%%%%%%%%%%%%%%%%%%%%%%%%%%%%%%%%%%%%%%%%%%%%%%%%%%%%%%%%%%%%%%%%%%%%%%%%%%%%%%%%%%%%%

\section{Other Symmetric Ciphers}\label{Se:OtherSymmetricCiphers}

\reading{pp175-180,189-194}

The DES cipher is gradually being phased out and replaced by more secure ciphers
such as AES with a longer key and so less susceptible to brute-force attacks.
However, AES is a newer and completely different algorithm which has
not yet had the same amount of cryptanalysis testing as has DES.
Furthermore, there has been a significant software and hardware
investment in the DES algorithm. Until AES is fully accepted as the defacto symmetric block
cipher a popular alternative has been to apply the DES encryption several times
using different keys.

\noindent
\begin{minipage}{0.65\textwidth}
One attempt at improving DES might be to apply it twice, once to
the plaintext with one 56-bit key $K_1$ and then to that ciphertext $X$
with another 56-bit key $K_2$, essentially resulting in a 112-bit key.
This way a basic brute-force attack would take
$2^{56}$ times longer than a brute-force attack on single DES, making it
infeasible with current technology.
However, simply applying a block cipher such as DES twice can be susceptible
to the \defn{meet-in-the-middle} attack
if some blocks of plaintext and its corresponding ciphertext are known to a cryptanalyst.
\hfill This attack works on double DES by taking a block of plaintext
\end{minipage}\hfill
\begin{minipage}{0.25\textwidth}
%\hfill
\begin{pspicture}[shift=*](0,0.1)(4,4.6)
  \rput(2,4.5){\rnode{P}{\psframebox{\makebox[30mm]{plaintext}}}}
  \rput(2,3.5){\rnode{E1}{\psframebox[fillstyle=solid,fillcolor=yellow]{encrypt}}}
  \rput(3.8,3.5){\rnode{K1}{$K_1$}}
  \rput(2,2.5){\rnode{X}{\psframebox{\makebox[30mm]{$X$}}}}
  \rput(2,1.5){\rnode{E2}{\psframebox[fillstyle=solid,fillcolor=yellow]{encrypt}}}
  \rput(3.8,1.5){\rnode{K2}{$K_2$}}
  \rput(2,0.5){\rnode{C}{\psframebox{\makebox[30mm]{ciphertext}}}}

  \psset{arrows=->}
  \ncline{P}{E1}\ncline{K1}{E1}
  \ncline{E1}{X}
  \ncline{X}{E2}\ncline{K2}{E2}
  \ncline{E2}{C}
\end{pspicture}
\end{minipage}

and encrypting it
with all $2^{56}$ possible values for $K_1$, storing the resulting possible $X$ blocks
in a suitable data structure along with the key used.
It then takes the corresponding ciphertext block and decrypts it with all
$2^{56}$ possible values for $K_2$ and checks if each resulting decrypted block
is already in the data structure. Whenever the same value of $X$ is obtained
the keys used for encryption and for decryption give potential values for
$K_1$ and $K_2$. Since this could happen various times by chance
the potential values for $K_1$ and $K_2$ are then checked with another known plaintext
and ciphertext.

\begin{minipage}{0.65\textwidth}
Because of the meet-in-the-middle attack three stages are instead used with DES.
\defn{Triple DES} with two keys $K_1$ and $K_2$ (112-bit key) is widely used, where $K_1$ is
used in the first stage to encrypt the plaintext, which then gets decrypted but by
using $K_2$ (which effectively scrambles it more), and then it is encrypted
using $K_1$ again in the third stage.
Decryption is chosen rather than encryption in the second stage
so that single DES encryption can be obtained (if required) by setting $K_1=K_2$.
Since there is a potential attack (albeit impractical as it requires $2^{56}$ plaintext and corresponding ciphertext
\end{minipage}\hfill%
\begin{minipage}{0.25\textwidth}
\begin{pspicture}[shift=*](0,0.1)(4,4.6)
  \rput(2,4.5){\rnode{P}{\psframebox{\makebox[30mm]{plaintext}}}}
  \rput(2,3.5){\rnode{E1}{\psframebox[fillstyle=solid,fillcolor=yellow]{encrypt}}}
  \rput(3.8,3.5){\rnode{K1}{$K_1$}}
  \rput(2,2.5){\rnode{E2}{\psframebox[fillstyle=solid,fillcolor=yellow]{decrypt}}}
  \rput(3.8,2.5){\rnode{K2}{$K_2$}}
  \rput(2,1.5){\rnode{E3}{\psframebox[fillstyle=solid,fillcolor=yellow]{encrypt}}}
  \rput(3.8,1.5){\rnode{K3}{$K_1$}}
  \rput(2,0.5){\rnode{C}{\psframebox{\makebox[30mm]{ciphertext}}}}
  \psset{arrows=->}
  \ncline{P}{E1}\ncline{K1}{E1}
  \ncline{E1}{E2}\ncline{K2}{E2}
  \ncline{E2}{E3}\ncline{K3}{E3}
  \ncline{E3}{C}
\end{pspicture}
\end{minipage}
\noindent

blocks to be available to a cryptanalyst),
triple DES is now often used with three keys, where the third key is used in the
last stage rather than again using $K_1$.
However, the required 168-bit key is quite a bit longer than 128-bit keys used by
other algorithms presumed to be at least as secure, and DES is itself considered quite slow
so applying it three times over for each block makes the encryption process even
three times slower.

Another popular symmetric block cipher is \defn{Blowfish}, designed in 1993 as an
open and general-purpose replacement for DES. It uses block sizes of 64 bits and a key
length between 32 and 448 bits (with default 128-bit key). Its algorithm
has a similar structure to DES, also using 16 rounds with two 32-bit halves,
but it uses pre-computed key-dependent S-boxes. This makes the algorithm quite fast
when the same key can be reused multiple times.
Although there is no publicly-known successful attack on Blowfish, block ciphers
that use larger blocks than the 64-bit blocks used by DES, triple DES,
and Blowfish are generally now preferred.
As a consequence, Blowfish evolved to become \defn{Twofish},
which uses 128-bit blocks (the same as the AES Rijndael algorithm),
and key length between 128 and 256 bits. On most platforms Twofish is slightly
slower than the Rijndael algorithm for 128-bit keys but is faster for 256-bit keys.

JCA supports triple DES with keys of length 112 (two key triple DES) and
168 (three key triple DES) using the algorithm name \begin{code}DESede\end{code}.
It also supports Blowfish with the algorithm name \begin{code}Blowfish\end{code}.

The most popular stream cipher in use is the \defn{Rivest Cipher 4} (RC4) cipher.
It has a remarkably fast and simple algorithm that can use a key length anywhere
between 8 and 2048 bits (1 to 256 bytes), although at the very minimum 40 bits
(5 bytes) should be used to produce a \defn{key stream} of bytes.
To encrypt the plaintext bytes, they are passed through a bitwise XOR with
the key stream bytes.
The encrypted bytes can then be decrypted by passing them again through a bitwise XOR
with the same key stream to recover the plaintext bytes.
The security of RC4 relies on the key stream bytes being unpredictable without
knowledge of the key that produced the stream.
RC4 starts by initializing a state array $S$, which is a \begin{code}byte[]\end{code}
array of length $256$ that holds some unpredictable permutation of the bytes
$0$, $1$, $2$, \dots, $255$. The initial permutation of the bytes is determined
by a key $K$.
\begin{figure*}[htb]
\begin{codebox}
\Procname{$\proc{RC4-Initialization}(K)$}
\li \Comment initialize the state array $S$ to contain the bytes $0,1,2,\dots,255$
\li \For $i \gets 0$ \To $255$ \Do
\li    $S[i] \gets i$
    \End
\li \Comment use key $K$ to permute the bytes in $S$
\li $m \gets \id{length}[K]$
\li $j \gets 0$
\li \For $i \gets 0$ \To $255$ \Do
\li    $j \gets (j + S[i] + K[i \mbox{ mod } m]) \mbox{ mod } 256$
\li    swap $S[i]$ and $S[j]$
    \End
\li \Comment initialize the index pointers $a$ and $b$
\li $a \gets 0$
\li $b \gets 0$
\end{codebox}

\begin{codebox}
\Procname{$\proc{RC4-GetNextByte}()$}
\li \Comment update the index pointers $a$ and $b$
\li $a \gets (a+1) \mbox{ mod } 256$
\li $b \gets (b+S[a]) \mbox{ mod } 256$
\li swap $S[a]$ and $S[b]$ \`\Comment further permute the bytes in $S$
\li $t \gets (S[a] + S[b]) \mbox{ mod } 256$ \`\Comment next index of $S$ to use
\li \Return $S[t]$ \`\Comment next byte to use in key stream
\end{codebox}
\end{figure*}
The array $S$ is used to determine which byte is next used in
the key stream. Instead of simply iterating through the bytes in $S$
using a single index variable $a$
(which would result in the key stream repeating itself after $256$ bytes),
another variable $b$ is also used to help determine which byte is next used
and to further permute the array $S$ with each byte generated.

Although RC4 is not considered to provide the same level of security
as ciphers such as AES, and the key stream it produces has been
shown to be biased in terms of certain sequences,
provided a reasonable length key (such as 128 bits)
is used it is considered reasonably secure for some practical uses.
Due to its encryption/decryption speed and its comparative simplicity,
the RC4 stream cipher was adopted for use in the
Secure Sockets Layer/Transport Layer Security (SSL/TLS) standards,
in the Wired Equivalent Privacy (WEP) protocol for 802.11 wireless network security,
and in the WiFi Protected Access (WPA) protocol.
However, its implementation in the WEP protocol used a flawed method to
generate keys, enabling the protocol to be compromised in 2001
by an attack due to Fluhrer, Mantin, and Shamir.

When an encryption technique, such as RC4, is performed by passing the plaintext through an XOR
with a key stream it is important that different keys be used to encrypt each plaintext.
Otherwise a cryptanalyst could perform a bitwise XOR of two encrypted messages to obtain
an XOR of the two original plaintext, and so obtain information where the plaintext differ
from each other.

For convenience, JCA offers the classes \begin{code}CipherInputStream\end{code}
and \begin{code}CipherOut\-putStream\end{code} that act as filtering streams,
providing encryption of a data stream.
Both filter streams accept an initialized \begin{code}Cipher\end{code} object passed
to their constructor. Interestingly, the \begin{code}Cipher\end{code} can itself
be a block cipher such as DES or AES.
If a block cipher is used with \begin{code}CipherOutputStream\end{code}
one should ensure that the \begin{code}CipherOutputStream\end{code} method
\begin{code}close\end{code} is called (rather than just \begin{code}flush\end{code}) when the
bytes have all been written to the output stream, since if the total number of bytes
written is not an exact multiple of the block size the \begin{code}flush\end{code}
method does not pad the last block so the block is still not written out.

The class \begin{code}RC4Example\end{code} uses JCA cipher streams
with an RC4 stream cipher given the name
\begin{code}ARCFOUR\end{code} (for Alleged-RC4 as RSA never officially
acknowledged that this is the true RC4 algorithm).
This example also demonstrates how a desired key can be created
using the \begin{code}SecretKeySpec\end{code} class rather than randomly
via a \begin{code}KeyGenerator\end{code}. Note however that
secret information such as the key should \underline{never} in practice
be stored in a \begin{code}String\end{code} (as any \begin{code}String\end{code}
is immutable its contents cannot be changed after use).
\begin{figure*}\begin{program}\begin{verbatim}
/**
   A simple class that demonstrates how RC4 encryption is achieved
   using the Java Cryptography Architecture
   @author Andrew Ensor
*/
...
public class RC4Example
{
   public static void main(String[] args)
   {  try
      {  // create a secret key for RC4 (using bytes of "Secret")
         // Note in practice information related to the key such as in
         // the next line should NEVER be stored in a String
         SecretKey key = new SecretKeySpec("Secret".getBytes(),
            "ARCFOUR");
         // create a cipher
         Cipher cipher = Cipher.getInstance("ARCFOUR");
         // obtain the plaintext
         Scanner keyboardInput = new Scanner(System.in);
         System.out.print("Please enter some plaintext:");
         byte[] plaintext = keyboardInput.nextLine().getBytes();
         // initialize cipher for encryption
         cipher.init(Cipher.ENCRYPT_MODE, key);
         // encrypt the plaintext and store in a List
         CipherInputStream cis = new CipherInputStream
            (new ByteArrayInputStream(plaintext), cipher);
         List<Integer> ciphertext = new ArrayList<Integer>();
         System.out.print("Ciphertext is:");
         int unsignedByte = cis.read();
         while (unsignedByte != -1)
         {  ciphertext.add(unsignedByte);
            System.out.print(Integer.toHexString(unsignedByte)+" ");
            unsignedByte = cis.read();
         }
         System.out.println();
         cis.close();
         // initialize cipher for decryption
         cipher.init(Cipher.DECRYPT_MODE, key);
         // decrypt the ciphertext and store in ByteArrayOutputStream
         ByteArrayOutputStream baos = new ByteArrayOutputStream();
         CipherOutputStream cos = new CipherOutputStream(baos, cipher);
         for (int data : ciphertext)
            cos.write(data);
         cos.close(); // don't just flush if using a block cipher
         byte[] deciphertext = baos.toByteArray();
         System.out.println("Deciphered is:"+new String(deciphertext));
      }
      ...
   }
}
\end{verbatim}\end{program}\end{figure*}

The .NET platform provides implementations of some common
cipher algorithms in the 
\begin{code}System.\\Security.Cryptography\end{code}
namespace, including DES with the class \begin{code}DESCryptoServiceProvi-\\der\end{code},
AES with the class \begin{code}RijndaelManaged\end{code}
(a managed wrapper on top of the Windows CAPI service),
and triple DES with the class \begin{code}TripleDESCryptoServiceProvider\end{code}.
There is no block mode of encryption or decryption in C\#,
all encryption is performed using streams with a \begin{code}CryptoStream\end{code}
filtering stream, even when using a block cipher.
When a \begin{code}CryptoStream\end{code} is created it is provided with an
\begin{code}ICryptoTransform\end{code} obtained from the cipher
that can either encrypt or else decrypt, and the \begin{code}CryptoStream\end{code}
is created to either read from an input stream or else write to an output stream:
\begin{quote}\begin{code}\begin{verbatim}
Stream outputStream = ...;
CryptoStream cs = new CryptoStream(outputStream,
   cipher.CreateEncryptor(), CryptoStreamMode.Write);
\end{verbatim}\end{code}\end{quote}
The C\# class \begin{code}NETExample\end{code} demonstrates
how AES encryption can be performed in C\#.
This example creates a \begin{code}RijndaelManaged\end{code} (AES) cipher
which itself generates a random 128-bit key and initialization vector.
The key and initialization vector are held by the
\begin{code}Key\end{code} and \begin{code}IV\end{code} properties of the cipher,
and get stored in variables in the example so that they can be specified when
the ciphertext is later decrypted.
\begin{figure*}\begin{program}\begin{verbatim}
/*
   A C# class that demonstrates how AES encryption is achieved in .NET
*/
using System;
using System.Collections;
using System.IO;
using System.Security.Cryptography;
using System.Text;

public class NETExample
{
   public static void Main(string[] args)
   {  // create UTF8 encoder for converting user string to/from byte[]
      UTF8Encoding encoder = new UTF8Encoding();
      try
      {  // create a cipher and have it generate a secret key
         RijndaelManaged cipher = new RijndaelManaged();
         byte[] key = cipher.Key;
         byte[] initVector = cipher.IV;
         // obtain the plaintext
         Console.Write("Please enter some plaintext:");
         byte[] plaintext = encoder.GetBytes(Console.ReadLine());
         // encrypt the plaintext and store in a List
         CryptoStream cis = new CryptoStream(
            new MemoryStream(plaintext), cipher.CreateEncryptor(),
            CryptoStreamMode.Read);
         IList ciphertext = new ArrayList();
         Console.WriteLine("Ciphertext is:");
         int unsignedByte = cis.ReadByte();
         while (unsignedByte != -1)
         {  ciphertext.Add(unsignedByte);
            Console.Write("{0:X} ",unsignedByte);
            unsignedByte = cis.ReadByte();
         }
         Console.WriteLine();
         cis.Close();
         // decrypt the ciphertext and store in MemoryStream
         MemoryStream ms = new MemoryStream(100);
         CryptoStream cos = new CryptoStream(ms,
            cipher.CreateDecryptor(key, initVector),
            CryptoStreamMode.Write);
         foreach (int data in ciphertext)
            cos.WriteByte((byte)data);
         cos.Close();
         byte[] deciphertext = ms.GetBuffer();
         Console.WriteLine("Deciphered is:"
            + encoder.GetString(plaintext));
      }
      catch (CryptographicException e)
      {  Console.WriteLine("Cryptographic exception: "+e);
      }
   }
}
\end{verbatim}\end{program}\end{figure*}


\begin{exercise}[Implementation of RC4]
Prepare a program that provides an implementation of the RC4 stream cipher
(note that the \begin{code}\%\end{code} modulo operation only performs as
expected for positive values, so before performing \begin{code}x \% 256\end{code}
repeatedly add $256$ to \begin{code}x\end{code} to ensure that it is not negative).
\end{exercise}


%%%%%%%%%%%%%%%%%%%%%%%%%%%%%%%%%%%%%%%%%%%%%%%%%%%%%%%%%%%%%%%%%%%%%%%%%%%%%%%%%%%%%%%%%
%%%%% Secret Key Distribution                                                       %%%%%
%%%%%%%%%%%%%%%%%%%%%%%%%%%%%%%%%%%%%%%%%%%%%%%%%%%%%%%%%%%%%%%%%%%%%%%%%%%%%%%%%%%%%%%%%

\section{Secret Key Distribution}\label{Se:SecretKeyDistribution}

\reading{pp200-227}

For a distributed computer system there are two common alternative approaches
for encrypting messages to ensure their confidentiality in the network:
\begin{description}
  \item[link encryption] where every vulnerable communication link has an
  encryption device at each end, encrypting each message before it gets
  transmitted along the link and decrypting it at the other end,
  \item[end-to-end encryption] where the sender of the message encrypts the
  message and the destination decrypts it when it is received.
\end{description}
Link encryption might result in a message being encrypted and decrypted
various times if it passes along several vulnerable links, however the
encryption process is often performed by hardware devices which can help
reduce transmission delays.
These devices might also perform \defn{traffic padding}, transmitting
random messages when the link is not being used to confuse attempts that try
to deduce when the link is transmitting important messages.
Besides the transmission delays another drawback to using link encryption
is that it requires a large number of encryption devices
covering every possible message path, which might be infeasible in a large network
(such as the Internet). Also, if one of the devices could be compromised
the plaintext messages could be obtained.
End-to-end encryption would typically be performed either at the network layer
(encrypting the contents of IP packets) or else at the application layer
(by application software). This approach ensures that user messages
are secure even across insecure links, but since the messages must be forwarded
by routers around the network the sender and recipient of each message
remain exposed.

In practice a combination of both link encryption and end-to-end encryption is
used to achieve greater security.
If a symmetric cipher is employed in either approach then the encrypting end
and the decrypting end must share a secret key. Each encrypting/decrypting
pair should use its own key, so that if an attack successfully obtained one of the keys
only one part of the network would be compromised.
However, this creates the problem of how to distribute secret keys to each pair,
which in a distributed system with $n$ processes could require a separate key
for each of the $\frac{n(n-1)}{2}$ possible pairs of processes that might communicate
with each other.

Confidential information about a key must not be sent as unencrypted plaintext
across a network so keys must be physically delivered to the relevant processes
in some secure way. For link encryption this is a reasonable requirement as each
encryption device only needs keys to perform encryption with its neighbouring
devices, which could be provided when the communication link is installed.
But for end-to-end encryption the physical distribution of all required keys
become impractical given the possibly large number of processes with which a
process might communicate, adding a new process to the system would require
physical delivery of a new key to each of the existing processes,
hindering the system's scalability.

Instead of physically distributing each key required by a process,
a large distributed system can have one or more distinguished processes known as
a \defn{key distribution centre}. Each process is physically delivered a single
secret \defn{master key} which it can use to encrypt/decrypt messages to/from the
key distribution centre, which could be provided when the process is
first added to the distributed system. Thus the key distribution centre is
able to communicate securely with each process in the system
using the correct master key for symmetric encryption with that process.
More than one key distribution centre might be used to improve the reliability
of the system in case one centre fails or is compromised.

Rather than performing all secure communication via the key distribution centre,
which would quickly become a bottleneck in the system, when a process wants to
initiate secure communication with another process (receiver) it uses the following
steps:
\begin{enumerate}
  \item
    The initiator sends a request encrypted using its master key
    to the key distribution centre asking for a
    \defn{session key} for its communication with the receiver, including
    a chosen \defn{nonce value} that is not easily predicted.
\begin{center}
  \begin{pspicture}[shift=-1.5](0,0)(5,1.5)
    \rput(0.8,0.25){\ovalnode{I}{\makebox[10mm]{\scriptsize Initiator}}}
    \rput(2.5,1.25){\ovalnode{K}{\makebox[10mm]{\scriptsize Key Dist.}}}
    \rput(4.2,0.25){\ovalnode{R}{\makebox[10mm]{\scriptsize Receiver}}}
    \ncline[arrows=->,linecolor=red,linewidth=1mm]{I}{K}
    \rput(1,0.85){\tiny\begin{tabular}{c}secure\\request\end{tabular}}
  \end{pspicture}\end{center}

  \item
    The key distribution centre sends an encrypted response back to the
    initiator with a suitable session key for it to use,
    as well as that session key and an identifier of the initiator encrypted using
    the master key for the receiver.
 \begin{center}
  \begin{pspicture}[shift=-1.5](0,0)(5,1.5)
    \rput(0.8,0.25){\ovalnode{I}{\makebox[10mm]{\scriptsize Initiator}}}
    \rput(2.5,1.25){\ovalnode{K}{\makebox[10mm]{\scriptsize Key Dist.}}}
    \rput(4.2,0.25){\ovalnode{R}{\makebox[10mm]{\scriptsize Receiver}}}
    \ncline[arrows=->,linecolor=red,linewidth=1mm]{K}{I}
    \rput(1.1,0.85){\tiny\begin{tabular}{c}secure\\reply\end{tabular}}
  \end{pspicture}
\end{center}
 
  This information for the receiver can only be successfully
  read by it but will be sent to it by the initiator to prove
  the initiator's identity. The key distribution centre also includes the
  same nonce in its response so that the initiator can match the response
  with its request and be sure that the message was not a \defn{replay attack},
  where an attacker has duplicated an earlier message
  (in an attempt to disrupt operation of the system or compromise a key).

  \item
    The initiator forwards the encrypted information it obtained to the receiver
    to request a secure connection with it.
    Since this information was encrypted by the key distribution
    centre for the receiver, only the receiver can decrypt it using its master key.
 \begin{center}
  \begin{pspicture}[shift=-1.5](0,0)(5,1.5)
    \rput(0.8,0.25){\ovalnode{I}{\makebox[10mm]{\scriptsize Initiator}}}
    \rput(2.5,1.25){\ovalnode{K}{\makebox[10mm]{\scriptsize Key Dist.}}}
    \rput(4.2,0.25){\ovalnode{R}{\makebox[10mm]{\scriptsize Receiver}}}
    \ncline[arrows=->]{I}{R}
    \rput(2.5,0.45){\tiny\begin{tabular}{c}insecure request\end{tabular}}
  \end{pspicture}
\end{center}

  \item
    The receiver sends a secure response encrypted using the new session key
    to acknowledge the request, including a nonce value that it chose and
    asking the initiator to modify the nonce in some well-known way (such
    as adding one to it so it changes).
 \begin{center}
  \begin{pspicture}[shift=-1.5](0,0)(5,1.5)
    \rput(0.8,0.25){\ovalnode{I}{\makebox[10mm]{\scriptsize Initiator}}}
    \rput(2.5,1.25){\ovalnode{K}{\makebox[10mm]{\scriptsize Key Dist.}}}
    \rput(4.2,0.25){\ovalnode{R}{\makebox[10mm]{\scriptsize Receiver}}}
    \ncline[arrows=->,linecolor=red,linewidth=1mm]{R}{I}
    \rput(2.5,0.5){\tiny\begin{tabular}{c}secure reply\end{tabular}}
  \end{pspicture}
\end{center}

  \item
    The initiator acknowledges the response from the receiver by sending an
    acknowledgement message encrypted using the session key and containing
    the modified nonce.
    Once the receiver obtains its nonce value modified it knows the connection request was genuine and not
    just a replay attack on it.
 \begin{center}
 \begin{pspicture}[shift=-1.5](0,0)(5,1.5)
    \rput(0.8,0.25){\ovalnode{I}{\makebox[10mm]{\scriptsize Initiator}}}
    \rput(2.5,1.25){\ovalnode{K}{\makebox[10mm]{\scriptsize Key Dist.}}}
    \rput(4.2,0.25){\ovalnode{R}{\makebox[10mm]{\scriptsize Receiver}}}
    \ncline[arrows=->,linecolor=red,linewidth=1mm]{I}{R}
    \rput(2.5,0.5){\tiny\begin{tabular}{c}secure acknowledge\end{tabular}}
  \end{pspicture}
\end{center} 
\end{enumerate}

Note that a master key is only used by a process with the key distribution
centre for securely requesting and transmitting new session keys, it is not
used for encrypting large amounts of plaintext, which hinders potential attacks on
a master key that rely on a large number of encrypted blocks.
Session keys are used rather than master keys for all the encryption
of information between processes in the distributed system.

If there is a lot of communication between two processes they might
periodically request a new session key from the key distribution centre
so as to avoid using the same key for encrypting many plaintext blocks.
If two processes already have secure communication a new
session key can be arranged between them without having to involve
the key distribution centre again.
To ensure that a request by one process for a change in session key is genuine
(and not a replay attack of a previous request),
the initiator of the change can send a nonce value with its encrypted request.
The other process sends an encrypted response containing the new suggested
session key and a well-known modification to the nonce, along with its
own nonce value. When the initiator receives the response it
sends an encrypted acknowledgement with a well-known modification
to the other's nonce value, proving that it was a genuine request.

Secret keys for symmetric encryption and nonce values to foil replay attacks
are typically generated using pseudo-random number generators.
It is essential for good security that the sequence of values produced by a random number
generator not be easily predicted by a third party. For instance, in 1995 it was announced
that the SSL used by some popular browsers could be broken in minutes, not
because of the encryption algorithms used by SSL but due to the predictable
way the browsers generated random numbers.
The \begin{code}java.util\end{code} class \begin{code}Random\end{code}
uses a \defn{linear congruential} formula with a 48-bit seed to
generate values $x_0, x_1, x_2, \dots$, which are given by the
recurrence relation $x_{n+1}=\left(a x_n+c\right)\mbox{ mod }m$ for some
carefully chosen values of $a$, $c$, and $m$.
Although adequate for many purposes such techniques are not considered
\defn{cryptographically strong}, not meeting guidelines for sufficiently unpredictable
sequences of random numbers and possibly being vulnerable to cryptanalysis attacks.
Hence, the \begin{code}java.security\end{code} package includes the
\begin{code}SecureRandom\end{code} class which uses cryptographic
algorithms to encrypt a seed and extract a pseudo-random number from the
resulting ciphertext. A statement such as:
\begin{quote}\begin{code}\begin{verbatim}
SecureRandom generator = new SecureRandom();
\end{verbatim}\end{code}\end{quote}
uses a random number generator chosen by an available cryptographic security provider,
whereas a statement such as:
\begin{quote}\begin{code}\begin{verbatim}
SecureRandom generator=SecureRandom.getInstance("SHA1PRNG");
\end{verbatim}\end{code}\end{quote}
specifies the particular cryptographic algorithm that should be used by the provider.
The Java \begin{code}SecureRandom\end{code} class should always be used in place of
\begin{code}Random\end{code} for generating confidential random numbers.

It is usually impossible to rely on link encryption when using a mobile device
for communication. Even if the telecommunication provider employs link encryption
for communication within its own network HTTP requests still traverse the Web
where link encryption is not usually used. Hence end-to-end encryption is
particularly important. Android includes required support for HTTPS
(discussed in Section \ref{Se:WebSecurity}) which is adequate for most
mobile applications, but if a protocol such as SMS messaging is used instead of
HTTP or the Web server receiving the HTTP request is not trusted by the application,
then the application must provide its own encryption. Bouncy Castle is a very popular
open-source Java package available from \url{http://www.bouncycastle.org/}
that provides an alternative JCA cryptographic security provider.
It provides implementations of all the main cryptographic algorithms
as well as some newer algorithms. Android uses the JCE cryptographic API with the Bouncy Castle security provider,
although older versions of Android have a fairly incomplete implementation of the
Bouncy Castle API
(so the Android-specific \defn{Spongy Castle Cryptography API} might instead be
added as a more comprehensive cryptographic library for Android devices prior to version 4.0).
The Eclipse project \begin{code}AndroidAESDemo\end{code} demonstrates an Android
application that again encrypts SMS messages with AES encryption and CBC mode.
The activity \begin{code}AndroidAESDemo\-Activity\end{code} encrypts the contents of
each SMS message before it is sent, and encodes the resulting \begin{code}byte[]\end{code}
as a \begin{code}String\end{code} using Base 64 encoding
(discussed in Section \ref{Se:DigitalSignatures}) as Android only supports text content
in SMS messages,
whereas the broadcast receiver \begin{code}SMSBroadcastReceiver\end{code}
decodes a \begin{code}String\end{code} SMS message back to a \begin{code}byte[]\end{code}
and decrypts it to retrieve the plaintext message contents.
\begin{figure*}\begin{program}\begin{verbatim}
/**
 * Android Activity that sends SMS messages which it has encrypted
 * using AES and Base64 encoded as a String
 * Note this example uses a hard-coded secret key for simplicity
 * @author Andrew Ensor
 */
package aut.hss;
...
public class AndroidAESDemoActivity extends Activity implements
   OnClickListener
{
   private Button sendButton;
   private SMSSentBroadcastReceiver sentBroadcastReceiver;
   private SMSDeliveredBroadcastReceiver deliveredBroadcastReceiver;
   private byte[] keyBytes = { 51, 50, 7, -19, 120, 111, -110, 52, 9,
      -21, -6, -15, -95, 117, 36, -89 }; // the secret key
   private byte[] ivBytes = { 1, -2, 3, -4, 5, -6, 7, -8, 9,
      -10, 11, -12, 13, -14, 15, -16 }; // random array of 16 bytes
   private SecretKeySpec key;
   private IvParameterSpec initVector;
   private final String SMS_SENT_ACTION = "SMS_SENT";
   private final String SMS_DELIVERED_ACTION = "SMS_DELIVERED";

   /** Called when the activity is first created. */
   @Override
   public void onCreate(Bundle savedInstanceState)
   {  super.onCreate(savedInstanceState);
      setContentView(R.layout.activity_main);
      // obtain reference to send button
      sendButton = (Button) findViewById(R.id.send_button);
      sendButton.setOnClickListener(this);
      key = new SecretKeySpec(keyBytes, "AES");
      initVector = new IvParameterSpec(ivBytes);
   }
   ...
   // implementation of OnClickListener method
   public void onClick(View view)
   {  if (view == sendButton)
      {  TextView numberTextView
            = (TextView)findViewById(R.id.number_text);
         String numberString = numberTextView.getText().toString();
         TextView messageTextView
            = (TextView)findViewById(R.id.message_text);
         String messageString = messageTextView.getText().toString();
         // send the sms message
         PendingIntent sentPendingIntent = PendingIntent.getBroadcast
            (this, 0, new Intent(SMS_SENT_ACTION), 0);
         PendingIntent deliveredPendingIntent
            = PendingIntent.getBroadcast(this, 0,
            new Intent(SMS_DELIVERED_ACTION), 0);
\end{verbatim}\hfill \emph{cont-}\end{program}\end{figure*}%
\begin{figure*}\begin{program}\emph{-cont}\begin{verbatim}
         SmsManager smsManager = SmsManager.getDefault();
         String sendString = encryptAndEncodeString(messageString);
         if (sendString != null)
            smsManager.sendTextMessage(numberString, null, sendString,
               sentPendingIntent, deliveredPendingIntent);
      }
   }

   private String encryptAndEncodeString(String message)
   {  String errorMessage = null;
      try
      {  // create a cipher
         Cipher cipher = Cipher.getInstance("AES/CBC/PKCS5Padding");
         // initialize cipher for encryption
         cipher.init(Cipher.ENCRYPT_MODE, key, initVector);
         // encrypt the plaintext
         byte[] plaintext = message.getBytes();
         byte[] ciphertext = cipher.doFinal(plaintext);
         // base 64 encode the ciphertext as a string
         String encodedString = Base64.encodeToString(ciphertext,
            Base64.DEFAULT);
         return encodedString;
      }
      ...
      Toast toast = Toast.makeText(this, errorMessage,
         Toast.LENGTH_SHORT);
      toast.show();
      return null;
   }
}
\end{verbatim}\end{program}\end{figure*}
\begin{figure*}\begin{program}\begin{verbatim}
/**
 * BroadcastReceiver that is notified when an SMS message is received
 * which it Base64 decodes and decrypts using AES with a hardcoded key
 * Note this BroadcastReceiver is statically registered in
 * AndroidManifest.xml and application requires permission
 * android.permission.RECEIVE_SMS
 * @see AndroidAESDemoActivity.java
 */
package aut.hss;
...
public class SMSBroadcastReceiver extends BroadcastReceiver
{
   private byte[] keyBytes = { 51, 50, 7, -19, 120, 111, -110, 52, 9,
      -21, -6, -15, -95, 117, 36, -89 }; // the secret key
   private byte[] ivBytes = { 1, -2, 3, -4, 5, -6, 7, -8, 9,
      -10, 11, -12, 13, -14, 15, -16 }; // random array of 16 bytes
   private SecretKeySpec key;
   private IvParameterSpec initVector;
\end{verbatim}\hfill \emph{cont-}\end{program}\end{figure*}%
\begin{figure*}\begin{program}\emph{-cont}\begin{verbatim}
   public SMSBroadcastReceiver()
   {  key = new SecretKeySpec(keyBytes, "AES");
      initVector = new IvParameterSpec(ivBytes);
   }

   public void onReceive(Context context, Intent intent)
   {  // obtain the SMS message
      Bundle bundle = intent.getExtras();
      if (bundle != null)
      {  Object[] pdus = (Object[]) bundle.get("pdus");
         StringBuilder stringBuilder = new StringBuilder();
         for (int i = 0; i < pdus.length; i++)
         {  SmsMessage message = SmsMessage.createFromPdu
               ((byte[]) pdus[i]);
            String senderAddress
               = message.getDisplayOriginatingAddress();
            String receivedString = message.getDisplayMessageBody();
            String messageString = decodeAndDecryptString(context,
               receivedString);
            stringBuilder.append("Received Encrypted SMS:\n");
            stringBuilder.append("  Sender: ").append(senderAddress);
            stringBuilder.append(" Message: ").append(messageString);
            stringBuilder.append("\n");
         }
         Toast toast = Toast.makeText(context,
            stringBuilder.toString(), Toast.LENGTH_SHORT);
         toast.show();
      }
      else
      {  Toast toast = Toast.makeText(context,
            "Error: no message data received", Toast.LENGTH_SHORT);
         toast.show();
      }
   }

   private String decodeAndDecryptString(Context context,
      String encodedString)
   {  String errorMessage = null;
      // base 64 decode the ciphertext as a byte[]
      byte[] ciphertext = Base64.decode(encodedString, Base64.DEFAULT);
      try
      {  // create a cipher
         Cipher cipher = Cipher.getInstance("AES/CBC/PKCS5Padding");
         // initialize cipher for encryption
         cipher.init(Cipher.DECRYPT_MODE, key, initVector);
         // decrypt the ciphertext
         byte[] deciphertext = cipher.doFinal(ciphertext);
         return new String(deciphertext);
      }
      ...
   }
}
\end{verbatim}\end{program}\end{figure*}


\begin{exercise}[AES Encryption on a Mobile]
Enhance one of the AES mobile examples so that the sender and receiver of SMS messages
can themselves enter a string that is used to determine the 128-bit key for AES encryption
and decryption.
\end{exercise}
