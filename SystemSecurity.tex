% This is the LaTeX file for the chapter:
%   System Security
% for the manual:
%   Highly Secure Systems
% Written by Andrew Ensor
% Last updated December 7, 2018.

\chapter{System Security}

%%%%%%%%%%%%%%%%%%%%%%%%%%%%%%%%%%%%%%%%%%%%%%%%%%%%%%%%%%%%%%%%%%%%%%%%%%%%%%%%%%%%%%%%%
%%%%% Intruders and Malicious Software                                              %%%%%
%%%%%%%%%%%%%%%%%%%%%%%%%%%%%%%%%%%%%%%%%%%%%%%%%%%%%%%%%%%%%%%%%%%%%%%%%%%%%%%%%%%%%%%%%

\section{Intruders and Malicious Software}
An \defn{intruder} is an individual who gains unauthorized access to a computer system,
either by exploiting the account of a legitimate user (a \defn{masquerader}), being
a legitimate user who accesses the system in a way that is unauthorized
(a \defn{misfeasor}), or else by obtaining some type of administration access to the
system which can be used to evade detection (a \defn{clandestine user}).

Some systems might include a \defn{back door} which provides a (non-obvious) way to bypass
the normal authentication requirements of the system.
Back doors are sometimes included by a programmer during development and testing of a system,
either to speed up testing or else to provide access in case there might be a problem with
the authentication process.
In other systems, back doors might occur due to subtle security flaws resulting from
a developer's failure to grasp complex security issues.
For instance, an intruder might gain access to security policy settings or
be able to forge a authentication ticket.
Back doors can also be set up by malicious software such as viruses or worms
when the access controls for a system have been compromised.
Instead, a malicious compiler could insert a back door into an application, which would be
difficult to detect as no evidence would appear in any source code of the application.

If a back door to a system cannot be found then the intruder must somehow gain access
via the authentication procedures of the system. Typically this is achieved by
the intruder obtaining the password of a legitimate user.
Attempts to guess a password during user authentication can be easily detected
by a system, which can foil a brute force attempt by insisting that each authentication
attempt take a certain time period, and/or disconnecting or temporarily invalidating
a user account after several failed attempts.
Although a system may try to prevent access to its files that store user passwords,
this cannot be relied on as a security lapse might make such a file available within
the system, an administrator with access to all files might use the same password
on another system which has already been compromised,
or a misfeasor might be able to impersonate software which is granted
read access to a password file for the purpose of authentication.
To prevent actual user passwords from becoming available to a potential intruder,
they are never stored as plaintext in a file. Instead, either only their hash codes
or some encrypted form of them (such as by password based encryption) is ever stored:

\begin{center}
\begin{tabular}{|l|l|l|} \hline
  \textit{Host} & \textit{User} & \textit{Password} \\ \hline\hline
  156.62.\%     & student       & \begin{code}6E41903A2A2A67A4\end{code} \\
  156.62.\%     & admin         & \begin{code}175EA45B7CDF2B4A\end{code} \\ \hline
\end{tabular}
\end{center}

\noindent
The security of the passwords in a password file can be improved by assigning a
random salt to each user, making the salts available to authentication software,
and using them in the calculation of the encrypted passwords.
Thus two users which happen to have the same password in a system would not
have the same encrypted form in the file, and if the password file were obtained
by an intruder any attempt to crack a password would have to focus on one user at a time.

One significant problem with password-based authentication is that many users
choose passwords that can be easily cracked by guessing, often choosing a password based
on their own or someone else's name, relevant personal information such as an address,
dictionary words, or simple variations of these using a mixture of upper and lower case.
If a copy of the password file for a system could be obtained then millions of
potential passwords could be checked by the intruder off-line. Studies
have shown that up to 25\% of user-chosen passwords can be cracked using
password lists commonly available to potential intruders.

Password cracking can be countered by having a system insist on \defn{password selection rules},
where a password selected by a user is scrutinized for its suitability before it
gets accepted. Some common rules are to insist that
every password be at least eight characters long (to counter a brute-force attack),
that it not appear in a common password list nor in a dictionary, include
a mix of upper and lower case and include at least one non-letter character.
To avoid storing an enormous list of non-allowable passwords a \defn{Bloom filter}
could be used, where several (short) hash functions are used to
calculate various hash codes for a password. Each hash function has an associated
hash table with a bit allocated for each of the possible hash codes it can produce.
A bit is set in each table corresponding to the hash code for each non-allowable password.
When a submitted password is scrutinized its hash codes are calculated, and if
the corresponding bit in every one of the tables is set then the password gets rejected.

The introduction of smart cards and/or biometric identification has greatly improved
the security of password-based authentication, reducing many of the security risks
associated with passwords and password files.
For instance a smart card can render itself invalid if incorrect pin are repeatedly provided,
it can ensure all communication with a host is confidential and authenticated,
and biometric information such as fingerprint or facial recognition can be
very difficult for an intruder to forge.

Although intrusion prevention is very important, a system should also have measures
in place for detecting intruders once they have successfully entered, and take
action to minimize the effects of the intrusion.
One technique for intrusion detection is to store some information about the
typical behaviour of each user, such as what applications and files are normally
used by that user, and when that user typically logs in and out of the system.
Then during each session the behaviour of a user is compared with their past
behaviour and anything unusual (such as after hours access) might indicate that
an intruder is masquerading as that user.
Another technique is to enforce rules on proper user behaviour where multiple unusual actions
might indicate an intruder, such as the copying of system files, attempts
to access the password file, or writing to another user's files.
Both these techniques rely on the system maintaining a log of significant
events, such as authentication and authorization requests, unusual network communication,
and access to/from outside the network.
However, intruders that gain administration access rights might be able to modify
the log files and hide their own actions.
A security-conscious system might include \defn{honeypots}, which are
machines or even entire fabricated subsystems designed to appear attractive
to intruders but which would be unknown to typical users,
luring intruders away from actual critical parts of the system and noting their
behaviour while countermeasures are taken against them.

Unfortunately, probabilistic measures for determining intruders suffer from
\defn{base-rate fallacy}, where the number of false alarms can be unsatisfactorarily high.
Bayes' Theorem is a result from probability theory for mutually
exclusive events $E_1$, $E_2$, \dots, $E_n$, one of which must occur
(so the sum of their probabilities is $1$). For an arbitrary event $A$ it states that:
\begin{displaymath}
  \mbox{Prob}(E_i \mbox{ given } A) =
  \frac{\mbox{Prob}(A \mbox{ given } E_i)\cdot\mbox{Prob}(E_i)}
  {\sum_{j=1}^{n}\mbox{Prob}(A \mbox{ given } E_j)\cdot\mbox{Prob}(E_j)}.
\end{displaymath}
To illustrate base-rate fallacy suppose a test is devised for detecting
intruders, where an actual intruder is correctly detected by the test $95\%$ of the
time, so $\mbox{Prob}(\mbox{detected given intruder})=0.95$,
and a legitimate user is also correctly detected $95\%$ of the time.
Suppose the proportion of intruders compared to legitimate users is low,
such as only $2\%$ of users are actually intruders.
Bayes' Theorem can be used to give the probability that a user which has been detected
as being an intruder is actually legitimate:

\begin{scriptsize}\begin{eqnarray*}
  \lefteqn{\mbox{Prob}(\mbox{legitimate given detected})} \\
  &=& \frac{\mbox{Prob}(\mbox{detected given legitimate})\cdot\mbox{Prob}(\mbox{legitimate})}
  {\mbox{Prob}(\mbox{detected given legit} )\cdot\mbox{Prob}(\mbox{legit}) +
  \mbox{Prob}(\mbox{detected given intruder} )\cdot\mbox{Prob}(\mbox{intruder})} \\
  &=& \frac{0.05\times0.98}{0.05\times0.98+0.95\times0.02} \\
  &=& 0.7205,
\end{eqnarray*}\end{scriptsize}
so about $72\%$ of the detected intruders are actually legitimate users.

Rather than intruders themselves trying to gain access into a system they
might instead try to introduce malicious software.
Some categories of malicious software are:
\begin{description}
  \item[Logic bomb] which contains code set to activate when certain conditions are
  met, such as at a particular time or when a certain application is utilized.
  \item[Trojan horse] which is software that appears useful to a user but which
  contains some hidden code that performs an unexpected function.
  \item[Key logger] which captures and stores keystrokes that are made on a machine,
  enabling an attacker to catch the password and confidential details typed by a user.
  \item[Zombie] which is software that secretly takes over a machine on a network,
  and which can be used by an attacker to launch attacks on other machines.
  Zombies are used by attackers to make it more difficult to trace their
  actual location and to have multiple machines at their disposal for attacking a target.
  To find a new zombie an attacker might scan a network for vulnerable machines or else
  use a known hit-list of poorly protected sites.
  \item[Virus] which is software that embeds itself in an executable program
  so that whenever the program is executed by a user the virus code is also performed.
  A virus might be introduced into a system and mistakenly executed by an
  application that automatically executes macro commands attached to an e-mail or spreadsheet.
  It propagates itself by locating some accessible executable programs on the user's
  account and tries to rewrite them by adding its own commands to the start
  of the executable file, thus infecting further files and possibly
  lodging itself inside operating system software.
  \item[Worm] which can replicate itself automatically (without requiring any action
  to be performed by a user) and send copies of itself over a network.
  A worm actively seeks out more machines to infect (for example, by using a
  user's e-mail contact list or by randomly generating IP addresses),
  and might replicate itself using e-mail
  or via some remote execution or remote login capability of the system.
  It might disguise itself as a system process running in the background,
  actively try to crack passwords, communicate with the command interpreter of the
  operating system, download further versions of itself,
  or exploit flaws in system software
  (for instance, several worms have very successfully exploited security flaws
  in Microsoft Internet Information Server software).
  Worms are often used to install back doors on machines, so that they can be used as zombies
  for sending spam or to exploit a possible buffer overflow vulnerability of the operating system.
\end{description}
In a \defn{distributed denial of service} (DDOS) attack many zombies
are used to swamp a target machine with network communication in order to
disrupt the functionality of the target and its availability to legitimate clients.
The communication might be TCP/IP synchronization/initialization packets with false IP return
address, or instead ECHO packets sent to innocent \defn{reflectors} giving the target machine
as the return address, so each reflector then sends an ECHO response
to the target. To counter such attacks a potential target needs to have backup
resources available in the event of an attack, be able to quickly identify a
potential DDOS attack, and filter received communication to minimize its effects.

Early antivirus software tried to detect malicious software by scanning for occurrences
of known virus code, but virus and worms were quickly developed by attackers that
modified themselves to evade detection, such as by randomly changing some code or else
encrypting parts of themselves using a randomly selected key (thus making
each virus unique and potentially mutating with each execution).
Some viruses compressed the original executable file before they embedded themselves,
decompressing the file again when executed, thus making it impossible to tell from
file sizes whether or not a file was infected.
Hence antivirus software was developed which could scan executable files and inspect
their code, such as checking whether they started by performing some
decryption or decompression, and if so find the key embedded within the malicious software
and expose the virus code.
Another approach was to calculate a hash code for each executable file and securely store
it, or use a digital signature, so that modified files could be identified.
Modern antivirus software can track the actions being performed by
an executable program as it runs to catch suspicious activity (called a \defn{activity trap}).
Instead, an executable might be initially run using a software CPU emulator when started
to try catching any virus before it gets a chance to cause damage to the system.
The longer the executable is left in the emulator the greater the chance that
virus activity gets detected, but the slower the execution of valid code.
Such features are sometimes incorporated into distributed system software,
with antivirus software running in the background on each machine to detect suspicious
activities and unexpected background processes, monitors checking network traffic,
and dedicated machines that are passed suspicious software for analysis and which
alert the entire system when new malicious code is found.


\begin{exercise}[Determining Software Emulation]
Determine a way which a virus might be able to decide whether it was being executed
on a software CPU emulator so that it could disguise itself, and a counter measure that
could be taken by the emulator to detect such a disguise.
\end{exercise}


%%%%%%%%%%%%%%%%%%%%%%%%%%%%%%%%%%%%%%%%%%%%%%%%%%%%%%%%%%%%%%%%%%%%%%%%%%%%%%%%%%%%%%%%%
%%%%% Securing Code                                                                 %%%%%
%%%%%%%%%%%%%%%%%%%%%%%%%%%%%%%%%%%%%%%%%%%%%%%%%%%%%%%%%%%%%%%%%%%%%%%%%%%%%%%%%%%%%%%%%

\section{Securing Code}\label{Se:SecuringCode}
Besides digital signatures being used to authenticate users and messages, they are
also widely used to authenticate code. \defn{Code signing} ensures the
recipient of a JAR file (such as a Java application, MIDlet suite, or browser applet),
or of a C\# assembly that the code did originate from the signer
and it has not been tampered with enroute since the code was signed.
Code is signed by the sender using the private key of a key pair, and can be verified
by a recipient if the recipient possesses a trusted certificate for the corresponding public key.

A JAR file can be signed using the command line \begin{code}jarsigner\end{code} utility
(found in the \begin{code}bin\end{code} folder of the Java SDK), specifying the
keystore and passwords, an optional name for the signed JAR,
the name of the JAR file to sign, and the alias of the private key to use for signing:
\begin{quote}\begin{code}
jarsigner  -keypass \emph{mykeypw} -keystore \emph{mykeystore.jks} \\
\trind -storepass \emph{mystorepw} -signedjar \emph{SignedJARName.jar} \\
\trind \emph{JARName.jar} \emph{mykeyalias}
\end{code}\end{quote}
Most IDEs such as Eclipse and the Wireless Toolkit have menu options for
facilitating the signing of JAR files, as well as the management of certificates and key pairs.
When a signed JAR file is received the \begin{code}jarsigner\end{code} utility
can also be used to verify the signature, and optionally display the certificates that
formed the certificate chain validating the signature:
\begin{quote}\begin{code}
jarsigner -keystore \emph{mykeystore.jks} -storepass \emph{mystorepw} \\
\trind -certs -verify \emph{SignedJARName.jar}
\end{code}\end{quote}
Usually browsers and mobile devices have their own trusted certificates
that they use to verify the signature of code they obtain, so the verification is
performed automatically and the user notified as to whether the signature was verified.
It must however be remembered that code signing only guarantees authenticity that the code
is indeed from the signer, not that the code itself does not perform any malicious function,
so a valid signature should not considered a complete guarantee about the code.

.NET also supports code signing. The command line \begin{code}sn\end{code}
(Strong Name) utility can be used to generate a key pair and place the pair
in an (insecure) file:
\begin{quote}\begin{code}
sn -k \emph{myKeyPairFile.snk}
\end{code}\end{quote}
Then the public key can be extracted from the file:
\begin{quote}\begin{code}
sn -p \emph{myKeyPairFile.snk} \emph{myPublicKey.snk}
\end{code}\end{quote}
Once a key pair has been obtained the following attribute is placed in the
\begin{code}AssemblyInfo.cs\end{code} class for the project indicating that the assembly
should be signed:
\begin{quote}\begin{code}
[assembly: AssemblyKeyFile(@"\emph{myKeyPairFile.snk}")]
\end{code}\end{quote}
Alternatively, a .NET file such as single compiled C\# class can be signed or verified using the
\begin{code}signtool\end{code} utility.

Developers of code often want to protect their code against \defn{reverse engineering}.
Both Java and .NET come with command line \defn{disassembler} utilities,
\begin{code}javap\end{code} and \begin{code}ildasm\end{code} respectively,
which present a human-readable version of any compiled bytecode.
However, a \defn{decompiler} can instead be used on compiled code to reveal
much more of the structure of the original source code,
producing a functionally equivalent source file.
Most decompilers can obtain quite a bit of the structure of the source,
including the original variable and method names
(which are retained by a compiler in compiled classes),
and a lot of information about how the methods were programmed,
apart from comments which are always removed by a compiler.
This can potentially put the intellectual property of a programmer at risk,
as recipients of compiled code can themselves determine details of
how the programmer wrote the software.
To counter attempts at reverse engineering code, a developer can use a
\defn{code obfuscator}. An obfustactor transforms a compiled class by scrambling and
renaming variable, method and class names, replacing pieces of code with equivalent but
more obscure code, in-lining code fragments, and possibly optimizing control structures.

For example, using a common decompiler (called JAD) on the compiled class
\begin{code}XMLSignerVerifier\end{code} %from Section \ref{Se:XMLWebServiceSecurity}
results in a reasonably close copy of the original source file
(see the file \begin{code}XMLSignerVerifierRE\end{code}).
Passing the compiled class through an obfuscator (called ProGuard) results
in a functionally equivalent class file. Decompiling this
obfuscated version exposes less information about the source
(see the file \begin{code}XMLSignerVerifierOB\end{code}), in particular
the field and inner class names have been lost.
However, it is clear from this example that obfuscation does not totally thwart
attempts at reverse engineering,
although it can make decompiled code (particularly for a large project)
more difficult to understand. To fully protect
code from attempts at reverse engineering it can be encrypted,
which would protect its source from all but the intended recipient.

Whenever bytecode for a Java class is loaded into a virtual machine for
execution, or a C\# assembly is loaded into the .NET common language runtime,
the (Java or .NET) virtual machine first performs \defn{bytecode verification}.
The verification involves checking the code to ensure issues
such as that all variables have been initialized before they will be used,
method signatures (parameter and return types) are correctly used,
scope rules (private and protected modifiers) are correctly followed,
and memory for the program is correctly accessed.
Since a Java or C\# compiler has itself already checked the source code for such problems
it might seem strange that all compiled code is checked again just before the code gets executed.
The reason for this is that a skilled attacker could maliciously modify the code output
from a compiler (or use a malicious compiler)
and try to arrange for some operation that is unsafe for the Java virtual machine or .NET platform,
such as accessing a memory location outside the bounds of the program or causing
a runtime stack overflow.
Hence, verification prevents a tampered class file from damaging the runtime environment.
Note that for Java Standard Edition and .NET all bytecode is automatically verified,
whereas for Java Micro Edition it performed as a separate step before being installed on
a device so to reduce the burden on the device's virtual machine.

Once compiled code has been checked by a virtual machine and allowed to execute
its authorization to perform specific operations while running is controlled by a
\defn{security manager}, which uses a \defn{security policy} specifying
the types of operations for which the code is granted permission.
A security manager is represented in Java by the \begin{code}SecurityManager\end{code}
class from the \begin{code}java.lang\end{code} package, or its
subclass \begin{code}RMISecurityManager\end{code} which is required
if dynamic loading of class files is used for remote objects.
By default a Java application has no security manager installed, so that all
Java operations are permitted.
One can be installed and a security policy file specified either
with the following lines of code in the application:
\begin{quote}\begin{code}
System.setProperty("java.security.policy","\emph{MyApp.policy}"); \\
System.setSecurityManager(new SecurityManager());
\end{code}\end{quote}
or else when the application is executed:
\begin{quote}\begin{code}
java -Djava.security.manager \\
\trind -Djava.security.policy=\emph{MyApp.policy} \emph{MyApplication}
\end{code}\end{quote}
Besides a specified policy file there are standard locations where
a security manager checks for further policy files,
the \begin{code}java.policy\end{code} file in the \begin{code}lib/security\end{code}
folder of the Java runtime, or else in a file called \begin{code}.java.policy\end{code}
in the user's home directory (standard locations can be specified in the
\begin{code}java.security\end{code} properties file).
A security manager adds (the union of) all the permissions granted in any of these policy files,
unless \begin{code}==\end{code} is used in the command line in place of
\begin{code}=\end{code}, which instructs the security manager to only use the
specified policy file.
A system administrator can modify the \begin{code}java.security\end{code} properties file
to specify security policy files that cannot be edited by the user and that additional
policy files are not allowed to be specified by the user.
For comparison, .NET instead uses four policy files, \defn{enterprise}
(used by a system administrator), \defn{machine} (used by a machine administrator),
\defn{user} (which can be modified by the user), and \defn{application domain}
(for a particular application), and takes the intersection of all the permissions
granted. Hence, an operation is only permitted in .NET if permission for it is granted
in all the policy files. These policy files are modified through the .NET administrative tools.

Java and .NET each include permissions for accessing file systems, network communication,
the display, reflection (the ability to determine and create instances of a class given an
instance), accessing the system clipboard, controlling threads,
particular types of database access, and using a printer.
In Java the location of a key store (holding certificates)
can be given at the start of a security policy file before the first \begin{code}grant\end{code}:
\begin{quote}\begin{code}
keystore "file:\emph{mytruststore.jks}", "JKS" \\
grant signedBy \emph{alias} codeBase \emph{uri} Principal \emph{PrinClass} \emph{name} \\
\{\dbind permission \emph{SecurityPermissionClass} \emph{Action}; \\
\trind \vdots \\
\trind permission \emph{SecurityPermissionClass} \emph{Action}; \\
\}; \\
\vdots
\end{code}\end{quote}
A \begin{code}java.io.FilePermission\end{code} specifies a file target:
\begin{quote}
  \begin{code}\emph{file}\end{code} for the specified file, \\
  \begin{code}*\end{code} for all files in the current directory, \\
  \begin{code}-\end{code} for all files in the current directory or in any subdirectory of it, \\
  \begin{code}\emph{directory}/\end{code} for the specified directory, \\
  \begin{code}\emph{directory}/*\end{code} for all files in specified directory, \\
  \begin{code}\emph{directory}/-\end{code} for all files in specified directory,
  or in a subdirectory\\
  \begin{code}<<ALL FILES>>\end{code} for all files in the file system,
\end{quote}
followed by a comma and any combination of \begin{code}read\end{code},
\begin{code}write\end{code}, \begin{code}execute\end{code}, \begin{code}delete\end{code}.
A \begin{code}java.net.\\SocketPermission\end{code} specifies a host:
\begin{quote}
  \begin{code}\emph{hostname}\end{code} for the specified host or IP address, \\
  \begin{code}localhost\end{code} or \begin{code}""\end{code} for the local host, \\
  \begin{code}*.\emph{domainSuffix}\end{code} for any host that ends with the given suffix, \\
  \begin{code}*\end{code} for all hosts,
\end{quote}
followed by a port range:
\begin{quote}
  \begin{code}:\emph{n}\end{code} for just the specified single port, \\
  \begin{code}:\emph{n}-\end{code} for all ports numbered \begin{code}\emph{n}\end{code}
  and above, \\
  \begin{code}:-\emph{n}\end{code} for all ports numbered \begin{code}\emph{n}\end{code}
  and below, \\
  \begin{code}:\emph{n1}-\emph{n2}\end{code} for the specified range of ports,
\end{quote}
followed by a comma and any combination of \begin{code}read\end{code},
\begin{code}write\end{code}.
Custom security permission classes can be made by extending the
\begin{code}Permission\end{code} class or one of its subclasses from the
\begin{code}java.security\end{code} package, providing implementations
for the methods \begin{code}getActions\end{code}, \begin{code}equals\end{code},
\begin{code}hashCode\end{code}, \begin{code}implies\end{code}.

Applets that are loaded over a network by a browser use a security manager provided
by the browser and are usually assigned a very restricted security policy.
By default the security policy prevents applets from performing operations
which might put the client at risk, such as reading and writing files
on the client file system, and from making network connections to any host other than
the host from where the applet was obtained.
In addition, applets loaded over a network are prevented from starting other programs
on the client, they are not allowed to load libraries,
or to define native method calls (to prevent applets from obtaining direct
access to the computer).
Applets that need some of these permissions granted are typically signed and a
security policy file is provided to the browser.


\begin{exercise}[Granting Permission to Signed Code]
Prepare a GUI that contains a text field for entering the name of a file
and a text area for displaying the contents of the file
(ensuring the file reader catches a possible \begin{code}Security\-Exception\end{code}).
Then sign the GUI and create a suitable security permission file so that the GUI
can only read from the file system if it is signed using a key for which there
is a trusted certificate.
\end{exercise}


%%%%%%%%%%%%%%%%%%%%%%%%%%%%%%%%%%%%%%%%%%%%%%%%%%%%%%%%%%%%%%%%%%%%%%%%%%%%%%%%%%%%%%%%%
%%%%% Firewalls                                                                     %%%%%
%%%%%%%%%%%%%%%%%%%%%%%%%%%%%%%%%%%%%%%%%%%%%%%%%%%%%%%%%%%%%%%%%%%%%%%%%%%%%%%%%%%%%%%%%

\section{Firewalls}\label{Se:Firewalls}
A \defn{firewall} is hardware and/or software technology that is placed on the
perimeter of a network and which acts as a security barrier around the network,
examining and filtering all communication passing in and out of the network.
Firewalls are used to restrict the Internet services available to users and help protect
machines within the network from external attacks.
Filtering rules can restrict communication to and from certain IP addresses
or restrict traffic to only certain protocols as determined by the firewall administrator.
A firewall can be classified as follows:
\begin{description}
  \item[Packet filtering firewall] which filters packets at the network protocol level,
  checking each IP packet against a list of security criteria, such as source and destination
  IP addresses, port numbers, and the communication protocol.
  Packets that do not meet the criteria are either discarded or else result in a message
  back to the sender.
  Packet filtering is usually performed by a router, providing some basic but limited
  security without adversely affecting the performance of the network.

  \item[Circuit level gateway] which filters packets at the transport or session protocol layer,
  monitoring TCP handshaking to determine whether a requested TCP session is legitimate,
  such as ensuring that was initiated by a known IP address within the network,
  not externally.
  Rather than permitting a direct end-to-end TCP connection between a machine inside
  the network with one outside, a circuit level gateway instead sets up its own connection
  with either end and relays TCP segments between the two connections,
  performing \defn{network address translation} (NAT) between the private IP addresses
  within the network and temporarily assigned IP addresses.
  Any machine outside the network thus communicates with the gateway, so
  actual IP addresses of machines within the network are hidden.

  \item[Application level gateway (proxy)] which filters packets at the application
  protocol layer, only allowing application protocols for which there is a corresponding
  proxy in the firewall, such as the freely available Squid web proxy which only
  allows HTTP traffic and can inspect \begin{code}GET\end{code} and
  \begin{code}POST\end{code} requests.
  Application level proxies offer a high level of security as they can inspect
  the data being transmitted but have a significant
  impact on network performance.

  \item[Stateful multilayer inspection firewall] which combines aspects
  of the other three types of firewall, filtering packets at network layer,
  session layer, and using algorithms to process application layer data
  rather then using application-specific proxies.
  A stateful firewall holds attributes of each connection in a table so that
  packets for connections that have already been screened can be quickly processed
  (interestingly, the most common DDOS attack is currently a SYN flood which
  attempts to overflow this firewall table).
  It offers good security and performance but is more complex to administer.
\end{description}

In the \defn{screened host with single-homed bastion} firewall configuration
a packet filtering router is used to block all packets
except those that are to or from a secure \defn{bastion host} which performs
circuit level or application level filtering.
Optionally the router might also allow communication with some public
information servers (such as web servers).

\noindent\begin{center}
\begin{pspicture}(0,0.4)(12,3.4)
  \rput(11.5,2){\rnode{I}{Internet}}
  \rput(9.5,2){\dianode{Ro}{{\makebox[6.5mm]%
    {\scriptsize\begin{tabular}{c}Filtering\\router\end{tabular}}}}}
  \rput(6.5,3){\rnode{BH}{\psframebox{\makebox[14mm]%
    {\rule[-2mm]{0mm}{6mm}\scriptsize\begin{tabular}{c}Bastion\\host\end{tabular}}}}}
  \rput(5,1){\rnode{IS}{\psframebox{\makebox[14mm]%
    {\rule[-2mm]{0mm}{6mm}\scriptsize\begin{tabular}{c}Information\\Server\end{tabular}}}}}
  \rput(6.5,2){\pnode{N1}}
  \rput(5,2){\pnode{N2}}
  \rput(3.5,2){\pnode{N3}}\rput(3.5,2.5){\pnode{N3End}}
  \rput(2,2){\pnode{N4}}\rput(2,1.5){\pnode{N4End}}
  \rput(0.5,2){\pnode{NEnd}}
  \rput(2,2.3){Internal network}

  \ncline{Ro}{I}
  \ncline[linecolor=red,linewidth=1mm]{Ro}{NEnd}
  \ncline[linecolor=red,linewidth=1mm]{N1}{BH}
  \ncline[linecolor=red,linewidth=1mm]{N2}{IS}
  \ncline[linecolor=red,linewidth=1mm]{N3}{N3End}
  \ncline[linecolor=red,linewidth=1mm]{N4}{N4End}
\end{pspicture}\end{center}

\noindent
However, if the router were to be compromised by an attacker, then the machines on
the internal network would become vulnerable as packets could then be sent directly to them.
To counter this risk, the \defn{screened host with dual-homed bastion}
firewall configuration places the bastion host between the internal network
and the router, providing dual layer security, requiring all communication
to physically pass through both the router and the bastion host.

\noindent\begin{center}
\begin{pspicture}(0,0.4)(12,2.8)
  \rput(11.5,2){\rnode{I}{Internet}}
  \rput(9.5,2){\dianode{Ro}{{\makebox[6.5mm]%
    {\scriptsize\begin{tabular}{c}Filtering\\router\end{tabular}}}}}
  \rput(7,2){\rnode{BH}{\psframebox{\makebox[14mm]%
    {\rule[-2mm]{0mm}{6mm}\scriptsize\begin{tabular}{c}Bastion\\host\end{tabular}}}}}
  \rput(5,1){\rnode{IS}{\psframebox{\makebox[14mm]%
    {\rule[-2mm]{0mm}{6mm}\scriptsize\begin{tabular}{c}Information\\Server\end{tabular}}}}}
  \rput(5,2){\pnode{N2}}
  \rput(3.5,2){\pnode{N3}}\rput(3.5,2.5){\pnode{N3End}}
  \rput(2,2){\pnode{N4}}\rput(2,1.5){\pnode{N4End}}
  \rput(0.5,2){\pnode{NEnd}}
  \rput(2,2.3){Internal network}

  \ncline{Ro}{I}
  \ncline[linecolor=red,linewidth=1mm]{Ro}{BH}
  \ncline[linecolor=red,linewidth=1mm]{BH}{NEnd}
  \ncline[linecolor=red,linewidth=1mm]{N2}{IS}
  \ncline[linecolor=red,linewidth=1mm]{N3}{N3End}
  \ncline[linecolor=red,linewidth=1mm]{N4}{N4End}
\end{pspicture}\end{center}

\noindent
The \defn{screened subnet} firewall configuration provides a third security layer
by placing another packet filtering router between the public (screen subnet) part
of the network (consisting of the bastion host, public information servers, modems)
and the internal network.

\noindent\begin{center}

\begin{pspicture}(0,0.4)(12,2.8)
  \rput(11.5,2){\rnode{I}{Internet}}
  \rput(9.9,2){\dianode{Ro}{{\makebox[6.5mm]%
    {\scriptsize\begin{tabular}{c}Outside\\router\end{tabular}}}}}
  \rput(8,2){\rnode{BH}{\psframebox{\makebox[14mm]%
    {\rule[-2mm]{0mm}{6mm}\scriptsize\begin{tabular}{c}Bastion\\host\end{tabular}}}}}
  \rput(6.25,1){\rnode{IS}{\psframebox{\makebox[14mm]%
    {\rule[-2mm]{0mm}{6mm}\scriptsize\begin{tabular}{c}Information\\Server\end{tabular}}}}}
  \rput(6.25,2){\pnode{N2}}
  \rput(4.75,2){\dianode{Ro2}{{\makebox[6.5mm]%
    {\scriptsize\begin{tabular}{c}Inside\\router\end{tabular}}}}}
  \rput(3.5,2){\pnode{N3}}\rput(3.5,2.5){\pnode{N3End}}
  \rput(2,2){\pnode{N4}}\rput(2,1.5){\pnode{N4End}}
  \rput(0.5,2){\pnode{NEnd}}
  \rput(7.325,2.8){Screened subnet}
  \rput(2,2.3){Internal network}

  \ncline{Ro}{I}
  \ncline[linecolor=red,linewidth=1mm]{Ro}{BH}
  \ncline[linecolor=red,linewidth=1mm]{BH}{Ro2}
  \ncline[linecolor=red,linewidth=1mm]{N2}{IS}
  \ncline[linecolor=red,linewidth=1mm]{Ro2}{NEnd}
  \ncline[linecolor=red,linewidth=1mm]{N3}{N3End}
  \ncline[linecolor=red,linewidth=1mm]{N4}{N4End}
\end{pspicture}
\end{center}

\noindent The outside router only allows external access to the screened subnet,
and does not allow direct access to the internal network. Similarly, the
inside router only allows the internal network access to the screened subnet.

The \defn{Sockets protocol} (SOCKS) is an intermediate protocol between the
transport and the application protocol layers that enables a client
located behind a firewall to transparently communicate with an external server.
When a TCP or UDP client wants to communicate with a server located outside
of the network firewall a TCP connection is made to a SOCKS proxy server
(running by default on port 1080), the client authenticates itself with the server,
and if authenticated it sends the SOCKS server a \defn{relay request}
for performing TCP or UDP communication with a stated external server.
A SOCKS proxy can be specified in Java through the system properties
\begin{code}socksProxyHost\end{code} and \begin{code}socksProxyPort\end{code}.

If a firewall does not allow application data to be communicated directly
using a protocol such as TCP then an HTTP tunnel can be set up.
An \defn{HTTP tunnel} wraps application data inside HTTP packets which are usually
allowed to pass through most firewalls.
The HTTP packets are then processed by a \defn{mediator server} outside the firewall
which converts the data back to the original application data and forwards
it to the intended server. The mediator also wraps the response from the server
and sends it back as an HTTP response to the application.

As an application of tunneling through a firewall, note that
the class \begin{code}SSL\-EchoClient\end{code} from Section \ref{Se:WebSecurity}
cannot itself connect to an external SSL host from behind a firewall.
However, the example class \begin{code}SSLTunnelSocketFactory\end{code} can
be used as its \begin{code}SSLSocketFactory\end{code} to create an
\begin{code}SSLSocket\end{code} for accessing an SSL host via a proxy tunnel
through the firewall.
First, a TCP connection is made to the proxy server using a normal \begin{code}Socket\end{code}
and a \begin{code}CONNECT\end{code} is sent to the proxy,
followed by the name of the SSL host, a colon, and the port of the SSL host.
A \begin{code}CONNECT\end{code} method is an HTTP method that requests the proxy
to just forward the communication to the specified host and pass the response back.
If the proxy agrees to the request (possibly requiring authentication of the client
via a Base 64 encoded user name and password) then an \begin{code}SSLSocket\end{code} is
layered on top of the \begin{code}Socket\end{code}.
Hence the proxy will be unable to inspect the communication it is forwarding
between the client and the SSL host.

\begin{figure*}[t]\begin{center}
		\begin{pspicture}(-6,-5.5)(6,5.5)
		\pscircle[fillcolor=magenta!20,fillstyle=solid](0,0){5.5}
		\pscircle[fillcolor=white,fillstyle=solid](0,0){5}
		\psline[fillcolor=yellow!60,fillstyle=solid]%
		(-5,-0.25)(5,-0.25)(5,0.25)(-5,0.25)(-5,-0.25)
		\psline[fillcolor=yellow!60,fillstyle=solid]%
		(-0.25,-5)(-0.25,5)(0.25,5)(0.25,-5)(-0.25,-5)
		\psline[fillcolor=yellow!60,fillstyle=solid]%
		(-4.205,-2.716)(4.455,2.284)(4.205,2.716)(-4.455,-2.284)(-4.205,-2.716)
		\psline[fillcolor=yellow!60,fillstyle=solid]%
		(-4.205,2.716)(4.455,-2.284)(4.205,-2.716)(-4.455,2.284)(-4.205,2.716)
		\psline[fillcolor=yellow!60,fillstyle=solid]%
		(-2.284,-4.455)(2.716,4.205)(2.284,4.455)(-2.716,-4.205)(-2.284,-4.455)
		\psline[fillcolor=yellow!60,fillstyle=solid]%
		(-2.284,4.455)(2.716,-4.205)(2.284,-4.455)(-2.716,4.205)(-2.284,4.455)
		\pscircle[fillcolor=green!30,fillstyle=solid](0,0){2.25}
		\pscircle[fillcolor=cyan,fillstyle=solid](0,0){1.75}
		\pscircle[fillcolor=red!50,fillstyle=solid](0,0){1.25}
		\pscircle[fillcolor=white,fillstyle=solid](0,0){0.75}
		\psset{linestyle=none}
		\rput(0,0.15){\rnode{S}{Applic}}
		\rput(0,-0.15){\rnode{S}{Service}}
		\pstextpath[c]{\psarcn(0,0){1}{270}{-90}}{Hardened Operating System}
		\pstextpath[c]{\psarcn(0,0){1.5}{180}{0}}{Configuration}
		\pstextpath[c]{\psarcn(0,0){2}{180}{0}}{Identity Management}
		\pstextpath[c]{\psarcn(0,0){5.25}{180}{0}}{Perimeter Security}
		\pstextpath[c]{\psline(2.25,0)(5,0)}{Auditing}
		\pstextpath[c]{\psline(1.949,1.125)(4.33,2.5)}{Authentication}
		\pstextpath[c]{\psline(1.125,1.949)(2.5,4.33)}{Authorization}
		\pstextpath[c]{\psline(0,2.25)(0,5)}{Availability}
		\pstextpath[c]{\psline(-2.5,4.33)(-1.125,1.949)}{Compliance}
		\pstextpath[c]{\psline(-4.33,2.5)(-1.949,1.125)}{Confidentiality}
		\pstextpath[c]{\psline(-5,0)(-2.25,0)}{Integrity}
		\pstextpath[c]{\psline(-4.33,-2.5)(-1.949,-1.125)}{Labelling}
		\pstextpath[c]{\psline(-2.5,-4.33)(-1.125,-1.949)}{Logging}
		\pstextpath[c]{\psline(0,-2.25)(0,-5)}{Management}
		\pstextpath[c]{\psline(1.125,-1.949)(2.5,-4.33)}{Policy}
		\pstextpath[c]{\psline(1.949,-1.125)(4.33,-2.5)}{PKI}
		\end{pspicture}\end{center}
\end{figure*}

A \defn{security wheel}
(taken from the book \emph{Core Security Patterns} by Steel, Nagappan, Lai)
illustrates all the security components for a secure system.
At the hub of the wheel is the business logic of the application or service that
must have security incorporated into its design.
Surrounding this is a hardened operating system (an
operating system that has its own security with unnecessary features removed),
which is securely configured to provide reliable provisioning mechanisms,
and which is protected by an identity management system.
The spokes of the security wheel represent the 12 core security services
that are incorporated:
\begin{description}
	\item[auditing] which provides regular records about the application or
	service activity, to support forensic investigations and regulatory compliance,
	\item[authentication] which verifies the identity of a subject,
	\item[authorization] which determines whether a subject is permitted to
	access the various features provided by the application or service,
	\item[availability] which ensures reliable and timely access to the application or service,
	\item[compliance] which ensures that standards or regulatory requirements
	are met,
	\item[confidentiality] which ensures information is safe from unauthorized access
	during transmission and storage,
	\item[integrity] which ensures information is not tampered with
	by unauthorized subjects,
	\item[labelling] which appropriately classifies information according to
	its security level to prevent unauthorized disclosure,
	\item[logging] which provides records of events for the diagnosis of problems,
	\item[management] which provides mechanisms for the central administration
	of security operations,
	\item[policy] which provides rules and procedures for access control,
	\item[public key infrastructure] which provides key management and distribution facilities.
\end{description}
All the spokes must be in place to ensure a robust security architecture.
Beyond the spokes is the perimeter security, consisting of a firewall along with
intrusion prevention and detection systems.
Note that it is essential when creating a secure system that all these security
components are considered throughout the development process rather than attempts
made to add security after the application or service has been developed.




\begin{exercise}[Proxy Tunneling Through a Firewall]
	Implement a simple firewall that allows proxy tunneling to an external SSL host
	and test it using a modified \begin{code}SSLEchoClient\end{code}.
\end{exercise}


\begin{lstlisting}[caption=SSL tunneling]
/**
   A factory class that can be used to create SSL sockets which use tunneling of SSL through a proxy. Adapted from JSSE samples
   @author Andrew Ensor
*/
import java.io.InputStream;
import java.io.IOException;
import java.io.OutputStream;
import java.io.UnsupportedEncodingException;
import java.net.InetAddress;
import java.net.Socket;
import java.net.UnknownHostException;
import java.util.ArrayList;
import java.util.List;
import javax.net.ssl.HandshakeCompletedEvent;
import javax.net.ssl.HandshakeCompletedListener;
import javax.net.ssl.SSLSocket;
import javax.net.ssl.SSLSocketFactory;
import sun.misc.BASE64Encoder; //undocumented Sun&IBM VM utility class

public class SSLTunnelSocketFactory extends SSLSocketFactory
{
   private String tunnelHost;
   private int tunnelPort;
   private String tunnelUserName;
   private String tunnelPassword;//should not store password as String
   private SSLSocketFactory defaultFactory;
   private boolean socketConnected;

   public SSLTunnelSocketFactory(String tunnelHost, int tunnelPort,
      String tunnelUserName, String tunnelPassword)
   {  this.tunnelHost = tunnelHost;
      this.tunnelPort = tunnelPort;
      this.tunnelUserName = tunnelUserName;
      this.tunnelPassword = tunnelPassword;
      defaultFactory = (SSLSocketFactory)SSLSocketFactory.getDefault();
      socketConnected = false;
   }

   public SSLTunnelSocketFactory(String tunnelHost, int tunnelPort)
   {  this(tunnelHost, tunnelPort, null, null);
   }

   public String[] getSupportedCipherSuites()
   {  return defaultFactory.getSupportedCipherSuites();
   }
   public String[] getDefaultCipherSuites()
   {  return defaultFactory.getDefaultCipherSuites();
   }

   public synchronized boolean isSocketConnected()
   {  return socketConnected;
   }

   public synchronized void setSocketConnected(boolean socketConnected)
   {  this.socketConnected = socketConnected;
   }

   public Socket createSocket(Socket s, String host, int port,
      boolean autoClose) throws IOException, UnknownHostException
   {  // create a regular socket to tunnel through the proxy host
      Socket tunnel = new Socket(tunnelHost, tunnelPort);
      doTunnelHandshake(tunnel, host, port);
      // layer an SSL socket over the top of the regular socket
      SSLSocket sslSocket = (SSLSocket)
         defaultFactory.createSocket(tunnel, host, port, autoClose);
      sslSocket.addHandshakeCompletedListener(
         new HandshakeCompletedListener()
         {  public void handshakeCompleted(HandshakeCompletedEvent e)
            {  System.out.println("Handshake completed with peer host "
                  + e.getSession().getPeerHost()
                  + " and assigned session ID " + e.getSession());
               setSocketConnected(true);
            }
         });
      return  sslSocket;
   }

   private void doTunnelHandshake(Socket tunnel, String host, int port)
      throws IOException
   {  OutputStream os = tunnel.getOutputStream();
      String connectionString = "CONNECT " + host + ":" + port
         + " HTTP/1.0\n" + "User-Agent: "
         + sun.net.www.protocol.http.HttpURLConnection.userAgent
         + "\r\n";
      if (tunnelUserName!=null  && tunnelPassword!=null)
      {  // add the Base 64 encoded user name and password
         BASE64Encoder encoder = new BASE64Encoder();
         String encodedString = encoder.encode((tunnelUserName + ":"
            + tunnelPassword).getBytes());
         connectionString += "Proxy-Authorization: Basic "
            + encodedString + "\r\n";
      }
      connectionString += "Content-Length: 0\r\n"
         + "Pragma: no-cache\r\n\r\n";
      // send the connection HTTP request using ASCII7 encoding
      byte[] connectionBytes;
      try
      {  connectionBytes = connectionString.getBytes("ASCII7");
      }
      catch (UnsupportedEncodingException e)
      {  connectionBytes = connectionString.getBytes();
      }
      os.write(connectionBytes);
      os.flush();
      // obtain the connection reply
      InputStream is = tunnel.getInputStream();
      List<Byte> responseList=new ArrayList<Byte>(); //response header
      int newLinesSeen = 0;
      boolean headerDone = false;
      while (newLinesSeen < 2)
      {  int data = is.read();
         if (data < 0)
            throw new IOException("Unexpected end of response");
         if (data == '\n')
         {  headerDone = true;
            newLinesSeen++;
         }
         else if (data != '\r')
         {  newLinesSeen = 0;
            if (!headerDone)
               responseList.add((byte)data);
         }
      }
      // convert the HTTP response into a string using ASCII7 encoding
      int responseLength = responseList.size();
      byte[] responseBytes = new byte[responseLength];
      for (int i=0; i<responseLength; i++)
         responseBytes[i] = responseList.get(i);
      String responseString;
      try
      {  responseString = new String(responseBytes, 0, responseLength,
            "ASCII7");
      }
      catch (UnsupportedEncodingException e)
      {  responseString = new String(responseBytes, 0, responseLength);
      }
      // check that response was successful
      if (responseString.toLowerCase().indexOf
         ("200 connection established")<0)
         throw new IOException("Unable to tunnel through "+tunnelHost
            + ":" + tunnelPort + ". Proxy returned " + responseString);
   }
   // overridden method of SocketFactory
   public Socket createSocket(InetAddress host, int port)
      throws IOException, UnknownHostException
   {  return createSocket(null, host.getHostName(), port, true);
   }

   // overridden method of SocketFactory
   public Socket createSocket(InetAddress address, int port,
      InetAddress localAddress, int localPort)
      throws IOException, UnknownHostException
   {  return createSocket(null, address.getHostName(), port, true);
   }

   // overridden method of SocketFactory
   public Socket createSocket(String host, int port)
      throws IOException, UnknownHostException
   {  return createSocket(null, host, port, true);
   }

   // overridden method of SocketFactory
   public Socket createSocket(String host, int port,
      InetAddress localHost, int localPort)
      throws IOException, UnknownHostException
   {  return createSocket(null, host, port, true);
   }
}
\end{lstlisting}

